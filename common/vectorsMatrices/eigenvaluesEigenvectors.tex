\subsection{Eigenvalues \& Eigenvectors}
\begin{definition}
	Let $A$ be an $n \times n$ matrix. A scalar $\lambda$ and a vector $\vec{v}$ are an eigenvalue and eigenvector of $A$ if
	\begin{equation*}
		A\vec{v} = \lambda\vec{v}.
	\end{equation*}
\end{definition}
\noindent
We call $p(\lambda) = \det{(A - \lambda I)}$ the characteristic polynomial of $A$. The eigenvalues for $A$ are the solutions to the equation
\begin{equation*}
	p(\lambda) = \det{(A - \lambda I)} = 0.
\end{equation*}
Once we have an eigenvalue, we can find the basis vectors for the corresponding eigenspace by solving the equation
\begin{equation*}
	\left( A - \lambda I \right)\vec{v} = \vec{0}.
\end{equation*}
The basis vectors of the eigenspace for $A$ are the union of the basis vectors of each eigenspace corresponding to each eigenvalue.

\begin{example}
	Find the eigenvalues and eigenvectors of
	\begin{equation*}
		A = \begin{bmatrix}
			2 & 1 & 3 \\
			1 & 2 & 3 \\
			3 & 3 & 20
		\end{bmatrix}.
	\end{equation*}
\end{example}
\begin{equation*}
	p(\lambda) = \begin{array}{|ccc|}
		2-\lambda & 1 & 3 \\
		1 & 2-\lambda & 3 \\
		3 & 3 & 20-\lambda
	\end{array} = 
	-(\lambda - 21)(\lambda - 2)(\lambda - 1) = 0 \implies \lambda = 1 \text{, } 2 \text{, and } 21.
\end{equation*}
When $\lambda = 1$,
\begin{equation*}
	A-\lambda I = \left[
	\begin{array}{ccc|c}
		2-1 & 1 & 3 & 0 \\
		1 & 2-1 & 3 & 0 \\
		3 & 3 & 20-1 & 0
	\end{array} 
	\right] \to \left[
	\begin{array}{ccc|c}
		1 & 1 & 0 & 0 \\
		0 & 0 & 1 & 0 \\
		0 & 0 & 0 & 0
	\end{array}
	\right] \implies \vec{v_{1}} = t \begin{bmatrix}
		-1 \\
		1 \\
		0
	\end{bmatrix}.
\end{equation*}
When $\lambda = 2$,
\begin{equation*}
	A-\lambda I = \left[
	\begin{array}{ccc|c}
		2-2 & 1 & 3 & 0 \\
		1 & 2-2 & 3 & 0 \\
		3 & 3 & 20-2 & 0
	\end{array} 
	\right] \to \left[
	\begin{array}{ccc|c}
		1 & 0 & 3 & 0 \\
		0 & 1 & 3 & 0 \\
		0 & 0 & 0 & 0
	\end{array}
	\right] \implies \vec{v_{2}} = \begin{bmatrix}
		-3 \\
		-3 \\
		1
	\end{bmatrix}.
\end{equation*}
When $\lambda = 21$,
\begin{equation*}
	A-\lambda I = \left[
	\begin{array}{ccc|c}
		2-21 & 1 & 3 & 0 \\
		1 & 2-21 & 3 & 0 \\
		3 & 3 & 20-21 & 0
	\end{array} 
	\right] \to \left[
	\begin{array}{ccc|c}
		1 & 0 & -1/6 & 0 \\
		0 & 1 & -1/6 & 0 \\
		0 & 0 & 0 & 0
	\end{array}
	\right] \implies \vec{v_{21}} = \begin{bmatrix}
		1 \\
		1 \\
		6
	\end{bmatrix}.
\end{equation*}
Bonus: $A$'s diagonalization is
\begin{equation*}
	A = PDP^{-1} \implies \begin{bmatrix}
		2 & 1 & 3 \\
		1 & 2 & 3 \\
		3 & 3 & 20
	\end{bmatrix} = \begin{bmatrix}
		-1 & -3 & 1 \\
		1 & -3 & 1 \\
		0 & 1 & 6
	\end{bmatrix} \begin{bmatrix}
		1 & 0 & 0 \\
		0 & 2 & 0 \\
		0 & 0 & 21
	\end{bmatrix} \begin{bmatrix}
		-1/2 & 1/2 & 0 \\
		-3/19 & -3/19 & 1/19 \\
		1/38 & 1/38 & 3/19
	\end{bmatrix}.
\end{equation*}