\subsection{Determinants}
\noindent
The determinant of a matrix is a signed number that tells by how much the transformation represented by a matrix scales volumes in a space.
The number is negative if the space was ``flipped" during a transformation.
The number is zero if the dimension of the output space is less than that of the input space.\\

\noindent
The determinant is only defined for square matrices. It's easiest to understand the definition of a determinant recursively.
\begin{align*}
	\det{\left[ a \right]} &= \lvert a \rvert = a \\
	\det{\left[
		\begin{array}{cc}
			a & b \\
			c & d
		\end{array}
		\right]} &= \begin{array}{|cc|}
		a & b \\
		c & d
	\end{array} = ad - bc.
\end{align*}
We can define $a_{ij}$ as the entry in the ith row and jth column of matrix $A$ and $A_{ij}$ as the adjudicate matrix, which is the matrix $A$ if row $i$ and column $j$ were removed. This allows us to write a general formula for the determinant.
\begin{definition}
	\begin{equation*}
		\det{A} = \sum_{j=1}^{n}{\left(-1\right)^{i+j}a_{ij}A_{ij}} \text{ (for fixed i)} = \sum_{i=1}^{n}{\left(-1\right)^{i+j}a_{ij}A_{ij}} \text{ (for fixed j)}
	\end{equation*}
\end{definition}
\noindent
This formula allows us to use any row or column to calculate the determinant, which is especially useful if a certain row contains lots of 0's.\\

\noindent
Below are some properties of the determinant for some $n \times n$ matrix $A$ and scalar $\lambda$.
\begin{align*}
	\det{I_n} &= 1 \\
	\det{(A^T)} &= \det{A} \\
	\text{If $A$ is invertible, } \det{(A^{-1})} &= \frac{1}{\det{A}} \\
	\det{(\lambda A)} &= \lambda^n\det{A} \\
	\det{(AB)} &= \det{A}\det{B} \\
	\text{If $A$ is triangular, } \det{A} &= \prod_{i=1}^{n}{a_{ii}}
\end{align*}

\begin{example}
	Find the determinant of the following 3 x 3 matrix.
	\begin{equation*}
		A = \begin{bmatrix}
			1 & 3 & 7 \\
			0 & 2 & -1 \\
			2 & 7 & 9
		\end{bmatrix}
	\end{equation*}
\end{example}
\noindent
We'll use the first column since it has only two non-zero entries.
\begin{equation*}
	\begin{bmatrix}
		1 & 3 & 7 \\
		0 & 2 & -1 \\
		2 & 7 & 9
	\end{bmatrix} = 1 \text{ } \begin{array}{|cc|}
		2 & -1 \\
		7 & 9
	\end{array} + 2 \text{ } \begin{array}{|cc|}
		3 & 7 \\
		2 & -1
	\end{array} = (18+7) + 2(-3-14) = -9.
\end{equation*}