\subsection{Trig Identities}
\noindent
As we could see in Figure \ref{unitCircle}, $\sin$ and $\cos$ form a right triangle with hypotenuse 1. So, using the Pythagorean Theorem,
\begin{equation*}
	\sin^2{\theta} + \cos^2{\theta} = 1.
\end{equation*}
By dividing by $\sin^2$ or $\cos^2$, we can also get
\begin{equation*}
	1 + \cot^2{\theta} = \csc^2{\theta} \text{ and } \tan^2{\theta} + 1 = \sec^2{\theta}.
\end{equation*}
Together, these 3 identities are called the Pythagorean Identities.\\

\noindent
We can also relate functions and co-functions.
\begin{equation*}
	\text{xxx}(\theta) = \text{coxxx}\left(\frac{\pi}{2} - \theta\right).
\end{equation*}

\noindent
Some of the most useful and used identities are the sum and difference.
\begin{align*}
	\sin{\left(\alpha \pm \beta\right)} &= \sin{\alpha}\cos{\beta} \pm \cos{\alpha}\sin{\beta} \\
	\cos{\left(\alpha \pm \beta\right)} &= \cos{\alpha}\cos{\beta} \mp \sin{\alpha}\sin{\beta} \\
	\tan{\left(\alpha \pm \beta\right)} &= \frac{\tan{\alpha} \pm \tan{\beta}}{1 \mp \tan{\alpha}\tan{\beta}} \\
	\sin{\alpha} \pm \sin{\beta} &= 2\sin{\left(\frac{\alpha \pm \beta}{2}\right)}\cos{\left(\frac{\alpha \mp \beta}{2}\right)} \\
	\cos{\alpha} + \cos{\beta} &= 2\cos{\left(\frac{\alpha + \beta}{2}\right)}\cos{\left(\frac{\alpha - \beta}{2}\right)} \\
	\cos{\alpha} - \cos{\beta} &= -2\sin{\left(\frac{\alpha + \beta}{2}\right)}\sin{\left(\frac{\alpha - \beta}{2}\right)} \\
\end{align*}