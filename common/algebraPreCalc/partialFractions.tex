\subsection{Partial Fractions}
\noindent
If we have a function of two polynomials $f(x) = \frac{P(x)}{Q(x)}$, it's often easier to break this quotient into a sum of parts where the denominator is a linear or quadratic factor and the numerator is always a smaller degree than the denominator.

\begin{example}
	\begin{equation*}
		\frac{2x-1}{x^3-6x^2+11x-6} = \frac{1/2}{x-1}+\frac{-3}{x-2}+\frac{5/2}{x-3}.
	\end{equation*}
\end{example}

\noindent
One natural way to find these small denominators comes from the linear factors of the denominator where we keep quadratics with complex roots.
This way, when making a common denominator, we get back the original big denominator.
However, there are a few special cases we have to take care of.

\subsubsection{Linear Factors}
\noindent
This is the the most basic type where the degree of the numerator is less than the degree of the denominator and the denominator factors into all linear factors with no repeated roots. In this case we can write
\begin{equation*}
	\frac{P(x)}{Q(x)} = \frac{A_1}{(x-a_1)} + \ldots + \frac{A_n}{(x-a_n)}.
\end{equation*}

\noindent
Multiplying each side by $Q(x)$,
\begin{equation*}
	P(x) = A_1(x-a_2) \ldots (x-a_n) + \ldots + A_n(x-a_1) \ldots (x-a_{n-1}).
\end{equation*}

\noindent
We can then find each $A_i$ by evaluating both sides at $x=a_i$, since every term except the ith has an $(x-a_i)$ factor that will go to 0. So,
\begin{equation*}
	A_i = \frac{P(a_i)}{(x-a_i) \ldots (x-a_{i-1})(x-a_{i+1}) \ldots (x-a_n)}.
\end{equation*}

\begin{example}
	Find the partial fraction decomposition of the following expression:
	\begin{equation*}
		\frac{2x-1}{x^3-6x^2+11x-6}.
	\end{equation*}
\end{example}
\noindent
Factoring,
\begin{equation*}
	x^3 - 6x^2 + 11x - 6 = (x-1)(x-2)(x-3).
\end{equation*}
So,
\begin{equation*}
	\frac{2x-1}{x^3-6x^2+11x-6} = \frac{A_1}{x-1}+\frac{A_2}{x-2}+\frac{A_3}{x-3}.
\end{equation*}
Multiplying each side by the denominator,
\begin{equation*}
	2x-1 = A_1(x-2)(x-3)+A_2(x-1)(x-3)+A_3(x-1)(x-2).
\end{equation*}
At $x=1$,
\begin{equation*}
	1 = A_1(1-2)(1-3) \implies A_1 = \frac{1}{2}.
\end{equation*}
At $x=2$,
\begin{equation*}
	3 = A_2(2-1)(2-3) \implies A_2 = -3.
\end{equation*}
At $x=3$,
\begin{equation*}
	5 = A_3(3-1)(3-2) \implies A_3 = \frac{5}{2}.
\end{equation*}
So,
\begin{equation*}
	\frac{2x-1}{x^3-6x^2+11x-6} = \frac{1/2}{x-1} + \frac{-3}{x-2} + \frac{5/2}{x-3},
\end{equation*}
just as was shown in the previous example.
\subsubsection{Repeated Linear Factors}
If $Q(x)$ has repeated roots, it factors into
\begin{equation*}
	Q(x) = R(x)(x-a)^k\text{, }k \geq 2\text{ and }R(a) \neq 0.
\end{equation*}

\noindent
When making the common denominator for each repeated root of multiplicity $k$, we do
\begin{equation*}
	\frac{P(x)}{R(x)(x-a)^k} = \left(\text{Decomposition of }R(x)\right) + \frac{A_1}{x-a}+\ldots+\frac{A_k}{(x-a)^k}.
\end{equation*}

\noindent
You would then multiply each side by the denominator like in the linear factors case and solve for the coefficients.
The only additional difficulty is that you might have to use previous results or solve a system of linear equations to get some of the constants.

\begin{example}
	Find the partial fraction of the following expression:
	\begin{equation*}
		\frac{x^2+5x-6}{x^3-7x^2+16x-12}.
	\end{equation*}
\end{example}
\noindent
Factoring,
\begin{equation*}
	x^3-7x^2+16x-12 = (x-3)(x-2)^2.
\end{equation*}
So,
\begin{equation*}
	\frac{x^2+5x-6}{x^3-7x^2+16x-12} =  \frac{A_1}{x-3} + \frac{A_2}{x-2} + \frac{A_3}{(x-2)^2}.
\end{equation*}
Multiplying each side by the denominator,
\begin{equation*}
	x^2+5x-6 = A_1(x-2)^2 + A_2(x-2)(x-3) + A_3(x-3).
\end{equation*}
At $x=2$,
\begin{equation*}
	8 = A_3(2-3) \implies A_3 = -8.
\end{equation*}
At $x=3$,
\begin{equation*}
	18 = A_1(3-2)^2 \implies A_1 = 18.
\end{equation*}
Now we'll use our results for $A_1$ and $A_3$ to find $A_2$ using a value for $x$ that isn't 2 or 3 so the $A_2$ term doesn't become 0. A good choice is $x=0$.\\
At $x=0$,
\begin{equation*}
	-6 = 18(0-2)^2 + A_2(0-2)(0-3) + -8(0-3) \implies A_2 = -17.
\end{equation*}
So,
\begin{equation*}
	\frac{x^2+5x-6}{x^3-7x^2+16x-12} = \frac{18}{x-3} - \frac{17}{x-2} - \frac{8}{(x-2)^2}.
\end{equation*}
\subsubsection{Quadratic Factors}
\noindent
If a quadratic doesn't have real roots, then we have a quadratic factor. Here, we'll assume that the quadratic factor isn't repeated. 
So, $Q(x) = R(x)(ax^2+bx+c)$, $b^2-4ac < 0$, and $R(x)$ is not evenly divisible by $ax^2+bx+c$.
In this case, we say
\begin{equation*}
	\frac{P(x)}{R(x)(ax^2+bx+c)} = \left(\text{Decomposition of }R(x)\right)+\frac{A_1x+B_1}{ax^2+bx+c}.
\end{equation*}
We then solve for the constants in the numerator, possibly having to solve a system of equations or using previous results and less convenient values for $x$.

\begin{example}
	Find the partial fraction decomposition of the following expression:
	\begin{equation*}
		\frac{6x^2+21x+11}{x^3+5x^2+3x+15}.
	\end{equation*}
\end{example}
\noindent
Factoring,
\begin{equation*}
	x^2+5x^2+3x+15 = (x+5)(x^2+3).
\end{equation*}
So,
\begin{equation*}
	\frac{6x^2+21x+11}{x^3+5x^2+3x+15} = \frac{A_1}{x+5}+\frac{A_2x+B_2}{x^2+3}.
\end{equation*}
Multiplying each side by the denominator,
\begin{equation*}
	6x^2+21x+11 = A_1(x^2+3)+(A_2x+B_2)(x+5).
\end{equation*}
At $x=-5$,
\begin{equation*}
	56 = 28A_1 \implies A_1 = 2.
\end{equation*}
Now we'll use the previous result and another value for $x$. We can use $x=0$ to not have to worry about the $A_2$ term.
At $x=0$,
\begin{equation*}
	11 = 2(3) + (B_2)(5) \implies B_2 = 1.
\end{equation*}
Now we'll use the previous 2 results to find $A_2$. $x=1$ is a good choice to keep the numbers small.
At $x=1$,
\begin{equation*}
	38 = 2(1+3)+(A_2+1)(6) \implies A_2 = 4.
\end{equation*}
So,
\begin{equation*}
	\frac{6x^2+21x+11}{x^3+5x^2+3x+15} = \frac{2}{x+5}+\frac{4x+1}{x^2+3}.
\end{equation*}

\subsubsection{Repeated Quadratic Factors}
\noindent
If a quadratic factor that can't be broken into linear factors is repeated, then we can write
$Q(x) = R(x)(ax^2+bx+c)^k$, $k \geq 0$, and $R(x)$ is not divisible by $(ax^2+bx+c)^k$.
Now we have to do a combination of what we did for repeated linear factors and quadratic factors.
We say
\begin{equation*}
	\frac{P(x)}{R(x)(ax^2+bx+c)^k}=\left(\text{Decomposition of }R(x)\right)+\frac{A_1x+B_1}{ax^2+bx+c}+\ldots+\frac{A_kx+B_k}{(ax^2+bx+c)^k}.
\end{equation*}
We then solve for the coefficients in the numerator.

\begin{example}
	Find the partial fraction decomposition of $\frac{3x^4-2x^3+6x^2-3x+3}{x^5+3x^4+4x^3+12x^2+4x+12}$.
\end{example}
\noindent
Factoring,
\begin{equation*}
	x^5+3x^4+4x^3+12x^2+4x+12 = (x+3)(x^2+2)^2.
\end{equation*}
So,
\begin{equation*}
	\frac{3x^4-2x^3+6x^2-3x+3}{x^5+3x^4+4x^3+12x^2+4x+12} = \frac{A_1}{x+3}+\frac{A_2x+B_2}{x^2+2}+\frac{A_3x+B_3}{(x^2+2)^2}.
\end{equation*}
Multiplying each side by the denominator,
\begin{equation*}
	3x^4-2x^3+6x^2-3x+3 = A_1(x^2+2)^2+(A_2x+B_2)(x^2+2)(x+3)+(A_3x+B_3)(x+3).
\end{equation*}
At $x=-3$,
\begin{equation*}
	363 = 121A_1 \implies A_1 = 3.
\end{equation*}
Now, we'll use our result for $A_1$ and pick a value for $x$ that minimizes the number of things we need to solve for. We'll have to solve a linear system with 4 unknowns, so we'll need up to 4 values.
At $x=0$,
\begin{equation*}
	3 = 3(2)^2 + B_2(2)(3)+B_3(3) \implies 2B_2 + B_3 = -3.
\end{equation*}
At $x=1$,
\begin{equation*}
	7 = 3(3)^2 + (A_2+B_2)(3)(4) + (A_3+B_3)(4) \implies 3A_2 + A_3 + 3B_2 + B_3 = -5.
\end{equation*}
At $x=-1$,
\begin{equation*}
	17 = 3(3)^2 + (-A_2+B_2)(3)(2) + (-A_3+B_3)(2) \implies -3A_2 - A_3 + 3B_2 + B_3 = -5.
\end{equation*}
At $x=2$,
\begin{equation*}
	53 = 3(6)^2 + (2A_2+B_2)(6)(5) + (2A_3+B_3)(5) \implies 12A_2 + 2A_3 + 6B_2 + B_3 = -11.
\end{equation*}
Now we have the following system of equations:
\begin{equation*}
	\begin{cases}
		0A_2  + 0A_3 + 2B_2 + B_3 &= -3 \\
		3A_2  + A_3  + 3B_2 + B_3 &= -5 \\
		-3A_2 - A_3  + 3B_2 + B_3 &= -5 \\
		12A_2 + 2A_3 + 6B_2 + B_3 &= -11
	\end{cases}.
\end{equation*}
Solving,
\begin{equation*}
	A_2 = 0, A_3 = 0, B_2 = -2, \text{ and } B_3 = 1.
\end{equation*}
So,
\begin{equation*}
	\frac{3x^4-2x^3+6x^2-3x+3}{x^5+3x^4+4x^3+12x^2+4x+12} = \frac{3}{x+3}-\frac{2}{x^2+2}+\frac{1}{(x^2+2)^2}.
\end{equation*}
\subsubsection{Improper Fractions}
\noindent
If the degree of the numerator is greater than or equal to the degree of the denominator, we have a case of improper fractions.
In this case, we have to do polynomial long division to get a quotient and remainder and then decompose the remainder if necessary.
So,
\begin{equation*}
	\frac{P(x)}{Q(x)} = R(x) + \frac{S(x)}{Q(x)}.
\end{equation*}

\begin{example}
	Find the partial fraction decomposition of the following expression:
	\begin{equation*}
		\frac{x^3+3}{x^2-2x-3}.
	\end{equation*}
\end{example}
\noindent
First we do polynomial long division to find that
\begin{equation*}
	\frac{x^3+3}{x^2-2x-3} = x + 2 + \frac{7x+9}{x^2-2x-3}.
\end{equation*}
Now that the numerator is of a lesser degree than the denominator, we can decompose it normally.
\begin{equation*}
	x^2-2x-3 = (x-3)(x+1).
\end{equation*}
So,
\begin{equation*}
	\frac{7x+9}{x^2-2x-3} = \frac{A_1}{x-3} + \frac{A_2}{x+1}.
\end{equation*}
Multiplying each side by the denominator,
\begin{equation*}
	7x+9 = A_1(x+1) + A_2(x-3).
\end{equation*}
At $x=-1$,
\begin{equation*}
	2 = -4A_2 \implies A_2 = \frac{-1}{2}.
\end{equation*}
At $x=3$,
\begin{equation*}
	30 = 4A_1 \implies A_1 = \frac{15}{2}.
\end{equation*}
So,
\begin{equation*}
	\frac{x^3+3}{x^2-2x-3} = x + 2 + \frac{15/2}{x-3} + \frac{-1/2}{x+1}.
\end{equation*}
