\subsection{Sets}
\begin{definition}
	A set $A$ is a collection of distinct elements. Those elements can be anything, like numbers, functions, and even other sets.
\end{definition}
We can define a set by giving its elements, like $A = \{-2, 5, 3\}$ or by describing its properties, like $A = \{x \mid x > 0\}$ where the vertical bar means "such that".
If an object $x$ is a member of the set $A$, we write $x\in A$.\bigskip


\noindent
A set $A$ is called a subset of a set $B$ if every element of $A$ is also an element of $B$. 
We can write this as $A \subseteq B$. 
For example, $\{7, 10, 16\} \subseteq \{5, 6, 7, 9, 10, 11, 16\}$. 
Note that this relation can be strict if there exists at least one element in $B$ that is not also an element of $A$. 
Some common sets and their informal definitions are given below:

\begin{table}[H]
	\centering
	\begin{tabular}{c|c|c}
		Set Name & Symbol & Informal Definition                                                                                                                         \\ \hline
		Natural numbers & $\mathbb{N}$ & $\{1, 2, 3, \dots\}$           						                                                                        \\
		Integers & $\mathbb{Z}$ & $\{\dots, -3, -2, -1, 0, 1, 2, 3, \dots\}$                                                                                            \\
		Rational numbers & $\mathbb{Q}$ & $\{\frac{m}{n} \mid m,n \in\mathbb{Z}$ and $n \neq 0\}$                                                                       \\
		Real numbers & $\mathbb{R}$ & Any number on the number line\footnote{Sadly, there is no way to give a stronger formal definition without higher mathematics.}   \\
	\end{tabular}
\end{table}

This means that $\mathbb{N} \subset \mathbb{Z} \subset \mathbb{Q} \subset \mathbb{R}$.\bigskip


\noindent
There are several common operations that can be performed on sets.
The union $A \cup B$ of two sets $A$ and $B$ is the set of all elements that are elements of $A$ or of $B$. 
Similarly, the intersection $A \cap B$ of two sets $A$ and $B$ is the set of all elements that are also elements of both $A$ and $B$.
\begin{example}
	If $A = \{\sqrt{2}, 2, 5, 8\}$ and $B = \{-9, 8, 2.3\}$, what are $A \cup B$ and $A \cap B$?		
\end{example}
To find the union, we combine the sets, making sure to include any repeated element only once:
\begin{equation*}
	A \cup B = \{-9, \sqrt{2}, 2, 2.3, 5, 8\}.
\end{equation*}
Then, since the only element both sets share is 8, we also have
\begin{equation*}
	A \cap B = \{8\}.
\end{equation*}

\subsection{Intervals}
\begin{definition}
	We call a subset $I$ of $\mathbb{R}$ an interval if, for any $a, b \in I$ and $x \in \mathbb{R}$ such that $a \leq x \leq b$, then $x \in I$. 
\end{definition}
We can write an interval more simply using the notation $[a, b]$, which is equivalent to $\{x \in \mathbb{R} \mid a \leq x \leq b\}$. 
This is called a closed interval, and to make the inequalities strict, we can also define an open interval by using parantheses instead of square brackets.\bigskip

In addition, we can mix the two to create half-open intervals, where one inequality is strict and the other isn't. 
For instance, $(2, 5]$ refers to the set $\{x \in \mathbb{R} \mid x < 2 \leq 5\}$
Finally, if the interval is unbounded in either direction, we use the notations $-\infty$ and $\infty$ to indicate that there is no minimum or maximum, respectively.

\begin{example}
	Is $8 \in (-\infty, 4) \cup [8, 100)$?
\end{example}
\begin{answer}
	Since $8 \leq 8 < 100$ is a true statement, $8 \in [8, 100)$. 
	Since we are taking the union with another set, all of the members of the right interval will also be members of the union of intervals. 
	Therefore, the statement is true.
\end{answer}