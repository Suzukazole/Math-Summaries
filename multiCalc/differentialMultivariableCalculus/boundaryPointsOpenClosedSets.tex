\subsection{Boundary Points, Open \& Closed Sets}
\noindent
Given some set $\Omega \subset \mathbb{R}^n$, $x$ is an interior point to $\Omega$ if there exists some $\delta$ such that $N(x, \delta) \subset \Omega$.
That is, $x$ is an interior point to $\Omega$ if you can draw a circle of non-zero radius around $x$ such that the entire circle is inside of $\Omega$.
All points that are not interior points are boundary points.
Formally, $x$ is a boundary point of $\Omega$ if for all $\delta$,  $N(x,\delta) \not\subset \Omega$.
Using our definitions of interior and boundary points, we can define and open set as one that doesn't contain any of its boundary points and a closed set as one that contains all of its boundary point. Note that a set that contains some of its boundary points is neither open nor closed.