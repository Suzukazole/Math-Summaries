\subsection{Linear Approximations}
\noindent
Since $\partial z = f_x\partial x + f_y\partial y$, we can approximate $\Delta z$ (the change in any function) as $\Delta z \approx f_x\Delta x + f_y\Delta y$, since values of $f$ and the tangent plane are close. We an rewrite this approximation as a dot product: $\Delta z \approx \langle f_x, f_y\rangle \cdot \langle \Delta x, \Delta y \rangle$.\\

For example, say a cylindrical can has a radius $r=1$ and a height $h=5$. If the radius is increased by .1 and the height is increased by 1, what is the approximate $\Delta V$?
\begin{equation*}
	V(r,h) = \pi r^2 h, V_r = 2\pi rh \text{, and } V_h = \pi r^2
\end{equation*}
\begin{equation*}
	V_{r}(1,5) = 10\pi  \text{ and } V_{h}(1,5) = \pi
\end{equation*}
\begin{equation*}
	\Delta V \approx 10\pi(.1) + \pi(1) = 2\pi	
\end{equation*}
Comparing this to the actual answer of $2.26\pi$, we see our approximation is decent.