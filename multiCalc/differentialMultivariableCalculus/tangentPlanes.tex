\subsection{Tangent Planes}
\noindent
Although the tangent lines at a point on a surface can all be different depending on the direction one approaches a point from, all of these tangent lines lie in the same plane, defining the tangent plane. This means that the tangent plane to $z=f(x_0,y_0)$ has the following properties:
\begin{itemize}
	\item The z-value of the tangent plane at $(x_0,y_0)$ is the same as $f(x_0,y_0)$.
	\item The value of the first-order partial derivatives of the tangent plane at $(x_0,y_0)$ should match those of $f(x_0,y_0)$.
\end{itemize}

\noindent
The general form of a plane at $(x_0,y_0,z_0)$ is $P(x,y)=A(x-x_0)+B(y-y_0)+z_0$.\\ 
We want $P_x=f_x$ and $P_y=f_y$.\\
This means that $P_x=f_x=A$ and $P_y=f_y=B$.\\
Rewriting, $P(x,y)=f_x(x-x0)+f_y(y-y_0)+z_0$.\\
The normal vector is $\langle \pm f_x,\pm f_y, \mp 1\rangle$.\\
So, the point normal form of the plane is $$\langle -f_x, -f_y, 1\rangle\cdot\langle x-x_0, y-y_0, z-f(x_0,y_0)\rangle=0$$.

[INSERT IMAGE]
