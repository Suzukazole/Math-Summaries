\subsection{Partial Derivatives of X, Y, and Z}
\noindent
It's common to look at the derivative when slicing a surface in the yz, xz, and xy planes. These are called partial derivatives.\\

\noindent
To compute $\frac{\partial}{\partial x}{f(x,y)}$, we take the derivative with respect to $x$ as if $y$ is constant. Formally,
$$\frac{\partial}{\partial x}f(x,y)=\lim_{h\to 0}{\frac{f(x+h,y)-f(x,y)}{h}}$$ and $$\frac{\partial}{\partial y}f(x,y)=\lim_{h\to 0}{\frac{f(x,y+h)-f(x,y)}{h}}$$

\noindent
We also use the shorthand $\frac{\partial}{\partial x}=f_x$ and $\frac{\partial}{\partial y}=f_y$. This shorthand can be extended to higher-order derivatives so that $\frac{\partial}{\partial y}\left(\frac{\partial}{\partial x}f(x,y)\right)=f_{xy}$.

\noindent
Fubini's Theorem (also called Tonelli's or Clairaut's Theorem) says $f_{xy}=f_{yx}$, $f_{xz}=f_{zx}$, and $f_{yz}=f_{zy}$. It extends into higher-order mixed partial derivatives, saying that two mixed partial derivatives of a function are equal as long as they both differentiate the same number of variables the same number of times. So, $f_{abcdab}=f_{aacdbb}$.