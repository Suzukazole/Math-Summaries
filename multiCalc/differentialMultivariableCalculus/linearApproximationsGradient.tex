\subsection{Linear Approximations with the Gradient}
\noindent
The 4th gradient property can be generalized a bit further. Suppose we have $f(x,y,z)$, $\vec{r}(u,v) = \langle x(u,v), y(u,v), z(u,v) \rangle$, and $g(u,v) = f\circ\vec{r}(u,v)$.
\begin{equation*}
	\frac{\partial g}{\partial u} = \frac{\partial f}{\partial x}\frac{\partial x}{\partial u} + \frac{\partial f}{\partial y}\frac{\partial y}{\partial u} + \frac{\partial f}{\partial z}\frac{\partial z}{\partial u}
\end{equation*}
\noindent
We can rewrite our linear approximation as $\Delta z\approx\nabla f\cdot\langle \Delta x, \Delta y\rangle$.\\
In fact, we how have a way to find derivatives of $f\circ\vec{r}(t)$.
\begin{equation*}
	\frac{\mathrm{d}f}{\mathrm{d}t} = \frac{\partial f}{\partial x}\frac{\mathrm{d}x}{\mathrm{d}t} + \frac{\partial f}{\partial y}\frac{\mathrm{d}y}{\mathrm{d}t} = \nabla f \cdot \langle x^{\prime}(t), y^{\prime}(t) \rangle
\end{equation*}