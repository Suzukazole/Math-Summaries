\subsection{Circles}
\noindent
You should already recognize $x^2 + y^2 = r^2$ as the equation of a circle with radius $r$ centered at the origin. A circle as a VVF is $\vec{r}(t) = \langle r\cos{t}, r\sin{t} \rangle$, which is identical to the parametric form of a circle.\\
In $\mathbb{R}^3$, the z-component is some constant that tells us which plane, $z=c$, the circle is in.\\ 
We can also have circles parallel to $x=0$ and $y=0$ planes by changing the positions of the $\sin$, $\cos$, and $c$ terms. For example, $\vec{r}(t) = \langle \cos{t}, c, \sin{t} \rangle$ is a circle in the $y=c$ plane.