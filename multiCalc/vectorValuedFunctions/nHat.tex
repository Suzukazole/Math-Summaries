\subsection{N-Hat $\left(\hat{N}\right)$}
\noindent
\begin{align*}
	\vec{r^{\prime}}(t) &= v(t)\hat{T}(t) \\
	\vec{r^{\prime\prime}}(t) &= v(t)\hat{T}^{\prime}(t)+\hat{T}(t)v^{\prime}(t)
\end{align*}

\noindent
We will show that $\hat{T}(t) \perp \hat{T}^{\prime}(t)$.
\begin{align*}
	\frac{1}{2}\frac{\mathrm{d}}{\mathrm{d}t}\left(\hat{T}(t)\cdot\hat{T}(t)\right) &= \hat{T}\cdot\hat{T}^{\prime}(t) \\
	\frac{\mathrm{d}}{\mathrm{d}t}\left(\hat{T}\cdot\hat{T}\right) &= \frac{\mathrm{d}}{\mathrm{d}t}1 = 0
\end{align*}
So, 
\begin{equation*}
	\hat{T}\cdot\hat{T}^{\prime}(t) = 0 \text{ and } \hat{T}(t) \perp \hat{T}^{\prime}(t)
\end{equation*}
\begin{equation*}
	\hat{N}(t)=\frac{\hat{T}^{\prime}(t)}{\norm{\hat{T}^{\prime}(t)}}\perp\hat{T}
\end{equation*}

\noindent
$\hat{N}$ is a unit vector perpendicular to $\hat{T}$ that points in the direction that the curve curls into. It is called the normal vector because it is perpendicular to the curve. It is in the same plane as $\vec{r}^\prime$, $\hat{T}$, and  $\vec{r}^{\prime\prime}$.

[INSERT IMAGE]

\noindent
$\hat{N}$ allows us to rewrite $\vec{r}^{\prime\prime}$.\\
\begin{equation*}
	\vec{r}^{\prime\prime}(t)=\frac{\mathrm{d}v}{\mathrm{d}t}\hat{T}(t)+v^{2}(t)\kappa(t)\hat{N}(t)
\end{equation*}

\noindent
We can see that $\vec{r}^{\prime\prime}(t)$ has two parts. If $\vec{r}(t)$ represents position, then $\frac{\mathrm{d}v}{\mathrm{d}t}$ represents linear acceleration and $v^2(t)\kappa(t)$ represents centripetal acceleration.\\ 
You might recognize the formula for centripetal acceleration in the 2nd part. If we let $R(t) = \frac{1}{\kappa(t)}$, then the 2nd part becomes $\frac{v^2(t)}{R(t)}$, which looks exactly like the formula for centripetal acceleration for uniform circular motion: $a_c = \frac{v^2}{r}$.