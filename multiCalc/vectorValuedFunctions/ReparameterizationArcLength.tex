\section{Reparameterization \& Arc Length}
\noindent
VVFs can be reparameterized to trace out the same curve at different speeds by replacing $t$ in $\vec{r}(t)$ with any non-decreasing function of $t$. This fact can come in handy to make the bounds of an integration problem more convenient.\\

\noindent
The integral of the derivative of a VVF gives the displacement vector because
\begin{equation*}
	\int_{a}^{b}{\vec{r^\prime}(t)\mathrm{d}t}=\vec{r}(b)-\vec{r}(a)
\end{equation*}
This is exactly like how velocity $\cdot$ time $=$ displacement.

\noindent
If we integrate the magnitude of $\vec{r^\prime}9t)$, we can use the fact that $\text{distance} = \text{speed}\cdot\text{time}$ to find the arc length of $\vec{r}(t)$ as
\begin{equation*}
	s=\int{\norm{\vec{r^\prime}(t)}\mathrm{d}t}=\int{\sqrt{\left(\frac{\mathrm{d}x}{\mathrm{d}t}\right)^2+\left(\frac{\mathrm{d}y}{\mathrm{d}t}\right)^2+\left(\frac{\mathrm{d}z}{\mathrm{d}t}\right)^2}\mathrm{d}t}
\end{equation*}
We can also write this as an arclength function,
\begin{equation*}
	s(t)=\int_{0}^{t}{\norm{\vec{r^{\prime}}(\tau)}\mathrm{d}\tau}
\end{equation*}

\noindent
If we have a function
\begin{equation*}
	f(t)=s(t)=\int_{0}^{t}{\norm{\vec{r^{\prime}}(\tau)}\mathrm{d}\tau}
\end{equation*}
where $s$ is strictly increasing, then $f$ has an inverse by the horizontal line test. That is $t(s) = f^{-1}(s)$ exists and is also non-decreasing. If we reparameterize $\vec{r}(t)$ to $\vec{r}(t(s))$, which is called the arc length parameterization, the parameterization will have a constant speed.