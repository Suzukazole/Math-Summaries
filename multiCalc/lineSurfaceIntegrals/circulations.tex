\subsubsection{Circulations}
\begin{definition}
	If $C$ is a simple, closed curve, then $\int\limits_{C}{\vec{F} \cdot \mathrm{d}\vec{r}}$ is the circulation of $\vec{F}$ on $C$.
\end{definition}
\noindent
We notate that we are taking a circulation as $\oint\limits_{C}{\vec{F} \cdot \mathrm{d}\vec{r}}$.

[INSERT IMAGE]

\noindent
For example, let's find the circulation of $\vec{F}(x,y,z) = \langle yz, xz, xy \rangle$ on the circle of radius 1 centered at $(0,0,1)$ in the $z = 1$ plane in the counter-clockwise direction.\\
\indent
$\vec{r}(t) = \langle \cos{t}, \sin{t}, 1 \rangle$, $- \leq t \leq 2\pi$\\
\indent
$\vec{F}\circ\vec{r} = \langle \sin{t}, \cos{t}, \sin{t}\cos{t} \rangle$\\
\indent
$\left(\vec{F}\circ\vec{r}\right) \cdot \vec{r^\prime}(t) = \cos{(2t)}$\\
\indent
$\oint\limits_{C}{\vec{F}\cdot\mathrm{d}\vec{r}} = \int_{0}^{2\pi}{\cos{(2t)}\mathrm{d}t}$\\
\indent
$= \frac{1}{2}\sin{(2t)}\rvert_{0}^{2\pi} = 0$\\