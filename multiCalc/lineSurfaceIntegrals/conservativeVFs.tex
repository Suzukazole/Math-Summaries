\subsection{Conservative Vector Fields}
\begin{definition}
	A vector field $\vec{F}$ is conservative if $\int\limits_{C}{\vec{F} \cdot \mathrm{d}\vec{r}}$ is the same for all $C$ connecting the same endpoints.
\end{definition}

\noindent
It's easy to see from this definition that vector fields of constant direction and magnitude, like $\vec{F}=\langle c, c, c \rangle$ is conservative, as its line integral only depends on the curve.\\

\begin{theorem}
	If $\vec{F}$ is conservative, then $\oint\limits_{C}{\vec{F} \cdot \mathrm{d}\vec{r}} = 0$.
\end{theorem}
\begin{proof}
	We can break the simple, closed curve, $C$ into two simple curves $C_1$ and $C_2$ that have the same endpoints and direction such that $C = C_1-C_2$.
	
	[INSERT IMAGE]
	
	\noindent
	So, 
	\begin{equation*}
		\oint\limits_{C}{\vec{F} \cdot \mathrm{d}\vec{r}} = \int\limits_{C_1}{\vec{f} \cdot \mathrm{d}\vec{r}} - \int\limits_{C_2}{\vec{F} \cdot \mathrm{d}\vec{r}}
	\end{equation*}
	Since $C_1$ and $C_2$ have the same direction and endpoints, and $\vec{F}$ is conservative, the line integrals have the same value, $L$.
	\begin{equation*}
		\oint\limits_{C}{\vec{F} \cdot \mathrm{d}\vec{r}} = L - L = 0	
	\end{equation*}
\end{proof}