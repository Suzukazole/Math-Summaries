\subsubsection{Potential Functions}
\noindent
Note that the FTC for line integrals doesn't care about the specific path taken between two points, but only the starting and ending points. Any paths that started and ended at the same points would have the same values, and paths that start and end at the same point would have a value of 0.\\
So, 
\begin{equation*}
	\oint\limits_{C}{\nabla f \cdot \mathrm{d}\vec{r}} = 0.
\end{equation*}
So, by our theorem that says all $\vec{F}$ such that $\oint\limits_{C}{\vec{F} \cdot \mathrm{d}\vec{r}} = 0$ are conservative, all vector fields $\vec{F} = \nabla f$ are conservative. $f$ is called the potential function of $\vec{F}$.\\

\noindent
You might recognize potential functions from physics. All conservative forces, like the force of gravity, have a potential energy function. For gravitational force, $U(r) = -\frac{GMm}{r}$ and $\vec{F_g}(r) = \frac{GMm}{r^2}\hat{r}$ where $\hat{r}$ is a radial unit vector pointing away from the object to which the force is applied.