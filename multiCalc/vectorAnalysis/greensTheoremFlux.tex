\subsection{Green's Theorem for Flux}
\begin{theorem}[Green's Theorem from Flux]
	Let $C$ be a closed, counter-clockwise oriented curve in $\mathbb{R}^2$ and let $D$ be the region contained within $C$. For any differentiable vector field $\vec{F}(x,y)$,
	\begin{equation*}
		\iint\limits_{D}{\nabla \cdot \vec{f}\mathrm{d}A} = \oint\limits_{C}{\vec{F} \cdot \hat{n}\mathrm{d}s}
	\end{equation*}
\end{theorem}

[INSERT IMAGE]

\noindent
This is saying that the sum of the divergence within $D$ is equal to the flux through $C$.\\
Our intuition for this is that divergence is the tendency for integral curves of $\vec{F}$ to spread out, and flux would then be these integral curves crossing the boundary curve $C$.\\

\noindent
This ability to convert between a line integral and surface integral often makes flux problems easier to solve. For example, one would have to calculate four integrals to find the flux through a rectangular region, but only a single double integral over the simple interior region.