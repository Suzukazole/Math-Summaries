\subsubsection{Electric Fields}
\noindent
Gauss's Law of Electricity is an important application of Divergence Theorem in physics. It says that $\Phi_{E} = \frac{Q_{in}}{\epsilon_{0}}$. That is, the electric flux through a closed surface (real or hypothetical) is proportional to the charge contained within that surface.\\
Rewritten more formally, $\oint\limits_{S}\oint{\vec{E} \cdot \mathrm{d}\vec{s}} = \iiint\limits_{V}{\frac{Q_{in}}{\epsilon_{0}}\mathrm{d}V}$\\
So, $\iiint\limits_{V}{\nabla\cdot\vec{E}\mathrm{d}V} = \iiint\limits_{V}{\frac{\sigma}{\epsilon_{0}}\mathrm{d}V}$ where $\sigma$ is the charge density.\\
If we let $V \to 0$, $\nabla \cdot \vec{E} = \frac{\sigma}{\epsilon_{0}}$. This is the differential form of Gauss's Law and the 1st of Maxwell's Equations.