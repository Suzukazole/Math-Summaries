\section{Stokes's Theorem}
\noindent
Let $C$ be a closed, counter-clockwise oriented curve in $\mathbb{R}^2$, and let $D$ be the region contained within $C$. Let $S$ be an open surface with opening boundary $C$, and lat $V$ be the region contained inside $\tilde{S} = D \cup S$.

[INSERT IMAGE]

\noindent
Finding the flux of $\nabla \times \vec{F}$,\\
$\oint\limits_{\tilde{S}}\oint{\nabla \times \vec{F}} = \iiint\limits_{V}{\nabla \cdot (\nabla \times \vec{F})\mathrm{d}V}$ by Divergence Theorem.\\
Since $\nabla \cdot (\nabla \times \vec{F}) = 0$, $\iint\limits_{D}{\nabla \times \vec{F}\mathrm{d}A} = \iint\limits_{S}{\nabla \times \vec{F} \cdot \mathrm{d}\vec{s}}$.\\
$\oint\limits_{C}{\vec{F} \cdot \mathrm{d}\vec{r}} = \iint\limits_{S}{\nabla \times \vec{F} \cdot \mathrm{d}\vec{s}}$ by Green's Theorem for Circulation. This is Stokes's Theorem.\\

\noindent
For example, let's use Stokes's Theorem to evaluate $\iint\limits_{S}{\nabla \times \vec{F} \cdot \mathrm{d}\vec{s}}$ where $S$ is the hemisphere $x^2 + y^2 + z^2 = 4, x \geq 0$ and $\vec{F}(x,y,z) = \langle yz, x\sin{z}, xyz^2 \rangle$.\\
\indent
$S = \left\{(x,y,z) \mid x^2 + y^2 + z^2 = 2^2, x \geq 0 \right\}$\\
\indent
$C = \left\{(y,z) \mid y^2 + z^2 = 2^2 \right\} = \left\{(r,\theta) \mid r = 2, 0 \leq \theta \leq 2\pi \right\}$ where $\theta$ is in the yz-plane.\\
\indent
$C = \vec{r}(t) = \langle 0, 2\cos{t}, 2\sin{t} \rangle, 0 \leq t \leq 2\pi$.\\
\indent
$\vec{F}\circ\vec{r} = \langle 4\cos{t}\sin{t}, 0, 0 \rangle$\\
\indent
$\vec{r^\prime}(t) = \langle 0, -2\sin{t}, 2\cos{t}\rangle$\\
\indent
$\left(\vec{F}\circ\vec{r}\right) \cdot \vec{r^\prime} = 0$\\
\indent
So, $\iint\limits_{S}{\nabla \times\ vec{F}\mathrm{d}A} = \oint_{0}^{2\pi}{0\mathrm{d}t} = 0$

\subsection{Faraday's Law of Induction \& Ampere's Law}
\noindent
Faraday’s Law of Induction quantifies the idea that changing magnetic flux in a coil induces a current in the coil. More precisely, it induces a voltage, the potential function of electric field $\left(\vec{E} = \nabla V\right)$, and this field will oppose the magnetic field that induced it. More formally,
\begin{equation*}
	-\frac{\partial}{\partial t}\iint\limits_{S}{\vec{B} \cdot \mathrm{d}\vec{s}} = \oint\limits_{C}{\vec{E} \cdot \mathrm{d}\vec{r}}
\end{equation*}
\begin{equation*}
	-\iint\limits_{S}{\frac{\partial}{\partial t}\vec{B} \cdot \mathrm{d}\vec{s}} = \iint\limits_{S}{\nabla \times \vec{E} \cdot \mathrm{d}\vec{s}}
\end{equation*}
by Stokes's Theorem.\\
As $S$ collapses to a point,
\begin{equation*}
	\nabla \times \vec{E} = -\frac{\partial}{\partial t}\vec{B}.
\end{equation*} 
This is Faraday's Law. It is the 3rd of Maxwell's Laws.\\

\noindent
The final of Maxwell's Equations is Ampere's Law (with Maxwell's correction). It says that 
\begin{equation*}
	\nabla \times \vec{B} = \mu_0\epsilon_0\frac{\partial\vec{e}}{\partial t} + \mu_0J
\end{equation*}
where $J$ is the current density and $\mu_0$ is the permeability of free space. Using Maxwell's equations and some basic properties of waves, we can derive the speed of light as $c = \frac{1}{\sqrt{\mu_0\epsilon_0}}$.