\section{Divergence Theorem}
\begin{theorem}
	Let $V$ be a compact solid, and let $S$ be its boundary surface. For any differentiable vector field $\vec{F}(x,y,z)$,
	\begin{equation*}
		\oint\limits_{S}{\oint{\vec{F} \cdot \mathrm{d}\vec{s}}} = \iiint\limits_{V}{\nabla \cdot \vec{F}\mathrm{d}V}
	\end{equation*}
	where $\mathrm{d}\vec{s} = \hat{n}\mathrm{d}s = (\vec{r_u}\times\vec{r_v})\mathrm{d}u\mathrm{d}v$ and $\mathrm{d}V$ is the volume differential.
\end{theorem}

[INSERT IMAGE]

\noindent
This says that the flux through a closed surface is equal to the sum of the divergence inside that surface.\\
Intuitively, divergence describes how much a vector field is going in or out at a point, so summing it up inside some solid would tell us the amount the vector field is going in or out on the solid’s boundary, which is flux.\\

\noindent
For example, let’s find the flux through the unit sphere centered at the origin from the vector field $\vec{F}(x,y,z) = \langle x, y, z^2 \rangle$.
\begin{equation*}
	S = \left\{(\rho, \theta, \phi) \mid \rho=1, 0 \leq \theta \leq 2\pi, 0 \leq \phi \leq \pi \right\}	
\end{equation*}
\begin{equation*}
	V = \left\{(\rho, \theta, \phi) \mid 0 \leq \rho \leq 1, 0 \leq\ theta \leq 2\pi, 0 \leq \phi \leq \pi \right\}	
\end{equation*}
\begin{align*}
	\text{Flux} &= \oint\limits_{S}{\oint{\vec{F} \cdot \mathrm{d}\vec{s}}}	\\
	&= \int_{0}^{1}{\int_{0}^{\pi}{\int_{0}^{2\pi}{\nabla \cdot \langle x, y, z^2 \rangle\rho^2\sin{\phi}\mathrm{d}\theta}\mathrm{d}\phi}\mathrm{d}\rho} \\
	&= \int_{0}^{1}{\int_{0}^{\pi}{\int_{0}^{2\pi}{(2 + 2z)\rho^2\sin{\phi}\mathrm{d}\theta}\mathrm{d}\phi}\mathrm{d}\rho} \\
	&= \int_{0}^{1}{\int_{0}^{\pi}{\int_{0}^{2\pi}{(2 + 2\rho\sin{\phi})\rho^2\sin{\phi}\mathrm{d}}\mathrm{d}}\mathrm{d}} \\
	&= 2\pi\int_{0}^{1}{\int_{0}^{\pi}{2\rho^2\sin{\phi} + 2\rho^3\sin{\phi}\cos{\phi}\mathrm{d}\phi}\mathrm{d}\rho} \\
	&= 4\pi\int_{0}^{\pi}{\frac{1}{3}\sin{\phi} + \frac{1}{4}\sin{\phi}\cos{\phi}\mathrm{d}\phi} \\
	&= \frac{8\pi}{3}
\end{align*}

\subsection{Gauss's Laws}
\subsubsection{Electric Fields}
\noindent
Gauss's Law of Electricity is an important application of Divergence Theorem in physics. It says that $\Phi_{E} = \frac{Q_{in}}{\epsilon_{0}}$. That is, the electric flux through a closed surface (real or hypothetical) is proportional to the charge contained within that surface.\\
Rewritten more formally, $\oint\limits_{S}\oint{\vec{E} \cdot \mathrm{d}\vec{s}} = \iiint\limits_{V}{\frac{Q_{in}}{\epsilon_{0}}\mathrm{d}V}$\\
So, $\iiint\limits_{V}{\nabla\cdot\vec{E}\mathrm{d}V} = \iiint\limits_{V}{\frac{\sigma}{\epsilon_{0}}\mathrm{d}V}$ where $\sigma$ is the charge density.\\
If we let $V \to 0$, $\nabla \cdot \vec{E} = \frac{\sigma}{\epsilon_{0}}$. This is the differential form of Gauss's Law and the 1st of Maxwell's Equations.
\subsubsection{Magnetic Fields}
\noindent
Gauss's Law for magnetism says that that magnetic flux through a closed surface is 0.\\
Rewritten more formally, $\oint\limits_{S}\oint{\vec{B} \cdot \mathrm{d}\vec{s}} = 0$.\\
So, $\iiint\limits_{V}{\nabla \cdot \vec{B}\mathrm{d}V} = \iiint\limits_{V}{0\mathrm{d}V}$.\\
If we let $V \to 0$, $\nabla \cdot \vec{B} = 0$. This means that there are no lone sources or sinks in magnetic fields. You may also hear this law summarized as "There are no magnetic monopoles." This is the 2nd of Maxwell's Equations.