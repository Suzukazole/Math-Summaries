\subsection{Faraday's Law of Induction \& Ampere's Law}
\noindent
Faraday’s Law of Induction quantifies the idea that changing magnetic flux in a coil induces a current in the coil. More precisely, it induces a voltage, the potential function of electric field $\left(\vec{E} = \nabla V\right)$, and this field will oppose the magnetic field that induced it.\\
More formally, $-\frac{\partial}{\partial t}\iint\limits_{S}{\vec{B} \cdot \mathrm{d}\vec{s}} = \oint\limits_{C}{\vec{E} \cdot \mathrm{d}\vec{r}}$.\\
$-\iint\limits_{S}{\frac{\partial}{\partial t}\vec{B} \cdot \mathrm{d}\vec{s}} = \iint\limits_{S}{\nabla \times \vec{E} \cdot \mathrm{d}\vec{s}}$ by Stokes's Theorem.\\
As $S$ collapses to a point, $\nabla \times \vec{E} = -\frac{\partial}{\partial t}\vec{B}$. This is Faraday's Law. It is the 3rd of Maxwell's Laws.\\

\noindent
The final of Maxwell's Equations is Ampere's Law (with Maxwell's correction). It says that $\nabla \times \vec{B} = \mu_0\epsilon_0\frac{\partial\vec{e}}{\partial t} + \mu_0J$ where $J$ is the current density and $\mu_0$ is the permeability of free space. Using Maxwell's equations and some basic properties of waves, we can derive the speed of light as $c = \frac{1}{\sqrt{\mu_0\epsilon_0}}$.