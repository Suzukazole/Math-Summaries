\subsection{Fubini's Theorem \& Domain Regions}
\begin{theorem}[Fubini's Theorem]
	The order of integration on a domain where the variables of integration $\left(x,y,\text{etc.}\right)$ vary independently doesn't matter.
\end{theorem}

\noindent
For example, let's find the volume under $f(x,y)=9-x^2-y^2, (x,y)\in[0,1]\times[1,2]$.\\
We will do so in two ways to show that they are equivalent: one with $x$ first and then $y$ and another with $y$ first and then $x$.\\
\begin{center}
	\begin{tabular}{c|c}
		$V=\int_{0}^{1}{\int_{1}^{2}{9-x^2-y^2\mathrm{d}y}\mathrm{d}x}$ & $V=\int_{2}^{1}{\int_{0}^{1}{9-x^2-y^2\mathrm{d}x}\mathrm{d}y}$ \\
		$=\int_{0}^{1}{\left[9y-x^2y-\frac{y^2}{3}\right]_{1}^{2}\mathrm{d}x}$ & $=\int_{1}^{2}{\left[9x-\frac{x^3}{3}-xy^2\right]_{0}^{1}\mathrm{d}y}$ \\
		$=\int_{0}^{1}{\frac{20}{3}-x^2\mathrm{d}x}$ & $=\int_{1}^{2}{\frac{26}{3}-y^2\mathrm{d}y}$ \\
		$=\left[\frac{20}{3}x-\frac{x^3}{3}\right]_{0}^{1}$ & $=\left[\frac{26}{3}y-\frac{y^3}{3}\right]_{1}^{2}$ \\
		$=\frac{19}{3}$ & $=\frac{19}{3}$ \\
	\end{tabular}
\end{center}

\noindent
Let's look at a case where $x$ and $y$ are not independent. specifically, where the bounds on $y$ are a function of $x$.

[INSERT IMAGE]

\noindent
This is called a Type I Region. Formally, a Type I Region is a domain $D=\left\{(x,y)|a\leq x\leq b, g(x)\leq y\leq h(x)\right\}$.\\

\begin{theorem}[Fubini's Theorem for Type I Regions]
	Let $D$ be a Type I Region in $\mathbb{R}^2$.\\
	$$\iint\limits_{D}{f(x,y)\mathrm{d}A}=\int_{a}^{b}{\int_{g(x)}^{h(x)}{f(x,y)\mathrm{d}y}\mathrm{d}x}$$
\end{theorem}

\noindent
It's also possible for y to have constant bounds and the bound for $x$ to be a function of $y$. this is a Type II Region. Formally, a Type II Region is a domain $D=\left\{(x,y)|g(y)\leq x\leq h(y), a\leq y\leq b\right\}$.

\begin{theorem}[Fubini's Theorem for Type II Regions]
	Let $D$ be a Type II Region in $\mathbb{R}^2$.\\
	$$\iint\limits_{D}{f(x,y)\mathrm{d}A}=\int_{a}^{b}{\int_{g(y)}^{h(y)}{f(x,y)\mathrm{d}x}\mathrm{d}y}$$
\end{theorem}

\noindent
Sometimes, a region can be describes as both Type I and Type II. You should pick whichever description is most convenient.\\

\noindent
The previous two theorems can be summarized as dependent variables need to be integrated before the variables that they depend on. This core idea extends into higher dimensions where classifying regions becomes tedious and not very helpful.\\

\noindent
One can split larger, harder to describe domains into smaller domains. Let $D_1\cup D_2=D$.

[INSERT IMAGE]

$$\iint\limits_{D}{f(x,y)\mathrm{d}A}=\iint\limits_{D_1}{f(x,y)\mathrm{d}A}+\iint\limits_{D_2}{f(x,y)\mathrm{d}A}-\iint\limits_{D_1\cap D_2}{f(x,y)\mathrm{d}A}$$