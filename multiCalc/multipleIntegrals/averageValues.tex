\subsection{Average Values}
\noindent
We can think of the average value of a function over some interval as the answer to the question: “If I flattened this function into a box over the interval, what would the height of the box be?”\\
For single-variable functions, the answer is $\bar{f} = \frac{1}{b-a}\int_{a}^{b}{f(x)\mathrm{d}x} = \frac{\int_{a}^{b}{f(x)\mathrm{d}x}}{\int_{a}^{b}{\mathrm{d}x}}$.\\
This idea of summing a function over a domain and the dividing by the size of that domain holds into multivariable as
\begin{equation*}
	\bar{f} = \frac{\int\limits_{D}{f(x,y)\mathrm{d}D}}{\int\limits_{D}{\mathrm{d}D}}
\end{equation*}

\subsubsection{Mean Value Theorem}
\begin{theorem}[Mean Value Theorem]
	If a fuction $f$ is continuous on a domain $D$, then there exists some point $p\in D$ such that $f(p)=\bar{f}$.
\end{theorem}