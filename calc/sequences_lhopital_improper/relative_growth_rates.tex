\section{Relative Growth Rates}
In many practical applications, like the run times of computer algorithms for example, we often want to know if a function grows slower, the same, or faster than another.
\begin{definition}
	If
	\begin{equation*}
		\lim_{x \to \infty}{\frac{f(x)}{g(x)}} = \infty \Leftrightarrow \lim_{x \to \infty}{\frac{g(x)}{f(x)}} = 0,
	\end{equation*}
	then $f$  grows faster than $g$.
	If
	\begin{equation*}
		\lim_{x \to \infty}{\frac{f(x)}{g(x)}} = c \Leftrightarrow \lim_{x \to \infty}{\frac{g(x)}{f(x)}} = \frac{1}{c}
	\end{equation*}
	for some non-zero constant $c$, then $f$ and $g$ grow at the same rate.
\end{definition}

\begin{example}
	Compare $e^x$ and $x^{100}$.
	Does one grow faster than the other, or do they grow at the same rate?
\end{example}
\begin{answer}
	\begin{align*}
		\lim_{x\to\infty}{\frac{e^x}{x^{100}}} &= \lim_{x\to\infty}{\frac{e^x}{100x^{99}}} \\
		&= \vdots \text{ (after many applications of L'H\^{o}pital's Rule)} \\
		&= \lim_{x\to\infty}{\frac{e^x}{100!}}
		&= \infty.
	\end{align*}
	
	So, $e^x$ grows faster than $x^{100}$.
	In fact, any exponential $b^x$ grows faster than any polynomial, as long as $b > 1$.
\end{answer}

\subsection{Transitive Grow Rates}
For sufficiently large $x$, growth rates are transitive.
That is, if $f$ grows the same/faster/slower/ than $g$, and $g$ grows the same/slower/faster than $h$, then $f$ also grows the same/faster/slower than $h$.

\begin{example}
	Show that $f(x)=\sqrt{x^2+5}$ and $g(x)=\left(2\sqrt{x}-1\right)^2$ grow at the same rate.
\end{example}
\begin{answer}
	We'll show that both $f$ and $g$ grow at the same rate as $h(x)=x$.
	Starting with $f$ and $h$,
	\begin{align*}
		\lim_{x\to\infty}{\frac{f(x)}{h(x)}} &= \lim_{x\to\infty}{\frac{\sqrt{x^2+5}}{x}} \\
		&= \lim_{x\to\infty}{\sqrt{\frac{x^2+5}{x^2}}} \\
		&= \sqrt{\lim_{x\to\infty}{\frac{x^2+5}{x^2}}} \text{ (by the Power Rule)} \\
		&= \sqrt{1} \\
		&= 1.
	\end{align*}
	
	So, $f$ and $h$ grow at the same rate.
	Moving on to $g$ and $h$,
	\begin{align*}
		\lim_{x\to\infty}{\frac{g(x)}{h(x)}} &= \lim_{x\to\infty}{\frac{\left(2\sqrt{x}-1\right)^2}{x}} \\
		&= \lim_{x\to\infty}{\left(\frac{2\sqrt{x}-1}{\sqrt{x}}\right)^2} \\
		&= \left(\lim_{x\to\infty}{\frac{2\sqrt{x}-1}{\sqrt{x}}}\right)^2 \text{ (by the Power Rule)} \\
		&= \left(2\right)^2 \\
		&= 4.
	\end{align*}
	
	So, $g$ and $h$ grow at the same rate.
	Since $f$ and $g$ both grow at the same rate as $h$, $f$ and $g$ must grow at the same rate as each other.
\end{answer}

\subsection{Growth Rate Hierarchy ($n^n$FEPL)}
For most of the common types of functions we see, we can establish families of functions and rank these families by their growth rates from fastest-growing to slowest-growing.
If two functions are in different families, we can be sure that one grows faster than the other.
If two functions are in the same family, we'll have to do more work to compare them.
These families are summarized by the acronym $n^n$FEPL\footnote{You might recognize these families as a sort of Big-O family from computer science.}.
\begin{itemize}[align=left, leftmargin=0.66in]
	\item[$\textbf{n}^\textbf{n}$] These are functions that have a variable both in the base and exponent.
	\item[\textbf{F}actorials] These are functions that have an $n!$ term.
	\item[\textbf{E}xponentials] These are functions that have a constant base and a variable exponent.
		Note that if the variable base if less than 1, the function actually gets smaller for larger $n$.
	\item[\textbf{P}olynomials] These are functions with a variable base and constant exponent.
		Certain polynomials can still grow faster than others.
		For example, $x^2$ grows faster than $x$, which grows faster than $\sqrt{x}$.
	\item[\textbf{L}ogarithms] These are functions that have a log of a polynomial.
\end{itemize}

Although these rules are indeed true, don't just apply them blindly.
You should try to simplify a function first before figuring out to which family it belongs.
For example, although $\ln{x^x}$ contains an $x^x$ and a $\ln$, it's neither in the $n^n$ family nor in the logarithms family.
In fact, although this function is not a polynomial, it grows faster than $x$ but slower than $x^2$. \\


A function belongs to the family of its fastest-growing positive term.
Negative terms can either be ignored or used to simplify other terms.
For example, although $x^3 + e^x$ contains a polynomial $x^3$ term, for very large $x$, the $e^x$ term dominates the growth, meaning this function is part of the exponentials family.