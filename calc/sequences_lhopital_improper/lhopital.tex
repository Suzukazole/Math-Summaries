\section{L'H\^{o}pital's Rule}
\subsection{Indeterminate Form 0/0}
If both $f(x)$ and $g(x)$ are 0 at $x=a$, then the limit
\begin{equation*}
	\lim_{x\to a}{\frac{f(x)}{g(x)}}.
\end{equation*}
is an indeterminate form of 0/0, meaning we can't substitute $x=a$ to evaluate the limit.
However, L'\^{o}pital's Rule allows us to modify this limit to get another limit, which might not have this indeterminate form but is guaranteed to have the same limit.

\begin{theorem}[L'H\^{o}pital's Rule, Weaker Form]
	If $f(a) = g(a) = 0$; $f^\prime(a)$ and $g^\prime(a) \neq 0$ exist, then
	\begin{equation*}
		\lim_{x\to a}{\frac{f(x)}{g(x)}} = \frac{f^\prime(a)}{g^\prime(a)}.
	\end{equation*}
\end{theorem}
\begin{proof}
	\begin{align*}
		\lim_{x\to a}{\frac{f(x)}{g(x)}} &= \lim_{x\to a}{\frac{f(x)-0}{g(x)-0}} \\
		&= \lim_{x\to a}{\frac{f(x)-f(x)}{g(x)-g(a)}} \\
		&= \lim_{x\to a}{\frac{\frac{f(x)-f(a)}{x-a}}{\frac{g(x)-g(a)}{x-a}}} \\
		&= \frac{\lim_{x\to a}{\frac{f(x)-f(a)}{x-a}}}{\lim_{x\to a}{\frac{g(x)-g(a)}{x-a}}} \\
		&= \frac{f^\prime(a)}{g^\prime(a)}.
	\end{align*}
\end{proof}

\begin{example}
	Find $\lim_{x\to 2}{\frac{x^2-4}{x-2}}$ using L'H\^{o}pital's Rule.
\end{example}
\begin{answer}
	\begin{equation*}
		\lim_{x\to 2}{\frac{x^2-4}{x-2}} = \frac{2(2)}{1} = 4.
	\end{equation*}
	
	Note that we get the same answer if we didn't use L'H\^{o}pital's Rule and instead factored.
	\begin{equation*}
		\lim_{x\to 2}{\frac{x^2-4}{x-2}} = \lim_{x\to 2}{\frac{(x+2)(x-2)}{x-2}} \lim_{x\to 2}{x+2} = 4.
	\end{equation*}
\end{answer}


It's possible that $f^\prime(a) = g^\prime(a) = 0$, meaning we're still left with the indeterminate form 0/0.
However, we can use a stringer form of L'H\^{o}pital's Rule that allows us to not have to immediate substitute $x=a$ and allows us to apply the rule multiple times if needed.
\begin{theorem}[L'H\^{o}pital's Rule, Stronger Form]
	If $f(a)=g(a)=0$; $f$ and $g$ are differentiable on an open interval $I$ that contains $a$; $g^\prime(x)\neq 0$ if $x\neq a$, then
	\begin{equation*}
		\lim_{x\to a}{\frac{f(x)}{g(x)}} = \lim_{x\to a}{\frac{f^\prime(x)}{g^\prime(x)}}
	\end{equation*}
	if the right-hand limit exists.
\end{theorem}

\begin{example}
	Find the following limit or show that it doesn't exist.
	\begin{equation*}
		\lim_{x\to 0}{\frac{\cos{x}-1}{e^x - x - 1}}.
	\end{equation*}
\end{example}
\begin{answer}
	$\cos{0}-1 = 0 = e^0 - 0 - 1$; $e^x - x - 1 \neq 0$ if $x\neq 0$ on all real numbers.
	\begin{equation*}
		\lim_{x\to 0}{\frac{\cos{x}-1}{e^x - x - 1}} = \lim_{x\to 0}{\frac{-\sin{x}}{e^x - 1}}.
	\end{equation*}
	
	$-\sin{0} = 0 = e^0 - 1$; $e^x - 1 \neq 0$ if $x\neq 0$ on all real numbers.
	\begin{equation*}
		\lim_{x\to 0}{\frac{-\sin{x}}{e^x - 1}} = \lim_{x\to 0}{\frac{-\cos{x}}{e^x}} = \frac{-1}{1} = -1.
	\end{equation*}
\end{answer}

\subsection{Indeterminate Forms $\infty/\infty$, $\infty\cdot 0$, \& $\infty - \infty$}
\subsubsection{$\infty/\infty$}
L'H\^{o}pital's still applies as written for the indeterminate form $\infty/\infty$.

\begin{example}
	Find the following limit or show that it doesn't exist.
	\begin{equation*}
		\lim_{x\to\pi/2}\frac{\tan{x}}{1+\tan{x}}.
	\end{equation*}
\end{example}
\begin{answer}
	Applying L'H\^{o}pital's Rule,
	\begin{equation*}
		\lim_{x\to\pi/2}{\frac{\tan{x}}{1+\tan{x}}} = \lim_{x\to\pi/2}\frac{\sec^2{x}}{\sec^2{x}} = 1.
	\end{equation*}
\end{answer}

\subsubsection{$\infty\cdot 0$}
We need to rearrange the limit into a 0/0 or $\infty/\infty$ indeterminate form.
\begin{example}
	Find the following limit or show that it doesn't exist.
	\begin{equation*}
		\lim_{x\to\infty}{x\sin{\frac{1}{x}}}.
	\end{equation*}
\end{example}
\begin{answer}
	Rearranging,
	\begin{equation*}
		\lim_{x\to\infty}{x\sin{\frac{1}{x}}} = \lim_{x\to\infty}\frac{\sin{\frac{1}{x}}}{\frac{1}{x}}.
	\end{equation*}
	
	Since the limit now has indeterminate form 0/0, we can apply L'H\^{o}pital's Rule.
	\begin{equation*}
		\lim_{x\to\infty}\frac{\sin{\frac{1}{x}}}{\frac{1}{x}} = \lim_{x\to\infty}\frac{\frac{-1}{x^2}\cos{\frac{1}{x}}}{\frac{-1}{x^2}} = cos(0) = 1.
	\end{equation*}
\end{answer}

\subsubsection{$\infty - \infty$}
We need to rearrange the limit into a 0/0 or $\infty/\infty$ indeterminate form.
\begin{example}
	Find the following limit or show that it doesn't exist.
	\begin{equation*}
		\lim_{x\to 1}{\frac{1}{\ln{x}} - \frac{1}{x-1}}.
	\end{equation*}
\end{example}
\begin{answer}
	Rearranging,
	\begin{equation*}
		\lim_{x\to 1}{\frac{1}{\ln{x}} - \frac{1}{x-1}} = \lim_{x\to 1}{\frac{x-1-\ln{x}}{(x-1)\ln{x}}}.
	\end{equation*}
	Since the limit now has indeterminate form 0/0, we can apply L'H\^{o}pital's Rule.
	\begin{equation*}
		\lim_{x\to 1}{\frac{x-1-\ln{x}}{(x-1)\ln{x}}} = \lim_{x\to 1}{\frac{1-\frac{1}{x}}{(x-1)\frac{1}{x} + \ln{x}}} = \lim_{x\to 1}{\frac{\frac{1}{x^2}}{\frac{1}{x} - (x-1)\frac{1}{x^2} + \frac{1}{x}}} = \frac{1}{1-0+1} = \frac{1}{2}.
	\end{equation*}
\end{answer}

\subsection{Indeterminate Forms $1^\infty$, $0^0$, \& $\infty^0$}
For indeterminate forms with exponents, we should take the natural log of the limit, solve that limit, and then exponentiate.

\subsubsection{$1^\infty$}
\begin{example}
	Find the following limit or show that it doesn't exist.
	\begin{equation*}
		\lim_{x\to \infty}{\left(1+\frac{1}{x}\right)^x}.
	\end{equation*}
\end{example}
\begin{answer}
	Let $L$ be the value of the limit.
	\begin{align*}
		L &= \lim_{x\to \infty}{\left(1+\frac{1}{x}\right)^x} \\
		\ln{L} &= \lim_{x\to\infty}{\ln{\left(\left(1+\frac{1}{x}\right)^x\right)}} \\
		&= \lim_{x\to\infty}{x\ln{\left(1+\frac{1}{x}\right)}} \text{ (indeterminate form $\infty\cdot 0$)} \\
		&= \lim_{x\to\infty}{\frac{\ln{\left(1+\frac{1}{x}\right)}}{\frac{1}{x}}} \text{ (indeterminate form $0/0$)} \\
		&= \lim_{x\to\infty}{\frac{\frac{1}{1+\frac{1}{x}}\frac{-1}{x^2}}{\frac{-1}{x^2}}} \\
		&= \lim_{x\to\infty}{\frac{1}{1+\frac{1}{x}}} \\
		&= 1 \\
		e^{\ln{L}} &= e^1 \\
		L &= e.
	\end{align*}
\end{answer}

\subsubsection{$0^0$}
\begin{example}
	Find the following limit or show that it doesn't exist.
	\begin{equation*}
		\lim_{x\to 0^+}{x^x}.
	\end{equation*}
\end{example}
\begin{answer}
	Let $L$ be the value of the limit.
	\begin{align*}
		L &= \lim_{x\to 0^+}{x^x} \\
		\ln{L} &= \lim_{x\to 0^+}{x\ln{x}} \text{ (indeterminate form $0\cdot-\infty$)} \\
		&= \lim_{x\to 0^+}{\frac{\ln{x}}{\frac{1}{x}}} \text{ (indeterminate form $-\infty/\infty$)} \\
		&= \lim_{x\to 0^+}{\frac{\frac{1}{x}}{\frac{-1}{x^2}}} \\
		&= \lim_{x\to 0^+}{\frac{1}{\frac{-1}{x}}} \\
		&= \lim_{x\to 0^+}{-x} \\
		&= 0 \\
		e^{\ln{L}} &= e^0 \\
		L &= 1.
	\end{align*}
\end{answer}

\subsubsection{$\infty^0$}
\begin{example}
	Find the following limit or show that it doesn't exist.
	\begin{equation*}
		\lim_{x\to\infty}{x^{\frac{1}{x}}}.
	\end{equation*}
\end{example}
\begin{answer}
	Let $L$ be the value of the limit.
	\begin{align*}
		L &= \lim_{x\to\infty}{x^{\frac{1}{x}}} \\
		\ln{L} &= \lim_{x\to\infty}{\frac{1}{x}\ln{x}} \text{ (indeterminate form $0\cdot\infty$)} \\
		&= \lim_{x\to\infty}{\frac{\ln{x}}{x}} \text{ (indeterminate form $\infty/\infty$)} \\
		&= \lim_{x\to\infty}{\frac{\frac{1}{x}}{1}} \\
		&= 0 \\
		e^{\ln{L}} &= e^0 \\
		L &= 1.
	\end{align*}
\end{answer}