\section{Sequences}
\begin{definition}
	A sequence $\left\{a_n\right\} = \left\{a_1, a_2, \ldots, a_n\right\}$ is an ordered list of numbers.
	Each element of a sequences is called a term and is identified by its index in the sequence.
	Sequences can be finite or infinite.
\end{definition}

\begin{example}
	The nth term of a sequence is defined by the following formula:
	\begin{equation*}
		a_n = \frac{(-1)^n}{n^2+1}.
	\end{equation*}
	Find the 1st, 2nd, and 100th terms of the sequence.
\end{example}
\begin{align*}
	a_1 &= \frac{(-1)^1}{1^2 + 1} = \frac{-1}{2} \\
	a_2 &= \frac{(-1)^2}{2^2 + 1} = \frac{1}{5} \\
	a_{100} &= \frac{(-1)^{100}}{100^2 + 1} = \frac{1}{10001}.
\end{align*}

\noindent
The above sequence was defined explicitly, meaning that we have a formula for the nth term of the sequence only in terms of n.
However, sequences can also be defined recursively, meaning the formula for subsequent terms of the sequence contains previous terms.
For a recursive sequence to be properly defined, there need to be one or more base terms that aren't defined recursively.
For example, the Fibonacci sequence, one of the most famous recursive sequences, as $a_1$ and $a_2$ as base terms.
\begin{equation*}
	a_n = \begin{cases}
		1 & n = 1, 2 \\
		a_{n-1} + a_{n-2} & n \geq 3
	\end{cases}.
\end{equation*}

\subsection{Common Types of Sequences}
There are some common types of sequences that you should be familiar with.
You might recognize these types of sequences and some of the formulas surrounding them from previous math classes.

\subsubsection{Arithmetic Sequences}
\begin{definition}
	An arithmetic sequence is one where $a_{n+1} - a_{n} = d$, a common difference, for all terms.
\end{definition}
\noindent
That is, to get the next term, we simply add some number $d$ (which could be negative) to the previous term.
Arithmetic sequences can be defined either explicitly or recursively.
Let $a_0$ be the starting term of the sequence.
\begin{align*}
	a_n &= dn + a_0 \\
	&= \begin{cases}
		a_0 & n = 0 \\
		d + a_{n-1} & n \geq 1
	\end{cases}.
\end{align*}
As we can see from the explicit formula, if we graphed values of an arithmetic sequence on in the plane with $x$ coordinate $n$ and $y$ coordinate $a_n$, all points would lie on a line with slope $d$ and $y$ intercept $a_0$.

\begin{example}
	Write an explicit formula for the following arithmetic sequence.
	\begin{equation*}
		\left\{\ln{2}, \ln{6}, \ln{18}, \ldots\right\}.
	\end{equation*}
\end{example}
Since we are given that this sequence is arithmetic, we'll find the common difference.
\begin{equation*}
	d = \ln{6} - \ln{2} = \ln{\frac{6}{2}} = \ln{3}.
\end{equation*}
\indent
So, applying the explicit formula for an arithmetic sequence with starting term $\ln{2}$ and common difference $\ln{3}$,
\begin{equation*}
	a_n = \ln{(3)}n + \ln{2}, n\geq 0.
\end{equation*}
\indent
We might have also noticed that each term inside the $\ln$ is triple the previous one, meaning we can write an explicit formula and then simplify to the same answer as before.
\begin{align*}
	a_n &= \ln{(3^{n}\cdot 2)}, n\geq 0 \\
	&= \ln{3^n} + \ln{2}, n \geq 0 \\
	&= \ln{(3)}n + \ln{2}, n \geq 0.
\end{align*}

\subsubsection{Geometric Sequences}
\begin{definition}
	A geometric sequence is one where $\frac{a_{n+1}}{a_n} = r$, a common ratio, for all terms.
\end{definition}
\noindent
That is, to get the next term, we simply multiply some number $r$ (which could be negative) by the previous term.
Geometric sequences can also be defined explicitly or recursively.
Let $a_0$ be the starting term of the sequence.
\begin{align*}
	a_n &= a_0(r)^n \\
	&= \begin{cases}
		a_0 & n = 0 \\
		ra_{n-1} & n \geq 1
	\end{cases}.
\end{align*}
\noindent
As we can see with the explicit formula, if we graphed terms of a geometric sequence for positive $r$, the points would lie on an exponential curve with $y$ intercept $a_0$ and exponential base $r$.

\begin{example}
	Write an explicit formula for the following geometric sequence.
	\begin{equation*}
		\left\{2,-6,18,-54,\ldots\right\}.
	\end{equation*}
\end{example}
Since we are given that this sequence is geometric, we'll find the common ratio.
\begin{equation*}
	r = \frac{-6}{2} = 3.
\end{equation*}
\indent
So, applying the explicit formula for a geometric sequence with starting term 2 and common ratio -3,
\begin{equation*}
	a_n = 2(-3)^n.
\end{equation*}

\subsection{Limits of a Sequence}
Once we have a formula for a sequence, we might be interested to know if $a_n$ tends towards some value as $n$ gets large.
\begin{definition}
	Let $L$ be a real number, the sequence $\left\{a_n\right\}$ as limit $L$ as $n$ approaches infinity if given any real $\epsilon > 0$, there is some index $m$ such that for all $n > m$
	\begin{equation*}
		\abs{a_n - L} < \epsilon.
	\end{equation*}
	We notate this as 
	\begin{equation*}
		\lim_{n\to\infty}{a_n} = L
	\end{equation*}
	and say the sequence converges to $L$.
	If the sequence does not have a limit, then we say the sequences diverges.
\end{definition}

\noindent
The following rules we gave for limits of a function: Sum and Difference Rule, Product Rule, Constant Multiple Rule, and Quotient Rule, all still apply to limits of sequences.
The only rule that doesn't still hold is the Power Rule because of the following sort of problem:
\begin{align*}
	a_n &= (-1)^n \\
	\lim_{n\to\infty}{a_n^2} &= 1 \\
	\left(\lim_{n\to\infty}{a_n}\right)^2 &= \text{DNE} \\
	1 &\neq \text{DNE}.
\end{align*}

\subsubsection{The Sandwich Theorem for Sequences}
\begin{theorem}[Sandwich Theorem for Sequences]
	If $\lim_{n\to\infty}{a_n} = \lim_{n\to\infty}{c_n} = L$ and there is an integer $m$ such that $a_n \leq b_n \leq c_n$ for all $n > m$, then $\lim_{n\to\infty}{b_n} = L$.
\end{theorem}
\noindent
This is essentially the same as the Sandwich Theorem for limits of a function.
The only added caveat is that we have to find some index $m$ for which the sandwiching inequality always holds for terms after the $m$th.

\begin{example}
	Determine if the following sequence converges or diverges.
	If it converges, find its limit.
	\begin{equation*}
		a_n = (-1)^n\frac{n-1}{2}, n\geq 1.
	\end{equation*}
\end{example}
We see that as $n$ grows large $\abs{a_n}$ approaches 1.
However, $a_n$ bounces between 1 and -1 depending on whether $n$ is even or odd.
Thus, we could let $\epsilon = 1/2$, which would show that the limit diverges.

\begin{example}
	Determine if the following sequences converges or diverges.
	If it converges, find its limit.
	\begin{equation*}
		a_n = \frac{\cos{n}}{n}, n\geq 1.
	\end{equation*}
\end{example}
It might at first seem that this limit diverges because $\cos$ bounces between -1 and 1.
However, we can use the Sandwich Theorem to show that the limit converges.
\begin{align*}
	\frac{-1}{n} &\leq \frac{\cos{n}}{n} \leq \frac{1}{n}, n\geq 1 \\
	\lim_{n\to\infty}{\frac{-1}{n}} &\leq \lim_{n\to\infty}{\frac{\cos{n}}{n}} \leq \lim_{n\to\infty}{\frac{1}{n}} \\
	0 &\leq \lim_{n\to\infty}{\frac{\cos{n}}{n}} \leq 0 \\
	\lim_{n\to\infty}{\frac{\cos{n}}{n}} &= 0.
\end{align*}

\subsubsection{Absolute Value Theorem}
We can apply the Sandwich Theorem to show that sequences whose absolute value converges to 0 must also converge to 0.
\begin{theorem}[Absolute Value Theorem]
	If $\lim_{n\to\infty}{\abs{a_n}} = 0$, then $\lim_{n\to\infty}{a_n} = 0$.
\end{theorem}
\begin{proof}
	For all $n$,
	\begin{equation*}
		-\abs{a_n} \leq a_n \leq \abs{a_n}.
	\end{equation*}
	Applying the Sandwich Theorem and limit properties,
	\begin{align*}
		-\lim_{n\to\infty}{\abs{a_n}} &\leq \lim_{n\to\infty}{a_n} \leq \lim_{n\to\infty}{\abs{a_n}} \\
		-0 &\leq \lim_{n\to\infty}{a_n} \leq 0 \\
		\lim_{n\to\infty}{a_n} &= 0.
	\end{align*}
\end{proof}