\subsection{2016 Free-Response Answers}

\begin{enumerate}
	\item \begin{enumerate}
		\item We can approximate $R^\prime(2)$ as the tangent slope between 1 and 3.
			\begin{equation*}
				R^\prime(2) \approx \frac{R(3)-R(1)}{3-1} = \frac{950-1190}{2} = -120.
			\end{equation*}
			So, we estimate $R^\prime(3)$ to be -120 liters/hour$^2$.
		\item Using a left Riemann sum using the values in the table,
			\begin{equation*}
				\int_{0}^{8}{R(t)\d{t}} \approx (1-0)1340 + (3-1)1190 + (6-3)950 + (8-6)740 = 1340 + 2380 + 2850 + 1480 = 8050.
			\end{equation*}
			So, the left Riemann sum approximates the total water removed to be 8050 liters.
		\item Using our answer from part (b) and integrating, we can find the total amount of water added or removed.
			\begin{equation*}
				\Delta \text{Water} = -8050 + \int_{0}{8}{2000e^{-t^2/20}\d{t}} \approx -214.
			\end{equation*}
			Since we know there is 50000 liters of water at $t=0$, we can add the net change to find the amount of water at $t=8$.
			\begin{equation*}
				\text{Water}_8 = \text{Water}_0 + \Delta \text{Water} = 50000 - 214 = 49768.
			\end{equation*}
			So, to the nearest liter, we approximate the amount of water in the tank at $t=8$ to be 49768 liters.
		\item At $t=0$, $W(0) = 2000 > R(0) = 1340$, so $W(0)-R(0) > 0$.
			At $t=8$, $W(8) = 81.52 < R(8) = 700$, so $W(8)-R(8) < 0$.
			Since both $W$ and $R$ are continuous, so is $W - R$.
			So, by the Intermediate Value Theorem, there is some $0 < t < 8$ such that $W(t)-R(t)=0$, or $W(t)=R(t)$.
	\end{enumerate}

	\item \begin{enumerate}
		\item We can apply the Fundamental Theorem of Calculus to find $x(3)$.
			\begin{equation*}
				\int_{0}^{3}{\dd{x}{t}\d{t}} = x(3) - x(0) = x(3) - 5 = 9.377 \implies x(3) = 14.377.
			\end{equation*}
			Looking at the graph, we see that $y(3)=-\frac{1}{2}$.
			So, at $t=3$, the particle's position is $\left(14.377, -0.5\right)$.
		\item Looking at the graph, we see that $y^\prime(3) = \frac{1}{2}$.
			Evaluating $\dd{x}{t}$ at $t=3$, we see that $x^\prime(3) = 9+\sin{27}$.
			So,
			\begin{equation*}
				\dd{y}{x} = \frac{\dd{y}{t}}{\dd{x}{t}} = \frac{\frac{1}{2}}{9+\sin{27}} \approx 0.0502.
			\end{equation*}
		\item Using our formula for parametric speed,
			\begin{equation*}
				s = \sqrt{(x^\prime(t))^2 + (y^\prime(t))^2} = \sqrt{\left(\frac{1}{2}\right)^2 + \left(9+\sin{27}\right)^2} \approx 9.969.
			\end{equation*}
		\item Starting with the formula for parametric arc length and splitting the integral into two peices,
			\begin{align*}
				\int_{0}^{2}{\sqrt{\left(y^\prime(t)\right)^2+\left(x^\prime(t)\right)^2}\d{t}} &= \int_{0}^{1}{\sqrt{(-2)^2+(t^2+\sin{(3t^2)})^2}\d{t}} + \int_{1}^{2}{\sqrt{(0)^2+(t^2+\sin{(3t^2)})^2}\d{t}} \\
				&\approx 2.237 + 2.112 \\
				&= 4.439.
			\end{align*}
	\end{enumerate}

	\item \begin{enumerate}
		\item Since we know by the Fundamental Theorem of Calculus that $f$ is the derivative of $g$, $g$ has critical points where $f$ is 0.
			$x=10$ is one such critical point.
			Both to the left and right of $x=10$, $f$ is negative.
			So, although $x=10$ is a critical point, it is neither a relative minimum or relative maximum of $g$ because $g$ is decreasing both left and right of $x=10$.
		\item Inflection points occur when the second derivative changes sign.
			Since $f$ is the derivative of $g$, inflection points of $g$ occur when the derivative of $f$ changes sign.
			To the left of $x=4$, the derivative of $f$ is positive.
			To the right of $x=4$, the derivative of $f$ is negative.
			So, $x=4$ is indeed an inflection point for $g$.
		\item $g$ has critical points where $f$ is 0, which is at $x=-2$, $x=2$, $x=6$, and $x=10$.
			Of these, $x=2$ and $x=10$ do not change sign, so they cannot be absolute extrema.
			Evaluating the remaining critical points and the endpoints using the geometry of the graph of $f$,
			\begin{align*}
				g(-4) &= -4 \\
				g(-2) &= -8 \\
				g(6) &= 8 \\
				g(12) &= -4.
			\end{align*}
			So, the absolute minimum of $g$ is at $x=-2$, and the absolute maximum of $g$ is at $x=6$.
		\item To the left $x=2$, $g(x)$ is negative whenever there is more area between $f$ and the $x$-axis above the $x$-axis than below.
			So, all points $[-4,2]$ have $g(x) \leq 0$.
			To the left of $x=2$, the opposite is true.
			So, all points $[10,12]$ have $g(x) \leq 0$.
			Putting these two results together, $g(x) \leq 0$ on $[-4,2] \cup [10,12]$. 
	\end{enumerate}

	\item \begin{enumerate}
		\item Implicitly differentiating,
			\begin{equation*}
				\d{^2y}{x^2} = 2x - \frac{1}{2}\dd{y}{x} = 2x - \frac{1}{2}\left(x^2 - \frac{1}{2}y\right).
			\end{equation*}
		\item Since we have expression for the first and second derivatives, it makes the most sense to apply a second derivative test.
			First, checking that $(-2,8)$ is indeed a critical point,
			\begin{equation*}
				\dd{y}{x}_{(-2,8)} = (-2)^2 - \frac{1}{2}(8) = 0.
			\end{equation*}
			Next, evaluating the second derivative,
			\begin{equation*}
				\dd{^2y}{x^2}_{(-2,8)} = 2(-2) - \frac{1}{2}\left((-2)^2 - \frac{1}{2}(8)\right) = -4 - 0 = -4.
			\end{equation*}
			Since the second derivative is negative at this critical point, the second derivative test tells us that this is a relative maximum.
		\item Since we know that $g(-1)=2$, both the numerator and denominator of the limit are in an indeterminate form of 0/0.
			So, we can apply L'H\^{o}pital's Rule.
			\begin{equation*}
				\lim_{x\to -1}{\left(\frac{g(x)-2}{3(x+1)^2}\right)} = \lim_{x\to-1}\left(\frac{g^\prime(x)}{6(x+1)}\right).
			\end{equation*}
			Using the given differential equation, we can see that at $(-1,2)$, $g^\prime(-1)=0$.
			So again we have an indeterminate form of 0/0 and can apply L'H\^{o}pital's Rule.
			\begin{equation*}
				\lim_{x\to-1}\left(\frac{g^\prime(x)}{6(x+1)}\right) = \lim_{x\to -1}{\frac{g^{\prime\prime}(x)}{6}}.
			\end{equation*}
			Using our answer from part (b), we know that at $(-1,2)$, $g^{\prime\prime}(-1) = -2$.
			So,
			\begin{equation*}
				\lim_{x\to -1}{\frac{g^{\prime\prime}(x)}{6}} = \frac{-2}{6} = -\frac{1}{3}.
			\end{equation*}
		\item Applying two iterations of Euler's method starting at $(0,2)$ with $\Delta x = \frac{1}{2}$,
			\begin{table}[H]
				\begin{center}
					\begin{tabular}{|c|c|c|c|c|}
						\hline
						$(x,y)$ & $\dd{y}{x}$ & $\Delta x$ & $\Delta y = \Delta x\dd{y}{x}$ & $(x+\Delta x, y+\Delta y)$ \\
						\hline
						$(0,2)$ & $-1$ & $\frac{1}{2}$ & $-\frac{1}{2}$ & $(\frac{1}{2},\frac{3}{2})$ \\
						\hline
						$(\frac{1}{2},\frac{3}{2})$ & $-\frac{1}{2}$ & $\frac{1}{2}$ & $-\frac{1}{4}$ & $(1,\frac{5}{4})$ \\
						\hline
					\end{tabular}
				\end{center}
			\end{table}
			So, our application of Euler's method approximates $h(1)$ to be 5/4.
	\end{enumerate}

	\item \begin{enumerate}
		\item Applying the formula for average value,
			\begin{equation*}
				\frac{1}{10-0}\int_{0}^{10}{\frac{1}{20}\left(3+h^2\right)\d{h}} = \frac{1}{200}\left(3h+\frac{h^3}{3}\right) = \frac{109}{60}.
			\end{equation*}
			So, the average radius of the funnel is $\frac{109}{60}$ inches.
		\item Applying the volume formula for a solid of revolution,
			\begin{align*}
				V &= \pi\int_{0}^{10}{\left(\frac{1}{20}\left(3+h^2\right)\right)\d{h}} \\
				&= \frac{\pi}{400}\left(\frac{h^5}{5}+2h^3+9h\right)\biggr\rvert_0^{10} \\
				&= \frac{\pi}{400}\left(20000+2000+90\right) \\
				&= \frac{2209\pi}{40}.
			\end{align*}
		\item Applying the chain rule,
			\begin{align*}
				\dd{r}{t} &= \dd{r}{h}\dd{h}{t} \\
				&= \frac{1}{10}h\dd{h}{t}.
			\end{align*}
			Solving with the information given,
			\begin{equation*}
				-\frac{1}{5} = \frac{1}{10}(3)\dd{h}{t} \implies \dd{h}{t} = -\frac{2}{3}.
			\end{equation*}
			So, at this instant, the height of the liquid is decreasing at 2/3 inches per second.
	\end{enumerate}

	\item \begin{enumerate}
		\item Finding several derivatives at $x=1$,
			\begin{align*}
				f(1) &= 1 \\
				f^\prime(1) &= -\frac{1}{2} \\
				f^{\prime\prime}(1) &= \frac{1}{4} \\
				f^{(3)}(1) &= -\frac{1}{4}.
			\end{align*}
			Applying the Taylor Series formula centered at $x=1$,
			\begin{equation*}
				f(x) = 1 - \frac{1}{2}(x-1) + \frac{1}{8}(x-1)^2 - \frac{1}{24}(x-1)^3 + \ldots + (-1)^n\frac{1}{n2^n}(x-1)^n.
			\end{equation*}
		\item Since we know that the series is centered at $x=1$ and the radius of convergence is 2, we know the series converges on $(-1,3)$ and need to check the endpoints.
			When $x=-1$,
			\begin{equation*}
				\sum_{n=1}^{\infty}{(-1)^n\frac{1}{n2^n}(-1-1)^n} = \sum_{n=1}^{\infty}{\frac{1}{n}}
			\end{equation*}
			diverges by the P-Test.
			When $x=3$,
			\begin{equation*}
				\sum_{n=1}^{\infty}{(-1)^n\frac{1}{n2^n}(3-1)^n} = \sum_{n=1}^{\infty}{(-1)^n\frac{1}{n}}
			\end{equation*}
			converges by the Alternating Series Test.
			So, the invterval of convergence is $(-1,3]$.
		\item Using the first three terms of our series from part (a) with $x=1.2$,
			\begin{equation*}
				f(1.2) \approx 1 - \frac{1}{2}(1.2-1) + \frac{1}{8}(1.2-1)^2 = 1 - \frac{1}{10} + \frac{1}{200} = 0.905.
			\end{equation*}
		\item Since this series is alternating, we can apply the Alternating Series Estimation Theorem.
			\begin{equation*}
				\abs{f(1.2)-P_2(1.2)} \leq \abs{\frac{-1}{2^3\cdot 3} (.2)^3 } = \frac{1}{3000} \leq 0.001.
			\end{equation*}
			So, by the Alternating Series Estimation Theorem, the error is certainly at most 0.001.
	\end{enumerate}
\end{enumerate}