\subsection{Complex Numbers}
\begin{definition}
	$i$ is called the imaginary unit. It's defined by $i^2 = -1$.
\end{definition}



Complex numbers ($\mathbb{C}$) have the form $z = \alpha + \beta i$, where $\alpha$ and $\beta$ are real numbers. The $\alpha$ part of $z$ is called the real part, so $\Re(z) = \alpha$. The $\beta$ part of $z$ is called the imaginary part, so $\Im(z) = \beta i$.\\


Often, complex numbers are visualized as points or vectors in a 2D plane, called the complex plane, where $\alpha$ is the x-component, and $\beta$ is the y-component. Thinking of complex numbers like points helps us define the magnitude of complex numbers and compare them. Since a point $(x,y)$ has a distance $\sqrt{x^2+y^2}$ from the origin, we can say the magnitude of $z$, $\lvert z \rvert$ is $\sqrt{\alpha^2 + \beta^2}$. Thinking of complex numbers like vectors helps us understand adding two complex numbers, since you just add the components like vectors.\\


A common operation on complex numbers is the complex conjugate. The complex conjugate of $z = \alpha + \beta i$ is $\overline{z} = \alpha - \beta i$. $z$ and $\overline{z}$ are called a conjugate pair.\\


Conjugate pairs have the following properties.
\\Let $z$, $w \in \mathbb{C}$.
\begin{enumerate}[label=]
	\item $\overline{z \pm w} = \overline{z} \pm \overline{w}$
	\item $\overline{zw}=\overline{z}\hspace{2pt}\overline{w}$
	\item $\overline{z}=z \Leftrightarrow z \in \mathbb{R}$
	\item $z\overline{z} = \lvert z \rvert^2 = \lvert \overline{z} \rvert^2$
	\item $\overline{\overline{z}} = z$
	\item $\overline{z}^n = \overline{z^n}$
	\item $z^{-1} = \frac{\overline{z}}{\lvert z \rvert^2}$
\end{enumerate}