\subsubsection{Repeated Linear Factors}

If $Q(x)$ has repeated roots, it factors into
\begin{equation*}
	Q(x) = R(x)(x-a)^k\text{, }k \geq 2\text{ and }R(a) \neq 0.
\end{equation*}
When making the common denominator for each repeated root of multiplicity $k$ (i.e the root appears $k$ times), we do
\begin{equation*}
	\frac{P(x)}{R(x)(x-a)^k} = \left(\text{Decomposition of }R(x)\right) + \frac{A_1}{x-a}+\ldots+\frac{A_k}{(x-a)^k}.
\end{equation*}
You would then multiply each side by the denominator like in the linear factors case and solve for the coefficients. The only additional difficulty is that you might have to use previous results or solve a system of linear equations to get some of the constants.

\begin{example}
	Find the partial fraction of $\frac{x^2+5x-6}{x^3-7x^2+16x-12}$.
\end{example}
\begin{answer}
	\begin{equation*}
		x^3-7x^2+16x-12 = (x-3)(x-2)^2.
	\end{equation*}
	So,
	\begin{equation*}
		\frac{x^2+5x-6}{x^3-7x^2+16x-12} =  \frac{A_1}{x-3} + \frac{A_2}{x-2} + \frac{A_3}{(x-2)^2}.
	\end{equation*}
	Multiplying each side by the denominator,
	\begin{equation*}
		x^2+5x-6 = A_1(x-2)^2 + A_2(x-2)(x-3) + A_3(x-3).
	\end{equation*}
	At $x=2$,
	\begin{equation*}
		8 = A_3(2-3) \implies A_3 = -8.
	\end{equation*}
	At $x=3$,
	\begin{equation*}
		18 = A_1(3-2)^2 \implies A_1 = 18.
	\end{equation*}
	Now we'll use our results for $A_1$ and $A_3$ to find $A_2$ using a value for $x$ that isn't 2 or 3 so the $A_2$ term doesn't become 0. A good choice is $x=0$.\\
	At $x=0$,
	\begin{equation*}
		-6 = 18(0-2)^2 + A_2(0-2)(0-3) + -8(0-3) \implies A_2 = -17.
	\end{equation*}
	So,
	\begin{equation*}
		\frac{x^2+5x-6}{x^3-7x^2+16x-12} = \frac{18}{x-3} - \frac{17}{x-2} - \frac{8}{(x-2)^2}.
	\end{equation*}
\end{answer}