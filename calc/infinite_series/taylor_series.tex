\section{Taylor Series}
\subsection{Construction}
Although we can approximate lots of functions using power series derived from the geometric series, it'd be nice if we had a more general way to approximate any function using a power series.
We already have an approximation using a tangent line: the 0th and 1st derivatives of the function and line are equal at the point of tangency.
We could extend this idea of derivatives being equal to higher-order derivatives and higher degree polynomials.

\begin{example}
	Construct  polynomial $P(x)=a_0+a_1x + a_2x^2 + a_3x^3 + a_4x^4$ with the following behavior at $x=0$:
	\begin{align*}
		P(0) &= 1 \\
		P^\prime(0) &= 2 \\
		P^{\prime\prime}(0) &= 3 \\
		P^{\prime\prime\prime}(0) &= 4 \\
		P^{(4)}(0) &= 5.
	\end{align*}
\end{example}
\begin{answer}
	Plugging in $x=0$ and solving for $a_0$,
	\begin{align*}
		a_0 + a_1(0) + a_2(0)^2 + a_3(0)^4 + a_4(0)^4 &= 1 \\
		a_0 &= 1. 
	\end{align*}
	
	Differentiating once, plugging in $x=0$, and solvinf for $a_1$,
	\begin{align*}
		P^\prime(x) &= a_1 + 2a_2x + 3a_3x^2 + 4a_4x^3 \\
		2 &= a_1 + 2a_2(0) + 3a_3(0)^2 + 4a_4(0)^3 \\
		a_1 &= 2.
	\end{align*}
	
	Continuing with differentiating and plugging in $x=0$, we get $a_2 = \frac{3}{2}$, $a_3 = \frac{2}{3}$, and $a_4 = \frac{5}{25}$.
	So,
	\begin{equation*}
		P(x) = 1 + 2x + \frac{3}{2}x^2 + \frac{2}{3}x^3 + \frac{5}{24}x^4.
	\end{equation*}
\end{answer}

\begin{example}
	Construct a degree 4 polynomial that approximates $\ln{(1+x)}$ at $x=0$.
\end{example}
\begin{answer}
	\begin{table}[H]
		\begin{center}
			\begin{tabular}{ccc}
				$f(x)$ & $P(x)$ & $a_n$ \\
				\hline
				$f(x)=\ln{(1+x)}$ & $P(x)=a_0 + a_1x + a_2x^2 + a_3x^3 + a_4x^4$ & \\
				$f(0)=\ln{(1+0)}=0$ & $P(0)=a_0$ & $a_0 = 0$ \\
				\hline
				$f^\prime(x)=\frac{1}{1+x}$ & $P^\prime(x) = a_1 + 2a_2x + 3a_3x^2 + 4a_4x^3$ & \\
				$f^\prime(0)=\frac{1}{1+0}=1$ & $P^\prime(0) = a_1$ & $a_1 = 1$ \\
				\hline
				$f^{\prime\prime}(x)=\frac{-1}{(1+x)^2}$ & $P^{\prime\prime}(x)=2a_2 + 6a_3x + 12a_4x^2$ & \\
				$f^{\prime\prime}(0)=\frac{-1}{(1+0)^2} = -1$ & $P^{\prime\prime}(x)=2a_2$ & $a_2 = \frac{-1}{2}$ \\
				\hline
				$f^{(3)}(x) = \frac{2}{(1+x)^3}$ & $P^{(3)}(x) = 6a_3 + 24a_4x$ & \\
				$f^{(3)}(0) = \frac{2}{(1+0)^3}=2$ & $P^{(3)}(0) = 6a_3$ & $a_3 = \frac{1}{3}$ \\
				\hline
				$f^{(4)}(x) = \frac{-6}{(1+x)^4}$ & $P^{(4)}(x) = 24a_4$ & \\
				$f^{(4)}(x) = \frac{-6}{(1+0)^4}=-6$ & $P^{(4)}(0) = 24a_4$ & $a_4 = \frac{-1}{4}$ \\
				\hline
			\end{tabular}
		\end{center}
	\end{table}
	
	So, our polynomial is
	\begin{equation*}
		P(x) = x - \frac{x^2}{2} + \frac{x^3}{3} - \frac{x^4}{4}.
	\end{equation*}
	
	Note that this polynomial exactly matches the power series we derived for $\ln{(1+x)}$. \\
	This polynomial is called the 4th order Taylor polynomial of $\ln{(1+x)}$ at $x=0$.
	The series created from all order Taylor polynomials is called the Taylor series of $\ln{(1+x)}$ at $x=0$.
\end{answer}

\subsection{Definition}
\begin{definition}
	Let $f$ be a $n$ times differentiable function where all derivatives exist at $x=a$.
	The Taylor series for $f$ at $x=a$ is
	\begin{equation*}
		\sum_{k=0}^{\infty}{\frac{f^{(k)}(x)}{k!}(x-a)^k} = f(a) + f^\prime(a)(x-a) + \frac{f^{\prime\prime}(a)}{2!}(x-a)^2 + \ldots + \frac{f^{(n)}(a)}{n!}(x-a)^n + \ldots.
	\end{equation*}
	The $n$th partial sum of the Taylor Series,
	\begin{equation*}
		P_n(x) = \sum_{k=0}^{n}{\frac{f^{(k)}(x)}{k!}(x-a)^k}
	\end{equation*}
	is the Taylor polynomial of order $n$ for $f$ at $x=a$.
\end{definition}

When $a=0$ you might also hear Taylor series referred to as Maclaurin series.
Like power series, Taylor series have intervals of convergence.

\begin{example}
	Find the Taylor series for $e^x$ at $x=0$.
	Verify using term-by-term differentiation that $e^x$ is its own derivative.
\end{example}
\begin{answer}
	We know that $e^x$ is its own derivative, so $f^{(k)}(0)=e^0 = 1$ for all $i$.
	Applying the definition,
	\begin{align*}
		e^x &= e^0 + e^0(x-0) + \frac{e^0}{2!}(x-0)^2 + \frac{e^0}{3!}(x-0)^3 + \ldots + \frac{e^0}{n!}(x-0)^n + \ldots \\
		&= 1 + x + \frac{x^2}{2!} + \frac{x^3}{3!} + \ldots + \frac{x^n}{n!} + \ldots.
	\end{align*}
	
	Differentiating term-by-term,
	\begin{equation*}
		\dd{}{x}e^x = 1 + x + \frac{x^2}{2!} + \ldots + \frac{x^{n-1}}{(n-1)!} + \ldots
	\end{equation*}
	we see that we get the same series, confirming that $e^x$ is its own derivative.
\end{answer}

\subsection{Common Maclaurin Series}
\begin{align*}
	\frac{1}{1-x} &= 1 + x + x^2 + \ldots = \sum_{k=0}^{\infty}{x^k}, \abs{x} < 1 \\
	\frac{1}{1+x} &= 1 - x + x^2 - \ldots = \sum_{k=0}^{\infty}{(-1)^kx^k}, \abs{x} < 1 \\
	e^x &= 1 + x + \frac{x^2}{2!} + \ldots = \sum_{k=0}^{\infty}{\frac{x^k}{k!}}, \text{ all real $x$} \\
	\sin{x} &= x - \frac{x^3}{3!} + \frac{x^5}{5!} - \ldots = \sum_{k=0}^{\infty}{(-1)^k\frac{x^{2k+1}}{(2k+1)!}}, \text{ all real $x$} \\
	\cos{x} &= 1 - \frac{x^2}{2!} + \frac{x^4}{4!} - \ldots = \sum_{k=0}^{\infty}{(-1)^k\frac{2^{2k}}{(2k)!}}, \text{ all real $x$} \\
	\ln{(1+x)} &= x - \frac{x^2}{2} + \frac{x^3}{3} - \ldots = \sum_{k=0}^{\infty}{(-1)^k\frac{x^{k+1}}{k+1}}, \abs{x} \leq 1 \\
	\arctan{x} &= x - \frac{x^3}{3} + \frac{x^5}{5} - \ldots = \sum_{k=0}^{\infty}{(-1)^k\frac{x^{2k+1}}{2k+1}}, \abs{x} \leq 1.
\end{align*}

\subsubsection{Euler's Identity}
You might have seen the identity $e^{i\pi} + 1 = 0$ or even $e^{ix} = \cos{x} + i\sin{x}$.
Using our common Taylor Series, we can derive this famous identity. \\


Starting with the Taylor series for $e^{ix}$,
\begin{align*}
	e^{ix} &= 1 + (ix) + \frac{(ix)^2}{2!} + \frac{(ix)^3}{3!} + \frac{(ix)^4}{4!} + \frac{(ix)^5}{5!} + \ldots = \sum_{k=0}^{\infty}{\frac{(ix)^k}{k!}}, \text{ all real $x$} \\
	&= 1 + ix + i^2\frac{x^2}{2!} + i^3\frac{x^3}{3!} + i^4\frac{x^4}{4!} + i^5\frac{x^5}{5!} + \ldots = \sum_{k=0}^{\infty}{i^k\frac{x^k}{k!}}, \text{ all real $x$} \\
	&= 1 + ix - \frac{x^2}{2!} - i\frac{x^3}{3!} + \frac{x^4}{4!} + i\frac{x^5}{5!} - \ldots, \text{ all real $x$} \\
	&= \left(1 - \frac{x^2}{2!} + \frac{x^4}{4!} + \ldots \right) + i\left(x - \frac{x^3}{3!} + \frac{x^5}{5!} + \ldots \right) \\
	&= \cos{x} + i\sin{x}.
\end{align*}

Plugging in $x=\pi$,
\begin{align*}
	e^{i\pi} &= \cos{\pi} + i\sin{\pi} \\
	&= -1 + 0 \\
	e^{i\pi} + 1 &= 0.
\end{align*}