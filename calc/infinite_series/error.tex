\section{Approximation Error}
Although knowing these infinite Taylor series is nice, if we want to use them to find function values, we'll need to approximate.
This means only using some finite number of terms in the Taylor series for our approximation.
Because we're not using all the terms, we'll naturally have some truncation error.
We'd like to be able to say how big this truncation error is so we can be sure that our approximation is good enough.

\subsection{Alternating Series Estimation Theorem}
\begin{theorem}[Alternating Series Estimation Theorem]
	Let $s$ be a convergent, alternating (i.e the sign of each term alternates) series where the terms of $\abs{s}$ are strictly decreasing.
	Then the $n$th term truncation error is the same sign as and less than in absolute value the $(n+1)$th term.
\end{theorem}

\begin{example}
	Give a bound for the truncation error of using the first 10 terms of the Maclaurin series of $\ln{(1+x)}$ to approximate $\ln{2}$.
\end{example}
\begin{answer}
	The Maclaurin series for $\ln{(1+x)}$ is
	\begin{equation*}
		\ln{(1+x)} = x - \frac{x^2}{2} + \frac{x^3}{3} - \ldots = \sum_{k=0}^{\infty}{(-1)^k\frac{x^{k+1}}{k+1}}, \abs{x} \leq 1,
	\end{equation*}
	
	which is an alternating series.
	We see that to approximate $\ln{2}$, we'd use $x=1$, which is in the interval of convergence, so the series converges.
	The first missing term of the series is $1/11$.
	Let $s_{10}$ be the partial sum of the first 10 terms when $x=1$.
	By the Alternating Series Estimation Theorem,
	\begin{equation*}
		0 < \ln{2} - s_{10} < \frac{1}{11}.
	\end{equation*}
\end{answer}

\subsection{Taylor's Theorem}
Although the Alternating Series Estimation Theorem is useful for quickly bounding the error of alternating series, we'd like something that can apply more generally to all Taylor series.
\begin{theorem}[Taylor's Theorem]
	Let $f$ be a $k+1$ times differentiable function on an open interval $I$ containing $a$.
	Then for all $x$ in $I$,
	\begin{equation*}
		f(x) = f(a) + f^\prime(a)(x-a) + \frac{f^{\prime\prime}(a)}{2!}(x-a)^2 + \ldots + \frac{f^{(n)}(a)}{n!}(x-a)^n + R_n(x)
	\end{equation*}
	where
	\begin{equation*}
		\abs{R_n(x)} = \frac{\abs{f^{(n+1)}(c)}}{(n+1)!}\abs{x-a}^{n+1}
	\end{equation*}
	for some $c$ between $x$ and $a$.
\end{theorem}

We can use this theorem to find the maximum value of $\abs{R_n(x)}$ over some interval.
We can also be more precise about what it means for a Taylor series to converge to some function over some interval.
\begin{definition}
	Let $R_n(x)$ be the remainder of truncating after the degree $n$ term in the Taylor series for $f$ centered at $x=a$.
	If for all $x$ in some interval $I$ containing $a$,
	\begin{equation*}
		\lim_{n\to\infty}{R_n(x)} = 0,
	\end{equation*}
	then we say the Taylor series for $f$ at $x=a$ converges to $f$ on $I$.
\end{definition}

\begin{example}
	Show that the Maclaurin series for $\sin{x}$ converges to $\sin{x}$ for all real $x$.
\end{example}
\begin{answer}
	We need to find the remainder and show that in the limit it goes to 0 as $n$ grows large.
	By Taylor's Theorem,
	\begin{align*}
		\abs{R_n(x)} &= \frac{\abs{f^{(n+1)}(c)}}{(n+1)!}\abs{x-a}^{n+1} \\
		&\leq \frac{\abs{x}^{n+1}}{(n+1)!}.
	\end{align*}
	
	The numerator is an exponential function, while the denominator is a factorial function, so using $n^n$FEPL,
	\begin{align*}
		0 \leq \lim_{n\to\infty}{R_n(x)} &\leq \lim_{n\to\infty}{\frac{\abs{x}^{n+1}}{(n+1)!}} = 0. \\
		\lim_{n\to\infty}{R_n(x)} &= 0.
	\end{align*}
	
	So, the Maclaurin series converges to $\sin{x}$ for all real $x$.
\end{answer}

\subsubsection{Remainder Estimation Theorem}
Notice that we didn't have to actually find the value of $f^{(n+1)}(c)$.
We just had to find a suitable upper bound where the limit would still go to 0.
\begin{theorem}[Remainder Estimation Theorem]
	If there are positive constants $M$ and $r$ such that
	\begin{equation*}
		\abs{f^{(n+1)}(t)} \leq Mr^{(n+1)}
	\end{equation*}
	for all $t$ between $a$ and $x$, then $R_n(x)$ satisfies the inequality
	\begin{equation*}
		\abs{R_n(x)} \leq M\frac{r^{n+1}\abs{x-a}^{n+1}}{(n+1)!}.
	\end{equation*}
\end{theorem}

\begin{example}
	Give a maximum error bound for using $\ln{(1+x)} = x - x^2/2$ when $\abs{x} \leq 0.1$.
\end{example}
\begin{answer}
	We are using the second-order Taylor polynomial, so we need to find $R_2(x)$.
	By Taylor's Theorem,
	\begin{align*}
		\abs{R_2(x)} &= \frac{\abs{f^{(2+1)}(c)}}{(2+1)!}\abs{x-0}^{2+1}  = \frac{\abs{f^{(3)}(c)}}{3!}\abs{x}^3\\
		f^{(3)}(x) &= \dd{^3}{x^3}\ln{(1+x)} = \frac{2}{(1+x)^3} \\
	\end{align*}
	
	On $-0.1 \leq x \leq 0.1$, $\abs{f^{(3)}(x)}$ is maximal at $(-0.1, \frac{2000}{729})$.
	\begin{equation*}
		\abs{R_2(x)} \leq \frac{\frac{2000}{729}}{6}\abs{x}^3.
	\end{equation*}
	
	On $-0.1 \leq x \leq 0.1$, $\abs{x}^3$ is maximal at $(-0.1, \frac{1}{1000})$.
	\begin{equation*}
		\abs{R_2(x)} \leq \frac{\frac{2000}{729}}{6}\frac{1}{1000} = \frac{1}{2187} \approx 4.572 \times 10^{-4}.
	\end{equation*}
	
	So, the error of approximating $\ln{(1+x)}$ with $x-x^2/2$ when $-0.1 \leq x \leq 0.1$ is at most $4.572 \times 10^{-4}$.
\end{answer}