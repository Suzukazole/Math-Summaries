\section{Convergence}
We have described what it means for a Taylor series to converge to a function over some interval.
What if we're given an infinite series that's not a function, just an infinite sum of numbers?
Can we still check if the series converges or diverges?
We'll develop several tests that we can apply to check for convergence or divergence.

\subsection{nth Term Test for Divergence}
\begin{lemma}
	Let $a_n$ be the nth term of a series $s$.
	If
	\begin{equation*}
		\lim_{n\to\infty}{a_n} \neq 0,
	\end{equation*}
	then the series diverges.
\end{lemma}
\begin{proof}
	Assume not.
	Let $a_n$ be the nth term and $s_n$ the nth partial sum of a convergent series $s$ whose terms do not tend to 0.
	Since $s$ converges to some value $L$, there exists some positive integer $m$ such that for all  $n > m$ and $\epsilon > 0$,
	\begin{equation*}
		\abs{s_n - L} < \epsilon.
	\end{equation*}
	We can also say the same for $s_{n+1}$.
	\begin{equation*}
		\abs{s_{n+1}-L} < \epsilon.
	\end{equation*}
	So, subtracting one inequality from the other,
	\begin{equation*}
		\abs{s_{n+1}-s_n} < 2\epsilon.
	\end{equation*}
	The difference between the two partial sums is just $a_{n+1}$.
	So,
	\begin{equation*}
		\abs{a_{n+1}} < 2\epsilon.
	\end{equation*}
	Since the terms of $s$ don't tend to 0, there exists some positive integer $h$ and real value $\delta > 0$ such that for all $n > h$,
	\begin{equation*}
		\abs{a_n} > \delta.
	\end{equation*}
	We can also say the same for $a_{n+1}$.
	\begin{equation*}
		\abs{a_{n+1}} > \delta.
	\end{equation*}
	Combining the two inequalities involving $a_{n+1}$, for all $n > \max{(m,h)}$,
	\begin{equation*}
		\delta < \abs{a_{n+1}} < 2\epsilon.
	\end{equation*}
	However, for $\epsilon \leq \delta/2$, the inequality creates a contradiction.
\end{proof}

\noindent
Although the proof has to be a bit specific to cover the case of alternating series, the idea behind the test makes sense.
If you're adding on terms that don't get smaller in absolute value, then you can't "zero in" on a particular value and converge.

\begin{example}
	Show that the following series diverges:
	\begin{equation*}
		\sum_{i=0}^{\infty}{(-1)^n} = 1 - 1 + 1 - 1 + \ldots.
	\end{equation*}
\end{example}
We see that
\begin{equation*}
	\lim_{n\to\infty}{a_n} = \text{DNE} \neq 0.
\end{equation*}
\indent
So, by the nth Term Test for Divergence, the series diverges.

\subsection{Geometric Series with $\abs{r} < 1$}
\begin{lemma}
	If $s$ is a geometric series with common ratio $r$, then $s$ converges if and only if $\abs{r} < 1$.
\end{lemma}
\noindent
We already proved this when talking about power series and gave a formula to find its value.

\begin{example}
	Show that the following series diverges:
	\begin{equation*}
		\sum_{i=0}^{\infty}{\frac{2^n}{3}}.
	\end{equation*}	
\end{example}
This is a geometric series with initial term $1/3$ and common ratio $r=2$.
Since $\abs{r} > 1$, the series diverges.

\subsection{P-Series Test}
\begin{lemma}
	The p-series
	\begin{equation*}
		\sum_{i=1}^{\infty}{\frac{1}{i^p}}
	\end{equation*}
	converges if and only if $p > 1$.
\end{lemma}

\begin{example}
	Show that the Harmonic Series diverges:
	\begin{equation*}
		\sum_{i=1}^{\infty}{\frac{1}{i}}.
	\end{equation*}
\end{example}
\indent
This is a p-series with $p=1$.
So, by the P-Series Test, the sum diverges.

\subsection{Direct Comparison Test}
\begin{lemma}
	Let $s = \sum_{i=0}^{\infty}{a_i}$ be a series where all $a_i \geq 0$.
	$s$ converges if there exists some convergent series $c = \sum_{i=0}^{\infty}{c_i}$ and positive integer $m$ such that for all $n > m$,
	\begin{equation*}
		c_n \geq a_n.
	\end{equation*}
	$s$ diverges if there exists some divergent series $d = \sum_{i=0}^{\infty}{d_i}$ and positive integer $m$ such that for all $n > m$,
	\begin{equation*}
		a_n \geq d_n.
	\end{equation*}
\end{lemma}

\begin{example}
	Show that the following series converges:
	\begin{equation*}
		\sum_{i=0}^{\infty}{\frac{1}{2 + 3^i}}.
	\end{equation*}
\end{example}
We see that this series looks really similar to a geometric series with common ratio 1/3, just with an extra 2 in the denominator.
For all $n > 0$,
\begin{equation*}
	\frac{1}{2+3^n} \leq \frac{1}{3^n}.
\end{equation*}
\indent
So, by the Direct Comparison Test, the series converges.

\begin{example}
	Show that the following series diverges:
	\begin{equation*}
		\sum_{i=0}^{\frac{1}{2+\sqrt{i}}}.
	\end{equation*}
\end{example}
Although the sum looks like a p-series with $p=1/2$, we can use that series because our series has smaller terms.
Instead, we can compare to a p-series where $p=1/2$.
\begin{align*}
	0 \leq \frac{1}{n} &\geq \frac{1}{2+\sqrt{n}} \\
	0 \leq 2 + \sqrt{n} &\leq n \\
	n & \geq 4.
\end{align*}
\indent
So, for all $n \geq 4$, our series has larger values than the p-series with $p=1$.
We know the p-series diverges by the P-Test, so by the Direct Comparison Test, our series also diverges.

\subsection{Limit Comparison Test}
\begin{lemma}
	Let $a_n$ and $b_n$ be the nth terms of two series $a$ and $b$ that have all positive terms after some point.
	If
	\begin{equation*}
		\lim_{n\to\infty}{\frac{a_n}{b_n}}
	\end{equation*}
	converges to a finite value greater than 0, then $a$ and $b$ either both converge or both diverge.
	If the limit converges to 0 and $b$ converges, then $a$ also converges.
	If the limit goes to $\infty$ and $b$ diverges, then $a$ also diverges.
\end{lemma}
\noindent
A common tactic is to select one of $a$ or $b$ to be a geometric or p-series.

\begin{example}
	Show that the following series diverges:
	\begin{equation*}
		\sum_{i=2}^{\infty}{\frac{2i}{i^2-i+a}}.
	\end{equation*}
\end{example}
Looking at this rational function, we see a degree 1 term in the numerator and a degree 2 term in the denominator.
So, we might expect that the series behaves similarly to $1/n$.
\begin{equation*}
	\lim_{n\to\infty}{\frac{\frac{2n}{n^2-n+a}}{\frac{1}{n}}} = \lim_{n\to\infty}{\frac{2n^2}{n^2-n+a}} = 2.
\end{equation*}
\indent
Since the limit converges to a finite value greater than 0, and we know that $1/n$ diverges by the P-Test, then the series must also diverge by the Limit Comparison Test.

\begin{example}
	Show that the following series converges:
	\begin{equation*}
		\sum_{i=1}^{\infty}{\frac{1}{2^i - 1}}.
	\end{equation*}
\end{example}
This series looks similar to the geometric series $1/2^n$.
\begin{equation*}
	\lim_{n\to\infty}{\frac{\frac{1}{2^n}}{\frac{1}{2^n-1}}} = \lim_{n\to\infty}{\frac{2^n - 1}{2^n}} = 1.
\end{equation*}
\indent
Since the limit converges to a finite value greater than 0, and we know that $1/2^n$ converges because it's a geometric series with $r=1/2$, then the series must also converge the the Limit Comparison Test.

\subsection{Integral Test}
\begin{lemma}
	Let $a$ be a sequence of positive terms where $a_n = f(n)$.
	If $f$ is continuous, positive after some $m$, and decreasing, then the series
	\begin{equation*}
		\sum_{i=m}^{\infty}{a_i}
	\end{equation*}
	converges if and only if the integral
	\begin{equation*}
		\int_{m}^{\infty}{f(x)\d{x}}
	\end{equation*}
	converges.
\end{lemma}

\begin{example}
	Show that the following series diverges:
	\begin{equation*}
		\sum_{i=1}^{\infty}{\frac{2i}{i^2+1}}.
	\end{equation*}
\end{example}
The function $f(x)=2x/(x^2+1)$ is continuous, positive for all $x \geq 1$, and decreasing.
\begin{equation*}
	\int_{1}^{\infty}{\frac{2x}{x^2+1}\d{x}} = (u = x^2+1) \int_{2}^{\infty}{\frac{\d{u}}{u}} = \ln{u}\biggr\rvert_{2}^{\infty} = \text{diverges}.
\end{equation*}
\indent
So, by the Integral Test, the series also diverges.

\begin{example}
	Show that the following series converges:
	\begin{equation*}
		\sum_{i=1}^{\infty}{\frac{1}{i^2+1}}.
	\end{equation*}
\end{example}
The function $f(x)=1/(x^2+1)$ is continuous, positive for all $x \geq 1$, and decreasing.
\begin{equation*}
	\int_{1}^{\infty}{\frac{\d{x}}{x^2+1}} = \arctan{x}\biggr\rvert_{1}^{\infty} = \frac{\pi}{2} - \frac{\pi}{4} = \frac{\pi}{4}.
\end{equation*}
\indent
So, by the Integral Test\footnote{We omitted evaluating the improper integral with limits, but the value is what you would get from doing that.}k the series also converges.

\subsection{Ratio Test}
\begin{lemma}
	Let $a$ be a series with only positive terms after some index.
	If the limit
	\begin{equation*}
		\lim_{n\to\infty}{\frac{a_{n+1}}{a_n}}
	\end{equation*}
	is less than 1, then the series converges.
	If the limit is greater than 1, then the series diverges.
	If the limit is equal to 1, then the test is inconclusive.
\end{lemma}

\begin{example}
	State whether the following series converges or diverges:
	\begin{equation*}
		\sum_{i=1}^{\infty}{\frac{i\ln{i}}{2^i}}.
	\end{equation*}
\end{example}
Taking the limit of the ratio of subsequent terms,
\begin{equation*}
	\lim_{n\to\infty}{\frac{\frac{(n+1)\ln{(n+1)}}{2^{n+1}}}{\frac{n\ln{n}}{2^n}}} = \lim_{n\to\infty}{\frac{(n+1)\ln{n+1}}{2n\ln{n}}} = \lim_{n\to\infty}{\frac{n+1}{2n}} = \frac{1}{2}.
\end{equation*}
\indent
Since the limit is less than 1, the series converges by the Ratio Test.

\subsection{nth Root Test}
\begin{lemma}
	Let $a$ be a series with all positive terms after some index.
	If the limit
	\begin{equation*}
		\lim_{n\to\infty}{\sqrt[n]{a_n}}
	\end{equation*}
	is less than 1, then the series converges.
	If the limit is greater than 1, the series diverges.
	If the limit is equal to 1, the test is inconclusive.
\end{lemma}
\noindent
This test is most useful for series that look like geometric series, but the common ratio is not a constant.

\begin{example}
	State whether the following series converges or diverges:
	\begin{equation*}
		\sum_{i=1}^{\infty}{\left(\frac{i}{2i-1}\right)^i}.
	\end{equation*}
\end{example}
Taking the limit of the nth root,
\begin{equation*}
	\lim_{n\to\infty}{\sqrt[n]{\left(\frac{n}{2n-1}\right)^n}} = \lim_{n\to\infty}{\frac{n}{2n-1}} = \frac{1}{2}.
\end{equation*}
\indent
Since the limit is less than 1, the series converges by the nth Root Test.

\subsection{Alternating Series Test}
All of the previous 8 tests have tested whether a series converges absolutely.
That is, all terms in the series could be made positive and the series would still converge or diverge.
There are some series that do converge but don't converge absolutely.
We say that these series converge conditionally.
You've already seen  few, like the formula for $\pi/4$ using the Maclaurin series for $\arctan{x}$.
All series that converge absolutely also converge conditionally.
Series that converge conditionally do so when they pass the following test.

\begin{lemma}
	The series
	\begin{equation*}
		\sum_{i=0}^{\infty}{(-1)^iu_i}
	\end{equation*}
	converges if all of the following conditions are satisfied.
	\begin{enumerate}
		\item All $u_i$ are positive.
		\item There exists an integer $m$ such that for all $n > m$, $u_{n+1} \leq u_n$.
		\item The series of $u$'s pass the nth Term Test.
	\end{enumerate}
\end{lemma}

\begin{example}
	State whether the following series converges or diverges:
	\begin{equation*}
		\sum_{i=1}^{\infty}{(-1)^i\frac{1}{i}}.
	\end{equation*}
\end{example}
\begin{enumerate}
	\item All the terms $1, 1/2, 1/3, \ldots$ are positive.
	\item All terms are less than the previous term.
	\item The terms tend to 0, passing the nth Term Test.
\end{enumerate}
\indent
So, by the Alternating Series Test, the series converges conditionally\footnote{You've already seen this series too. It converges to $\ln{2}$.}.
Note that the series does not converge absolutely because it fails the P-Test. \\