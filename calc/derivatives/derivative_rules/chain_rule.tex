\subsection{Chain Rule}
\begin{lemma}
	Let $f$ and $g$ be differentiable functions. Then
	\begin{equation}
		\dd{}{x}f(g(x)) = f^\prime(g(x))g^\prime(x).
	\end{equation}
	Equivalently, if $f$ is a function of $g$ and $g$ is a function of $x$,
	\begin{equation}
		\dd{f}{x} = \dd{f}{g}\hspace{3pt}\dd{g}{x}.
	\end{equation}
\end{lemma}
\begin{proof}
	Applying the limit definitions of the derivative,
	\begin{align*}
		\dd{f}{x} &= \lim_{h \to 0}{\frac{f(g(x+h))-f(g(x))}{h}} \\
		&= \lim_{h\to 0}{\frac{f(g(x+h))-f(g(x))}{g(x+h)-g(x)}\hspace{3pt}\frac{g(x+h)-g(x)}{h}} \\
		&= \lim_{h\to 0}{\frac{f(g(x+h))-f(g(x))}{g(x+h)-g(x)}} \hspace{3pt} \lim_{h\to 0}{\frac{g(x+h)-g(x)}{h}} \\
		&= \left(\lim_{h\to 0}{\frac{f(g(x+h))-f(g(x))}{g(x+h)-g(x)}}\right) \hspace{3pt} \dd{g}{x} \\
		&= \dd{f}{g}\hspace{3pt}\dd{g}{x}.
	\end{align*}
\end{proof}

\begin{example}
	Find the derivative of $y = (x^2 + 1)^5$ using the chain rule.
\end{example}
\begin{answer}
	$y$ is a composition of the two functions $x^5$ and $x^2 + 1$.
	Applying the chain rule, 
	\begin{equation*}
		y^\prime = \dd{}{(x^2+1)}(x^2+1)^5 \hspace{3pt} \dd{}{x}(x^2+1)
	\end{equation*}
	
	Making the substitution $u = x^2 + 1$,
	\begin{align*}
		y^\prime &= \dd{}{u}u^5 \hspace{3pt} \dd{}{x}(x^2+1) \\
		&= 5u^{4}2x.
	\end{align*}
	
	Substituting back,
	\begin{align*}
		y^\prime &= 5(x^2+1)^{4}2x \\
		&= 10x(x^2+1)^4.
	\end{align*}
\end{answer}

\subsubsection{$u$ Substitutions}
As we did in the example, we can substitute a variable, usually called $u$ when applying the chain rule.
\begin{example}
	Given that the derivative of $\sin{x}$ is $\cos{x}$, find the derivative of $f(x)=\sin^5{x}$.
\end{example}
\begin{answer}
	$f$ is a composition of $x^5$ and $\sin$.
	Substituting $u=\sin{x}$, we can rewrite $f$ as $u^5$.
	Applying the chain rule,
	\begin{align*}
		\dd{f}{x} &= \dd{f}{u} \hspace{3pt} \dd{u}{x} \\
		&= 5u^4 \hspace{3pt} \cos{x} \\
		&= 5\sin^4{x}\cos{x}.
	\end{align*}
\end{answer}