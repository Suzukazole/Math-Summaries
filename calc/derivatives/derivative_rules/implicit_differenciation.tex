\subsection{Implicit Differentiation}
Although we normally have equations of the form $y = f(x)$, some equations might be given in are are more convenient to write in different forms.
Using our understanding of the chain rule, we can still work with these forms and find $y^\prime$. There are generally three steps to finding $y^\prime$ when equations are given in these different forms.
\begin{enumerate}
	\item Get all terms involving $y$ to one side of the equation. This step is technically optional but usually makes the step 3 more convenient.
	\item Take the derivatives of both sides of the equation, using the chain rule.
	\item Rearrange to solve for $y^\prime$, making substitutions for $y$ to get the answer in terms of input parameters (e.g $x$).
\end{enumerate}
It's possible that you can get multiple solutions for $y^\prime$.
You'll need to check if 

\begin{example}
	Given that $y^2 = x$, find $y^\prime$ using implicit differentiation.
\end{example}
\begin{answer}
	Step 1 is already complete by what's given.
	Taking the derivative of both sides, remembering to use the chain rule,
	\begin{equation*}
		2yy^\prime = 1.
	\end{equation*}
	
	Solving for $y^\prime$,
	\begin{equation*}
		y^\prime = \frac{1}{2y}.
	\end{equation*}
	
	Looking back at our original equation, we see $y = \pm\sqrt{x}$.
	Substituting back into our work,
	\begin{equation*}
		y^\prime = \pm\frac{1}{2\sqrt{x}}.
	\end{equation*}
\end{answer}

Sometimes, it's not possible or is not necessary for what you're working on to complete step 3, meaning you leave your answer for $y^\prime$ in terms of $y$ and input parameters.
\begin{example}
	Find the derivative for $y$ with respect to $x$ at $(0,1)$ if $y^4 = x^3 + x + y$.
\end{example}
\begin{answer}
	Rearranging to complete step 1,
	\begin{equation*}
		y^4 - y = x^3 + x.
	\end{equation*}
	
	Taking the derivative of both sides,
	\begin{equation*}
		(4y^3 - 1)y^\prime = 3x^2 + 1.
	\end{equation*}
	
	Solving for $y^\prime$ and completing step 2,
	\begin{equation*}
		y^\prime = \frac{3x^2 + 1}{4y^3 - 1}.
	\end{equation*}
	
	Since we have both the $x$ and $y$ coordinates of where we're looking for the slope, we don't need to complete step 3.
	We can simply substitute to find our numerical answer for $y^\prime$.
	\begin{equation*}
		y^\prime_{(0,1)} = \frac{0 + 1}{4 - 1} = \frac{1}{3}.
	\end{equation*}
\end{answer}

Sometimes, we might already be given a formula for $y$, but rearranging and then doing implicit differentiation is easier.
\begin{example}
	Find the derivative with respect to $x$ of $y = \ln{x}$.
\end{example}
\begin{answer}
	We don't have any way to find the derivative of $\ln$ directly, but we do know how $\ln$ relates to exponential functions, a form we do know how to differentiate.
	Exponentiating both sides with base $e$,
	\begin{equation*}
		e^y = e^{\ln{x}} = x.
	\end{equation*}
	 
	Implicitly differentiating,
	\begin{equation*}
		e^{y}y^\prime = 1.
	\end{equation*}
	
	Solving for $y^\prime$,
	\begin{equation*}
		y^\prime = \frac{1}{e^y}.
	\end{equation*}
	
	Substituting what we were given for $y$,
	\begin{equation*}
		y^\prime = \frac{1}{e^{\ln{x}}} = \frac{1}{x}.
	\end{equation*}
\end{answer}

\subsubsection{Logarithmic Differentiation}
Logarithmic differentiation is a certain type of differentiation where you take to natural log of both sides and then implicitly differentiate.
It's especially useful when input parameters appear in both the base and exponent.
\begin{example}
	Find the derivative with respect to $x$ of $y = x^{\ln{x}}$.
\end{example}
\begin{answer}
	Taking the natural log of both sides,
	\begin{equation*}
		\ln{y} = \ln^2{x}.
	\end{equation*}
	
	Implicitly differentiating,
	\begin{align*}
		\frac{1}{y}y^\prime &= \frac{2\ln{x}}{x} \\
		y^\prime &= y\frac{2\ln{x}}{x} \\
		&= x^{\ln{x}}\frac{2\ln{x}}{x}.
	\end{align*}
\end{answer}


We now have the tools to prove the power rule for all real exponents.
\begin{proof}
	Let $n$ be an real number.
	\begin{align*}
		y &= x^n \\
		\ln{y} &= n\ln{x} \\
		\frac{1}{y}y^\prime &= n\frac{1}{x} \\
		y^\prime &= \frac{ny}{x} \\
		&= \frac{nx^n}{x} \\
		&= nx^{n-1}.
	\end{align*}
\end{proof}

\subsubsection{Combining Power and Exponential Rule}
We can also find the derivative of $f(x)^{g(x)}$, which combines the power and exponential rules and is something you'll likely won't see in a standard calculus course.
\begin{align*}
	y &= f^g \\
	\ln{y} &= g\ln{f} \\
	\frac{1}{y}y^\prime &= g\frac{1}{f}f^\prime + \ln{f}g^\prime \\
	y^\prime &= f^g\left(g\frac{1}{f}f^\prime + \ln{f}g^\prime\right) \\
	&= f^{g-1}\left(gf^\prime + fg^\prime\ln{f}\right).
\end{align*}