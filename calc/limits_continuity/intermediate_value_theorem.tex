\section{Intermediate Value Theorem}
\begin{theorem}[Intermediate Value Theorem (IVT)]
	If $f$ is continuous on the closed interval $[a,b]$, then for all $c \in [f(a), f(b)]$, there exists $x \in [a,b]$ such that $f(x) = c$.
\end{theorem}

That is, if $f$ is continuous on $[a,b]$, then $f$ must take on every value between $f(a)$ and $f(b)$.
This encapsulates the idea that if a function is continuous on some interval, then it is "connected" on that interval.

\begin{example}
	Use the IVT to show that $e^{-x} = x$ has at least one solution.
\end{example}
\begin{answer}
	Let $f(x) = e^{-x} - x$.
	We are looking for $x$ where $f(x) = 0$.
	$f(0) = 1$ and $f(1) = \frac{1}{e} - 1$.
	Since $f$ is continuous on the closed interval $[0,1]$, it must take on every value between $1$ and $\frac{1}{e} - 1$.
	Since $1$ is positive and $\frac{1}{e} - 1$ is negative, 0 is between these two values.
	Thus, by the IVT, there must exist a solution between $x=0$ and $x=1$.
\end{answer}