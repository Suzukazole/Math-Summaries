\section{Infinite Limits}
Although our limit definition works for finite values of $c$, it's also useful to think about what happens as $c$ goes to $\pm\infty$.
We'll need to add to our limit definition to incorporate infinite values, since it doesn't make sense to talk about neighborhoods at infinity.
\begin{definition}
	Let $f$ be a real-valued function defined on some subset $D \subseteq \R$ that contains arbitrarily large values.
	\begin{equation*}
		\lim_{x \to \infty}{f(x)} = L
	\end{equation*}
	if for every real $\epsilon > 0$, there is a real number $N > 0$ such that for all $x \in D$,
	\begin{equation}
		x > N \implies \abs{f(x) - L} < \epsilon.
	\end{equation}
\end{definition}
\noindent
All the same properties that we described for finite limits, like the Sum and Difference Rule, still hold for infinite limits.

\subsection{End Behavior Model}
When x is numerically large, we can often model the behavior of a complicated function with a simplier one that behaves roughly the same for numerically large input values and is the same in the limit.
There are a few rules that these follow.
\begin{enumerate}
	\item For a polynomial, the end-behavior is highest-degree term.
	\item For a rational function, like a ratio of polynomials, the end behavior is the ratio of the highest degree terms.
	\item For more complicated functions, we may need to use some reasoning about the graph of the function and limit properties to determine end-behavior.
\end{enumerate}

\subsection{Horizontal Asymptotes}
Horizontal Asymptotes are a special type of end-behavior model.
\begin{definition}
	The line $y=b$ is a horizontal asymptote of $y = f(x)$ if $\lim_{x\to \infty}{f(x)} = b$ or $\lim_{x \to -\infty}{f(x)} = b$.
\end{definition}

We can determine horizontal asymptotes for rational functions (usually quotient of polynomials).
There are a few cases to consider
\begin{enumerate}
	\item If the numerator is a higher degree than the denominator, there is no horizontal asymptote, so we'll need a different method to calculate what happens at $\pm\infty$.
	\item If the denominator is a higher degree than the numerator, then there is a horizontal asymptote at $y = 0$.
	\item If the numerator and denominator have the same degree, there is a horizontal asymptote at $y = k$ where k is the ratio of the highest degree terms.
\end{enumerate}

\begin{example}
	Find the following limits, if they exist.\\
	\begin{table}[H]
	\begin{center}
	\begin{tabular}{ l l l}
		1. $\begin{aligned}[t]
			\lim_{x \to \infty}{\frac{x^3 - 6x + 1}{x^2 + 2x - 3}}
		\end{aligned}$ & 
		2. $\begin{aligned}[t]
			\lim_{x\to -\infty}{\frac{x-9}{2x-x^2}}
		\end{aligned}$ &
		3. $\begin{aligned}[t]
			\lim_{x\to \infty}{\frac{6x^2-4x^5+7x-1}{12x^5-3x^2+2}}
		\end{aligned}$ \\
		\hspace{1pt} & \hspace{1pt}\\
		4. $\begin{aligned}[t]
			\lim_{x\to \infty}{\frac{3x+1}{\abs{x}+2}}
		\end{aligned}$ &
		5. $\begin{aligned}[t]
			\lim_{x \to \infty}{x + e^{-x}}
		\end{aligned}$ &
		6. $\begin{aligned}[t]
			\lim_{x \to -\infty}{x + e^{-x}}
		\end{aligned}$
	\end{tabular}
	\end{center}
	\end{table}
\end{example}
\begin{enumerate}
	\item Since the numerator degree is bigger than the denominator degree, we'll need to use the end behavior model.
		The end behavior model tells us that the numerator term dominates and has positive values, so the limit evaluates to $\infty$.
	\item Since the denominator has higher degree than the numerator, there is a HA at $y=0$, so the limit evaluates to $0$.
	\item Since the numerator and denominator have the same degree, the limit is the ratio of the highest-degree coefficients, $\frac{-1}{3}$.
	\item The numerator and denominator have the same degree. For $x > 0$, $\abs{x}+2 = x+2$, so the limit is the ratio of highest-degree coefficients, $3$.
	\item Looking at the two terms, we can see that as $x$ gets large, $e^{-x}$ gets very small, contributing less and less to the overall value.
		So, we can say that this function as a right end behavior model of $x$, so the limit is $\infty$.
	\item Looking at the two terms, we that that as $x$ gets very large and negative, $e^{-x}$ changes much faster than $x$.
		That is, $e^{-x}$ contributes more and more to the overall value of the function compared to $x$.
		So, we can say that this function has a left end behavior model of $e^{-x}$, so the limit is $\infty$.
\end{enumerate}