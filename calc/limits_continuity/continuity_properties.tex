\section{Continuity Properties}
These properties should look very similar to the properties of limits.
Let $f$ and $g$ be continuous functions at $c$.
\begin{align*}
	\textbf{Sum and Difference Rule: }& f \pm g \text{ is continuous at } c. \\
	\textbf{Product Rule: }& f \cdot g \text{ is continuous at } c. \\
	\textbf{Constant Multiple Rule: }& kf \text{ is continous at } c \text{ for all real } k. \\
	\textbf{Quotient Rule: }& \frac{f}{g} \text{ is continuous at } c \text{ as long as the value of the extended function of } \\
		g \text{ at } c \text{ is not } 0. \\
	\textbf{Composition Rule: }& f \circ g \text{ is continuous at } c \text{ if } f \text{ is continuous at } g(c). \\
	\textbf{Absolute Value Rule: }& \abs{f} \text{ is continuous at } c.
\end{align*}

The following types of functions are continuous on their domains
\begin{itemize}
	\item Polynomials
	\item Rational functions, except where the denominator is 0
	\item Trigonometric functions where defined
\end{itemize}

\begin{example}
	Show that the following function is continuous.
	\begin{equation*}
		f(x) = \tan{\left(\frac{x^2}{x^2+4}\right)}.
	\end{equation*}
\end{example}
\begin{answer}
	We can write $f$ as the composition of $\tan{x}$ and $\frac{x^2}{x^2+4}$.
	$\tan{x}$ is continuous on its domain because it is a trigonometric function.
	The only points not in its domain are $(2n+1)\frac{\pi}{2}$, where $n$ is an integer.
	$\frac{x^2}{x^2+4}$ is a rational function, but it's denominator is never $0$, so it is continuous over all real numbers.
	Now, we just need to check that all points in the range of $\frac{x^2}{x^2+4}$ are in the domain of $\tan{x}$.
	The range of $\frac{x^2}{x^2+4}$ is $[0,1)$.
	None of the points in this interval are not in the domain of $\tan{x}$, so the composition is continuous over all real numbers.
\end{answer}