\section{Sandwich Theorem}
We can use the Sandwich Theorem to indirectly find limits by "sandwiching" the function in question between two functions we do know the limit of.
If these two sandwiching functions go to the same value in the limit, then so to must the function in question.
\begin{theorem}[The Sandwich Theorem]
	If $g(x) \leq f(x) \leq h(x)$ and $\lim_{x \to c}{g(x)} = \lim_{x\to c}{h(x)} = L$, then $\lim_{x \to c}{f(x)} = L$.
\end{theorem}

\begin{example}
	Evaluate the following limit: $\lim_{x \to 0}{x\sin{\frac{1}{x}}}$.
\end{example}
We know that one of the properties of $\sin$ is that it oscillates between values of $-1$ and $-1$ for all input values.
So,
\begin{equation*}
	-1 \leq \sin{\frac{1}{x}} \leq 1.
\end{equation*}
Multiplying all terms by $x$,
\begin{equation*}
	-x \leq x\sin{\frac{1}{x}} \leq x.
\end{equation*}
Adding the limits,
\begin{equation*}
	\lim_{x\to 0}{-x} \leq \lim_{x \to 0}{x\sin{\frac{1}{x}}} \leq \lim_{x \to 0}{x}.
\end{equation*}
Evaluating the outer limits of the inequality,
\begin{equation*}
	0 \leq \lim_{x \to 0}{x\sin{\frac{1}{x}}} \leq 0
\end{equation*}
So, by the Sandwich Theorem,
\begin{equation*}
	\lim_{x \to 0}{x\sin{\frac{1}{x}}} = 0.
\end{equation*}