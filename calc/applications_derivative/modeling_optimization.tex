\section{Modeling \& Optimization}
Now that we know some ways to find extreme values, we can apply them to answer optimization problems.
The general steps needed to solve an optimization problem are:
\begin{enumerate}
	\item Write an equation that represents what you're trying to maximize/minimize. This is called your primary equation.
	\item Use additional information to eliminate excess variables.
	\item Find extreme values.
	\item Select the extreme values that fit the problem's constraints. Make sure your answer is what the problem is asking for.
\end{enumerate}

\begin{example}
	A farmer has 1000 linear feet of fence and wants to create a rectangular pasture.
	The pasture borders a river, which doesn't need a fence.
	What is the maximum area he can enclose?
\end{example}
\begin{answer}
	\begin{enumerate}
		\item Since the pasture is rectangular, we know the two lengths of fence perpendicular to the river will be the same length, which we'll call $x$.
			We'll say the remaining side parallel to the river has length $y$.
			So, the area enclosed is
			\begin{equation*}
				A(x,y) = xy.
			\end{equation*}
		\item There is no reason for the farmer not to use all 1000 linear feet of fence, so we'd expect the sum of the lengths of the 3 fenced sides to be equal to 1000 feet: $2x + y = 1000$, or $y = 1000 - 2x$.
			We can then substitute back into our primary equation to get it in terms of just $x$.
			\begin{equation*}
				A(y) = x(1000-2x) = 1000x - 2x^2.
			\end{equation*}
		\item We'll do a first derivative test to find extreme values.
		\begin{align*}
			A^\prime(y) &= 1000 - 4x \\
			A^\prime(y) &= 0 \text{ at } x = 250 \\
			A^\prime(200) &= 200 > 0 \\
			A^\prime(300) &= -200 < 0.
		\end{align*}
		\item So, $x=250 \implies y = 500$ is a local maximum.
			This corresponds to an area of $A = 250\cdot 500 = 125000\text{ft}^2$.
	\end{enumerate}
\end{answer}

\begin{example}
	What is the maximum area of a rectangle that has two vertices on the $x$-axis and two vertices on the portion of the graph $y=8-x^2$ where $y > 0$?
\end{example}
\begin{answer}
	\begin{enumerate}
		\item We can define any such trapezoid (which include all such rectangles) by the $x$ values of the vertices on the $x$-axis.
			We'll call these two values $x_1$ and $x_2$.
			We'll say arbitrarily that $x_1 < x_2$.
			So, the area enclosed is
			\begin{equation*}
				A(x_1, x_2) = (x_2 - x_1)\frac{f(x_1) + f(x_2)}{2}.
			\end{equation*}
		\item However, the problem specifically restricts us to a rectangle, not a trapezoid.
			So, the heights of each side must be equal.
			\begin{align*}
				f(x_1) &= f(x_2) \\
				8 - x_1^2 &= 8 - x_2^2 \\
				x_1^2 &= x_2^2 \\
				\pm x_1 &\ \pm x_2 \\
				-x_1 &= x_2 \text{ because $x_1 \neq x_2$}.
			\end{align*}
			Substituting back into our primary equation,
			\begin{equation*}
				A(x_2) = (x_2 - (-x_2))\frac{f(x_2) + f(-x_2)}{2} = 2x_2f(x_2) = 16x_2 - 2x_2^3.
			\end{equation*}
		\item We'll do a second derivative test for finding extreme values.
			\begin{align*}
				A^\prime(x_2) &= 16 - 6x_2^2 \\
				A^\prime(x_2) &= 0 \text{ at } x = \pm\frac{4}{\sqrt{6}} \\
				A^{\prime\prime}(x_2) &= -12x_2 \\
				A^{\prime\prime}\left(-\frac{4}{\sqrt{6}}\right) &= 8\sqrt{6} > 0 \\
				A^{\prime\prime}\left(\frac{4}{\sqrt{6}}\right) &= -8\sqrt{6} < 0.
			\end{align*}
		\item So, $x_2 = 4/\sqrt{6}$ is a local maximum, and $x_2 = -4/\sqrt{6}$ is a local minimum.
			The problem asks for a maximum, so we select $x_2 = 4/\sqrt{6}$.
			The problem asks for a maximum area, so
			\begin{equation*}
				A_{max} = 16\left(\frac{4}{\sqrt{6}}\right) - 2\left(\frac{4}{\sqrt{6}}\right)^3 = \frac{128}{3\sqrt{6}} \approx 17.419.
			\end{equation*}
	\end{enumerate}
\end{answer}