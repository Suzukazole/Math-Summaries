\subsection{$u$ Substitution (Reversing the Chain Rule)}
The chain rule tells us that $\dd{}{x}f(g(x)) = f^\prime(g(x))g^\prime(x)$.
So, if we see that we are integrating a function that contains within it a function and its derivative, we might want to try running the chain rule in reverse.
\begin{example}
	Solve the following indefinite integral:
	\begin{equation*}
		\int{x^2\cdot 2x\d{x}}.
	\end{equation*}
\end{example}
\begin{answer}
	Since we know that $\dd{x^2}{x} = 2x$, or in differential form $\d{x^2} = 2x\d{x}$, we can make a substitution.
	Letting $u = x^2$,
	\begin{align*}
		\int{x^2\cdot 2x\d{x}} &= \int{u\d{u}} \\
		&= \frac{u^2}{2} + C \\
		&= \frac{x^4}{2} + C.
	\end{align*}
\end{answer}


Sometimes the derivative itself won't be there, but a constant multiple of the derivative.
\begin{example}
	Solve the following indefinite integral:
	\begin{equation*}
		\int{\frac{x^2}{1-x^3}\d{x}}.
	\end{equation*}
\end{example}
\begin{answer}
	A good way to spot that there might be the possibility of a $u$-substitution here is the fact that the denominator is a polynomial of one higher degree than the numerator.
	So, if we took the derivative of the denominator, we'd get a polynomial of the same degree as the numerator.
	Let $u = 1-x^3$.
	Then $\d{u} = -3x^2\d{x}$; $x^2\d{x} = -\d{u}/3$.
	\begin{align*}
		\int{\frac{x^2}{1-x^3}\d{x}} &= \int{\frac{-\d{u}}{u}} \\
		&= -\ln{\abs{u}} + C \\
		&= -\ln{\abs{1-x^3}} + C.
	\end{align*}
\end{answer}


We can also apply $u$-substitutions to definite integrals.
We'll need to change the bounds of integration to be in terms of $u$ before we evaluate.
\begin{example}
	Solve the following definite integral:
	\begin{equation*}
		\int_{0}^{1}{15x^2\sqrt{5x^3+4}\d{x}}.
	\end{equation*}
\end{example}
\begin{answer}
	Let $u = 5x^3 + 4$.
	Then $\d{u} = 15x^2\d{x}$.
	When $x=1$, $u=5(0)^3 + 4 = 4$; when $x=1$, $u = 5(1)^3  +4 = 9$.
	\begin{align*}
		\int_{0}^{1}{15x^2\sqrt{5x^3+4}\d{x}} &= \int_{4}^{9}{\sqrt{u}\d{u}} \\
		&= \frac{2u\sqrt{u}}{3} \biggr\rvert_{4}^{9} \\
		&= \frac{2(9)\sqrt{9}}{3} - \frac{2(4)\sqrt{4}}{3} \\
		&= \frac{54}{3} - \frac{16}{3} \\
		&= \frac{38}{3}.
	\end{align*}
\end{answer}

\subsubsection{Trig Functions}
We can get the integrals of some of the trig functions by simply applying the derivatives for trig functions in reverse.
\begin{table}[H]
	\begin{center}
		\begin{tabular}{ l l }
			$\begin{aligned}\int{\sin{x}\d{x}}=-\cos{x} + C\end{aligned}$ & $\begin{aligned}\int{\cos{x}\d{x}}=\sin{x}+C\end{aligned}$
		\end{tabular}
	\end{center}
\end{table}


We can get the integrals of $\tan{x}$ and $\cot{x}$ using basic $u$-substitutions.
\begin{align*}
	\int{\tan{x}\d{x}} &= \int{\frac{\sin{x}}{\cos{x}}\d{x}} \\
	u = \cos{x} &\text{ and } \d{u} = -\sin{x}\d{x} \\
	&= \int{\frac{-\d{u}}{u}} \\
	&= -\ln{\abs{u}} + C \\
	&= -\ln{\abs{\cos{x}}} + C.
\end{align*}
\begin{align*}
	\int{\cot{x}\d{x}} &= \int{\frac{\cos{x}}{\sin{x}}\d{x}} \\
	u = \sin{x} &\text{ and } \d{u} = \cos{x}\d{x} \\
	&= \int{\frac{\d{u}}{u}} \\
	&= \ln{\abs{u}} + C \\
	&= \ln{\abs{\sin{x}}} + C.
\end{align*}


We can also get the integrals of $\sec{x}$ and $\csc{x}$ using $u$-substitutions, but we have to be a bit more clever and multiply the integrand by a strange-looking fraction that equals 1.
\begin{align*}
	\int{\sec{x}\d{x}} &= \int{\sec{x}\frac{\sec{x} + \tan{x}}{\tan{x}+\sec{x}}\d{x}} \\
	&= \int{\frac{\sec^2{x}+\sec{x}\tan{x}}{\tan{x}+\sec{x}}} \\
	u = \tan{x} + \sec{x} &\text{ and } \d{u} = (\sec^2{x} + \sec{x}\tan{x})\d{x} \\
	&= \int{\frac{\d{u}}{u}} \\
	&= \ln{\abs{u}} + C \\
	&= \ln{\abs{\tan{x} + \sec{x}}} + C.
\end{align*}
\begin{align*}
	\int{\csc{x}\d{x}} &= \int{\csc{x}\frac{\csc{x}+\cot{x}}{\cot{x}+\csc{x}}\d{x}} \\
	&= \int{\frac{\csc^2{x}+\csc{x}\cot{x}}{\cot{x}+\csc{x}}\d{x}} \\
	u = \cot{x} + \csc{x} &\text{ and } \d{u} = -(\csc^2{x}+\csc{x}\cot{x})\d{x} \\
	&= \int{\frac{-\d{u}}{u}} \\
	&= -\ln{\abs{u}} + C \\
	&= -\ln{\abs{\cot{x}+\csc{x}}} + C.
\end{align*}