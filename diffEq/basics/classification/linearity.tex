\subsection{Linearity}
\noindent
Another useful way to classify differential equations is by linearity.
\begin{definition}
	A differential equation is linear if all terms in the equation involving the dependent varaibles and its derivatives are in linear terms.
\end{definition}
\noindent
``Linear terms'' means that dependent varaibles should all be of degree 1, not be multiplied by a derivative involving the same variable, and not be in other functions like $\sin$ or $\ln$. However, this does not exclude differential equations from having parts that are functions of only independent variables, like in equation 4.\\

\noindent
Below is a table of linearity and equation numbers.
\begin{table}[H]
	\centering
	\begin{tabular}{c|c}
		Linearity & Equation Number \\
		\hline
		Linear &  2, 3, 4, 5, 8, 9 \\
		Nonlinear & 1, 6, 7, 10 \\
	\end{tabular}
\end{table}
\noindent
We'll focus a lot of time on linear equations because we have some mathematical tools that are good at dealing with them.