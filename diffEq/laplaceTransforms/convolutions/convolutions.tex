\section{Convolutions}
\subsection{Motivation}
Taking Laplace transforms can be difficult.
We already have some basic formulas derived from linearity properties of integrals that allow us to not evaluate an integral every time.\\

\noindent
One sort of formula that would be nice to have is one relating products of functions.
If we know that
\begin{equation*}
	H(s) = F(s)G(s),
\end{equation*}
where $F$ and $G$ are the Laplace transforms of $f$ and $g$, it'd be nice if we could find the inverse Laplace transform of $H$ to get $h$.
This is one way to define the convolution.

\subsection{Definition \& Convolution Theorem}
\begin{definition}
	Let $F(s)$ and $G(s)$ be the Laplace transforms of functions $f(t)$ and $g(t)$.
	Then
	\begin{equation*}
		F(s)G(s) = \Laplace{f \star g}.
	\end{equation*}
\end{definition}
\noindent
Note how using this definition we can see that $\star$, the convolution operator, is commutative.

\noindent
Defining this function that works like multiplication over Laplace transforms is only really useful if we have a formula for it, so let's derive one.
Starting from the left-hand side of the equation in the definition and applying the integral definition of the Laplace transform,
\begin{equation*}
	F(s)G(s) = \int_{0}^{\infty}{e^{-su}f(u)\d{u}} \cdot \int_{0}^{\infty}{e^{-sv}f(v)\d{v}}.
\end{equation*}
Making this product of integrals into a double integral (we usually do this in reverse when simplifying a double integral),
\begin{equation*}
	= \int_{0}^{\infty}{\int_{0}^{\infty}{e^{-s(u+v)}f(u)g(v)\d{u}}\d{v}}.
\end{equation*}
Letting $t = u+v$,
\begin{equation*}
	= \int_{0}^{\infty}{\int_{0}^{t}{e^{-st}f(u)g(t-u)\d{u}}\d{t}}.
\end{equation*}
Bringing $e^{-st}$ outside of the innermost integral,
\begin{equation*}
	= \int_{0}^{\infty}{e^{-st}\left[\int_{0}^{t}{f(u)g(t-u)\d{u}}\right]\d{t}}.
\end{equation*}
We see that we have the Laplace transform of the expression in the square brackets.
So,
\begin{equation*}
	f \star g = \int_{0}^{t}{f(u)g(t-u)\d{u}}.
\end{equation*}
This is an equivalent way to define the convolution.

\begin{example}
	Compute the convolution of $t^2$ with $t$.
\end{example}
\noindent
Applying the formula we just derived,
\begin{align*}
	t^2 \star t &= \int_{0}^{t}{u^2(t-u)\d{u}} \\
	&= t\frac{u^3}{3} - \frac{u^4}{4} \bigg\rvert_{0}^{t} \\
	&= \frac{t^4}{3} - \frac{t^4}{4} \\
	&= \frac{t^4}{12}. 
\end{align*}

\subsection{Properties}
Convolution inherits all the linearity properties of integration.
Let $f(t)$, $g(t)$, and $h(t)$ be piecewise continuous on $[0,\infty)$.
Let $a$ be a real constant.
Then
\begin{enumerate}[label=]
	\item \textbf{Commutativity} -- $f \star = g \star f$
	\item \textbf{Associativity} -- $(f \star g) \star h = f \star (g \star h)$
	\item \textbf{Distributivity over Addition} -- $f \star (g + h) = (f \star g) + (f \star h)$
	\item \textbf{Associativity over Scalar Multiplication} -- $a(f \star g) = (af) \star g$
\end{enumerate}

\subsection{Applications}
When we're using Laplace transforms to solve a differential equation, if we recognize $\Laplace{y}$ as the product of two functions for which we know the inverse Laplace transforms, we can compute a convolution to find $y$.

\begin{example}
	Solve the following differential equation using Laplace transforms and convolutions.
	Let $a$ and $c$ be real constants where $a \neq c$.
	\begin{equation*}
		y' - ay = e^{ct}.\footnotemark
	\end{equation*}
\end{example}
\footnotetext{When we're taking the Laplace transform of both sides, we implicitly use that $y(0)=0$.}
\noindent
Taking the Laplace transform of both sides,
\begin{align*}
	s\Laplace{y} - a\Laplace{y} &= \frac{1}{s-c} \\
	\Laplace{y}(s-a) &= \frac{1}{s-c}.
\end{align*}
Solving for $\Laplace{y}$,
\begin{equation*}
	\Laplace{y} = \frac{1}{s-c} \cdot \frac{1}{s-a}.
\end{equation*}
Recognizing the right-hand side of the equation as the product of the Laplace transforms of $e^{ct}$ and $e^{at}$, we can solve for $y$ by finding the convolution of $e^{ct}$ and $e^{at}$.
Applying our formula,
\begin{align*}
	F(s) = \frac{1}{s-c} &\text{ and } G(s) = \frac{1}{s-a} \\
	f(t) = e^{ct} &\text{ and } g(t) = e^{at}. \\
	y(t) &= f(t) \star g(t) \\
	&= \int_{0}^{t}{e^{cu}e^{a(t-u)}\d{u}} \\
	&= e^{at}\int_{0}^{t}{e^{u(c-a)}\d{u}} \\
	&= e^{at}\left(\frac{e^{(c-a)t}}{c-a} - \frac{1}{c-a}\right) \\
	&= \frac{e^{ct} - e^{at}}{c-a}.
\end{align*}