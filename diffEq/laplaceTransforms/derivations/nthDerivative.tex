\subsection{$n^{th}$ Derivative}
\noindent
We'll use induction to show that
\begin{equation*}
\Laplace{f^{(n)}(t)} = s^n\Laplace{f(t)} - f^{(n-1)}(0) - sf^{(n-2)}(0) - \ldots - s^{n-1}f(0).
\end{equation*}

\noindent
We'll start with the first derivative as a base case. We could use the $0^{th}$ derivative, but this case will give us a little more insight into where the formula comes from.\\
Let $f$ be a differentiable function.
\begin{equation*}
\Laplace{f'(t)} = \lim\limits_{n\to\infty}{\int_{0}^{n}{f'(t)e^{-st}\mathrm{d}t}}.
\end{equation*}
Integrating by parts and using the fundamental theorem of calculus,
\begin{equation*}
	 = \left(\lim\limits_{n\to\infty}{\left(f(n)e^{-sn}\right)} - f(0)e^{-s\cdot 0}\right) + \lim\limits_{n\to\infty}{s\int_{0}^{n}{f(t)e^{-st} \mathrm{d}t}}.
\end{equation*}
Assuming that $f(n)$ grows slower than $e^{-sn}$,
\begin{equation*}
	 = \left(0 - f(0)\right) + \lim\limits_{n\to\infty}{s\int_{0}^{n}{f(t)e^{-st} \mathrm{d}t}}.
\end{equation*}
Since the right part of the expression is just $s$ times the definition of $\Laplace{f}$,
\begin{equation*}
	 = s\Laplace{f} - f(0).
\end{equation*}
So,
\begin{equation*}
	\Laplace{f'(t)} = s\Laplace{f} - f(0).
\end{equation*}
Assuming the following is true,
\begin{equation*}
	\Laplace{f^{(n)}(t)} = s^n\Laplace{f(t)} - f^{(n-1)}(0) - sf^{(n-2)}(0) - \ldots - s^{n-1}f(0).
\end{equation*}
We'll show that the $n+1$ case follows.
\begin{equation*}
	\Laplace{f^{(n+1)}(t)} = \Laplace{\left(f^{(n)}\right)'}.
\end{equation*}
Using our first derivative formula,
\begin{equation*}
	 = s\Laplace{f^{(n)}(t)} - f^{(n)}(0).
\end{equation*}
Using our general formula,
\begin{align*}
	&= s \left(s^n\Laplace{f(t)} - f^{(n-1)}(0) - sf^{(n-2)}(0) - \ldots - s^{n-1}f(0)\right) - f^{(n)}(0) \\
	&= s^{n+1}\Laplace{f(t)} - f^{(n)}(0) - sf^{(n-1)}(0) s^2f^{(n-2)}(0) - \ldots - s^{n}f(0),
\end{align*}
which is the $n+1$ case, meaning we have proven the general formula as correct.