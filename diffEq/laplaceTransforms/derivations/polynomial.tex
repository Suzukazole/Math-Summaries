\subsection{Polynomials}
\noindent
We'll use our derivative formula and induction to show that
\begin{equation*}
	\Laplace{t^n} = \frac{n!}{s^{n+1}} \text{, } n \geq 0
\end{equation*}
Although the formula for $n = 0$ is clearly the same as $\Laplace{1}$, we'll use $n = 1$ as a base case to get a little more insight into where the formula comes from.\\
\begin{equation*}
	\Laplace{1} = \Laplace{t'} = \frac{1}{s}
\end{equation*}
Using our derivative formula,
\begin{equation*}
	\frac{1}{s} = s\Laplace{t} - 0^{1}
\end{equation*}
\begin{equation*}
	\implies \Laplace{t} = \frac{1}{s^2}
\end{equation*}
Assuming the following is true,
\begin{equation*}
	\Laplace{t^n} = \frac{n!}{s^{n+1}} \text{, } n \geq 0
\end{equation*}
we'll show that the $n+1$ case follows.
\begin{equation*}
	\Laplace{t^n} = \Laplace{\left(\frac{t^{n+1}}{n+1}\right)'} = \frac{n!}{s^{n+1}}
\end{equation*}
Using our derivative formula and the linearity of the Laplace transform,
\begin{equation*}
	\frac{n!}{s^{n+1}} = \frac{s}{n+1}\Laplace{t^{n+1}} - 0^{n+1}
\end{equation*}
\begin{equation*}
	\implies \Laplace{t^{n+1}} = \frac{(n+1)!}{s^{n+2}}
\end{equation*}
which is the $n+1$ case, meaning we have proven the general formula as correct.