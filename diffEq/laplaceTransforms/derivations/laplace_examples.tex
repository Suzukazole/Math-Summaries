\begin{example}
	Find the Laplace transform of $e^{-t}\sin{(3t)}$.
\end{example}
\noindent
Using the translation property,
\begin{equation*}
	\Laplace{e^{-t}\sin{(3t)}} = \mathcal{L}\left\{\sin{(3t)}\right\}(s+1).
\end{equation*}
We from our $\sin$ formula that
\begin{equation*}
	\Laplace{\sin{(3t)}} = \frac{3}{s^2 + 3^2}.
\end{equation*}
So,
\begin{equation*}
	\Laplace{e^{-t}\sin{(3t)}} = \mathcal{L}\left\{\sin{(3t)}\right\}(s+1) = \frac{3}{(s+1)^2 + 3^2}.
\end{equation*}

\begin{example}
	Find the Laplace transform of $t^2(t^2 + 1)(t - 1)(t + 2)$.
\end{example}
\noindent
Using the derivative of the Laplace transform property,
\begin{equation*}
	\Laplace{t^2(t^2 + 1)(t - 1)(t + 2)} = (-1)^2\frac{\mathrm{d}^2}{\mathrm{d}s^2}\Laplace{(t^2 + 1)(t - 1)(t + 2)}.
\end{equation*}
Expanding out the polynomial,
\begin{equation*}
	 = \frac{\mathrm{d}^2}{\mathrm{d}s^2}\Laplace{t^4 + t^3 - t^2 + t - 2}.
\end{equation*}
Using the polynomial formula,
\begin{equation*}
	 = \frac{\mathrm{d}^2}{\mathrm{d}s^2}\left(\frac{4!}{s^5} + \frac{3!}{s^4} - \frac{2!}{s^3} + \frac{1!}{s^2} + \frac{2\cdot 0!}{s^1}\right).
\end{equation*}
Taking the second derivative,
\begin{equation*}
	 \Laplace{t^2(t^2 + 1)(t - 1)(t + 2)} = \frac{6!}{s^7} + \frac{5!}{s^6} - \frac{4!}{s^5} + \frac{3!}{s^4} + \frac{2\cdot 2!}{s^3}.
\end{equation*}

\begin{example}
	Find the Laplace transform of $\left(x^2\right)''$ using the derivative formula. Show it's equal to $\Laplace{2}$.
\end{example}
\noindent
Using the derivative formula,
\begin{equation*}
	\Laplace{\left(x^2\right)''} = s^2\Laplace{x^2} - s\left(x^2\right)'_{x=0} - \left(x^2\right)_{x=0}.
\end{equation*}
Using the polynomial formula,
\begin{equation*}
 	= s^2\frac{2!}{s^3} - 0 - 0.
\end{equation*}
Simplifying,
\begin{equation*}
	\Laplace{\left(x^2\right)''} = \frac{2}{s} = \Laplace{2}.
\end{equation*}

\begin{example}
	Find the Laplace transform of $e^{it}$. Show that it's equal to $\Laplace{\cos{t} + i\sin{t}}$.
\end{example}
Using the exponential formula,
\begin{equation*}
	\Laplace{e^{it}} = \frac{1}{s - i}.
\end{equation*}
Now we'll work on $\cos{t} + i\sin{t}$.
Using the linearity of the Laplace transform,
\begin{equation*}
	\Laplace{\cos{\theta} + i\sin{\theta}} = \Laplace{\cos{t}} + i\Laplace{\sin{t}}.
\end{equation*}
Using the $\sin$ and $\cos$ formulas,
\begin{equation*}
	= \frac{s}{s^2 + 1^2} + i\frac{1}{s^2 + 1^2}.
\end{equation*}
Combining into one fraction,
\begin{equation*}
	= \frac{s+i}{s^2 + 1^2}.
\end{equation*}
Although we normally avoid it, we can factor the denominator into its two complex roots,
\begin{equation*}
	= \frac{s+i}{(s+i)(s-i)}.
\end{equation*}
Canceling out common factors we see that we get the same Laplace transform as we did with $e^{it}$.
\begin{equation*}
	\Laplace{\cos{\theta} + i\sin{\theta}} = \frac{1}{s-i} = \Laplace{e^{it}}.
\end{equation*}
