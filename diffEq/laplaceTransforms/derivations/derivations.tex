\section{Derivations}
\noindent
We can use the definition of the Laplace transform and the fact that the Laplace transform is a linear operator to find the Laplace transform of some common functions.

% constant
\subsection{Constant}
Let $a$ be a constant.\\
By definitions of a Laplace transform and an improper integral,
\begin{align*}
	\Laplace{a} &= \lim\limits_{n\to\infty}{\int_{0}^{n}{ae^{-st}\mathrm{d}t}} \\
	&= \frac{a}{s}\lim\limits_{n\to\infty}\left[-e^{-st}\right]_{0}^{n} \text{, } s \neq 0 \\
	&= \frac{a}{s}\lim\limits_{n\to\infty}{\left(1-e^{-sn}\right)} \text{, } s \neq 0 \\
	&= \frac{a}{s} \text{, } s > 0.
\end{align*}
So,
\begin{equation*}
	\Laplace{a} = \frac{a}{s} \text{, } s > 0.
\end{equation*}
% exponential
\subsection{Exponential}
\noindent
Let $a$ be a constant.\\
By definitions of a Laplace transform and an improper integral,
\begin{equation*}
\Laplace{e^{at}} = \lim\limits_{n\to\infty}{\int_{0}^{n}{e^{at}e^{-st}\mathrm{d}t}}
\end{equation*}
\begin{equation*}
 = \lim\limits_{n\to\infty}{\int_{0}^{\infty}{e^{(a-s)t} \mathrm{d}t}}
\end{equation*}
\begin{equation*}
	= \frac{1}{a-s}\lim\limits_{n\to\infty}{\left[e^{(a-s)n}\right]}_{0}^{n} \text{, } s \neq a
\end{equation*}
\begin{equation*}
	= \frac{1}{a-s}\lim\limits_{n\to\infty}{\left(e^{(a-s)n} - 1\right)} \text{, } s \neq a
\end{equation*}
\begin{equation*}
	= \begin{cases}
		\frac{-1}{a-s} & s > a \\
		\text{DNE} & s \leq a
	\end{cases}
\end{equation*}
So,
\begin{equation*}
	\Laplace{e^{at}} = \frac{-1}{a-s} \text{, } s > a
\end{equation*}
% sin and cos
\subsection{Sine and Cosine}

% Sin
\noindent
\subsubsection{$\textbf{Sin}$}
Let $a$ be a constant.\\
By definitions of a Laplace transform and an improper integral,
\begin{equation*}
	\Laplace{\sin{(at)}} = \lim\limits_{n\to\infty}{\int_{0}^{n}{\sin{(at)}e^{-st}\mathrm{d}t}}
\end{equation*}
\begin{equation*}
	= \frac{-1}{s^2 + a^2}\lim\limits_{n\to\infty}{\left[e^{-st}\left(s\sin{(at)} + a\cos{(at)}\right)\right]_{0}^{n}}
\end{equation*}
\begin{equation*}
	 = \frac{-1}{s^2 + a^2}\left(\lim\limits_{n\to\infty}\left(e^{-sn}\left(s\sin{(an)} + a\cos{(an)}\right)\right) - \left(e^{-s\cdot 0}\left(s\sin{(a\cdot 0)} + a\cos{(a\cdot 0)}\right)\right)\right)
\end{equation*}
Both $\sin$ and $\cos$ have maximum values of 1, so we can say that the left part of them expression has a maximum value at most $s + a$. For positive $s$, the exponential dominates and the expression goes to 0 in the limit.
\begin{equation*}
	 = \frac{-1}{s^2 + a^2}\left(0 - a\right) \text{, } s > 0
\end{equation*}
\begin{equation*}
	 = \frac{a}{s^2 + a^2} \text{, } s > 0
\end{equation*}
So,
\begin{equation*}
	\Laplace{\sin{(at)}} = \frac{a}{s^2 + a^2} \text{, } s > 0
\end{equation*}
% Cos
\subsubsection{Cosine}
\noindent
Let $a$ be a constant.\\
By definitions of a Laplace transform and an improper integral,
\begin{equation*}
\Laplace{\cos{(at)}} = \lim\limits_{n\to\infty}{\int_{0}^{n}{\cos{(at)}e^{-st}\mathrm{d}t}}
\end{equation*}
\begin{equation*}
	= \frac{1}{s^2 + a^2}\lim\limits_{n\to\infty}{\left[e^{-st}\left(a\sin{(at)} - s\cos{(at)}\right)\right]_{0}^{n}}
\end{equation*}
\begin{equation*}
	 = \frac{1}{s^2 + a^2}\left(\lim\limits_{n\to\infty}{\left(e^{-sn}\left(a\sin{sn} - s\cos{(an)}\right)\right)} - \left(e^{-s\cdot 0}\left(a\sin{(a\cdot 0)} - s\cos{(a\cdot 0)}\right)\right)\right)
\end{equation*}
Both $\sin$ and $\cos$ have maximum values of 1 and minimum values of -1, so we can way that the left part of the expression has a maximum value of at most $a + s$. For positive $s$, the exponential dominates and the expression goes to 0 in the limit.
\begin{equation*}
	 = \frac{1}{s^2 + a^2}\left(0 + s\right) \text{, } s > 0
\end{equation*}
\begin{equation*}
	 = \frac{s}{s^2 + a^2} \text{, } s > 0
\end{equation*}
So,
\begin{equation*}
	\Laplace{\cos{(at)}} = \frac{s}{s^2 + a^2} \text{, } s > 0
\end{equation*}
% nth derivative
\subsection{$n^{th}$ Derivative}
\noindent
We'll use induction to show that
\begin{equation*}
\Laplace{f^{(n)}(t)} = s^n\Laplace{f(t)} - f^{(n-1)}(0) - sf^{(n-2)}(0) - \ldots - s^{n-1}f(0).
\end{equation*}

\noindent
We'll start with the first derivative as a base case. We could use the $0^{th}$ derivative, but this case will give us a little more insight into where the formula comes from.\\
Let $f$ be a differentiable function.
\begin{equation*}
\Laplace{f'(t)} = \lim\limits_{n\to\infty}{\int_{0}^{n}{f'(t)e^{-st}\mathrm{d}t}}
\end{equation*}
Integrating by parts and using the fundamental theorem of calculus,
\begin{equation*}
	 = \left(\lim\limits_{n\to\infty}{\left(f(n)e^{-sn}\right)} - f(0)e^{-s\cdot 0}\right) + \lim\limits_{n\to\infty}{s\int_{0}^{n}{f(t)e^{-st} \mathrm{d}t}}.
\end{equation*}
Assuming that $f(n)$ grows slower than $e^{-sn}$,
\begin{equation*}
	 = \left(0 - f(0)\right) + \lim\limits_{n\to\infty}{s\int_{0}^{n}{f(t)e^{-st} \mathrm{d}t}}.
\end{equation*}
Since the right part of the expression is just $s$ times the definition of $\Laplace{f}$,
\begin{equation*}
	 = s\Laplace{f} - f(0).
\end{equation*}
So,
\begin{equation*}
	\Laplace{f'(t)} = s\Laplace{f} - f(0).
\end{equation*}
Assuming the following is true,
\begin{equation*}
	\Laplace{f^{(n)}(t)} = s^n\Laplace{f(t)} - f^{(n-1)}(0) - sf^{(n-2)}(0) - \ldots - s^{n-1}f(0).
\end{equation*}
We'll show that the $n+1$ case follows.
\begin{equation*}
	\Laplace{f^{(n+1)}(t)} = \Laplace{\left(f^{(n)}\right)'}
\end{equation*}
Using our first derivative formula,
\begin{equation*}
	 = s\Laplace{f^{(n)}(t)} - f^{(n)}(0).
\end{equation*}
Using our general formula,
\begin{align*}
	&= s \left(s^n\Laplace{f(t)} - f^{(n-1)}(0) - sf^{(n-2)}(0) - \ldots - s^{n-1}f(0)\right) - f^{(n)}(0) \\
	&= s^{n+1}\Laplace{f(t)} - f^{(n)}(0) - sf^{(n-1)}(0) s^2f^{(n-2)}(0) - \ldots - s^{n}f(0),
\end{align*}
which is the $n+1$ case, meaning we have proven the general formula as correct.
% polynomial
\subsection{Polynomials}
\noindent
We'll use our derivative formula and induction to show that
\begin{equation*}
	\Laplace{t^n} = \frac{n!}{s^{n+1}} \text{, } n \geq 0.
\end{equation*}
Although the formula for $n = 0$ is clearly the same as $\Laplace{1}$, we'll use $n = 1$ as a base case to get a little more insight into where the formula comes from.\\
\begin{equation*}
	\Laplace{1} = \Laplace{t'} = \frac{1}{s}
\end{equation*}
Using our derivative formula,
\begin{align*}
	\frac{1}{s} &= s\Laplace{t} - 0^{1} \\
	&\implies \Laplace{t} = \frac{1}{s^2}.
\end{align*}
Assuming the following is true,
\begin{equation*}
	\Laplace{t^n} = \frac{n!}{s^{n+1}} \text{, } n \geq 0.
\end{equation*}
We'll show that the $n+1$ case follows.
\begin{equation*}
	\Laplace{t^n} = \Laplace{\left(\frac{t^{n+1}}{n+1}\right)'} = \frac{n!}{s^{n+1}}
\end{equation*}
Using our derivative formula and the linearity of the Laplace transform,
\begin{align*}
	\frac{n!}{s^{n+1}} &= \frac{s}{n+1}\Laplace{t^{n+1}} - 0^{n+1} \\
	&\implies \Laplace{t^{n+1}} = \frac{(n+1)!}{s^{n+2}},
\end{align*}
which is the $n+1$ case, meaning we have proven the general formula as correct.
% translation
\subsection{Translation}
\noindent
Let $a$ be a constant.
\begin{equation*}
	\Laplace{e^{at}f(t)} = \int_{0}^{\infty}{e^{at}f(t)e^{-st} \mathrm{d}t} = \int_{0}^{\infty}{f(t)e^{-(s-a)t} \mathrm{d}t} = \mathcal{L}\left\{f(t)\right\}\left(s-a\right).
\end{equation*}
So,
\begin{equation*}
	\Laplace{e^{at}f(t)} = \mathcal{L}\left\{f(t)\right\}\left(s-a\right).
\end{equation*}
This illustrates how multiplying by $e^{at}$ in the $t$ domain corresponds to a translation by $a$ in the $s$ domain.
% derivative of laplace transforms
\subsection{Derivative of a Laplace Transform}
\noindent
Consider the $n^{th}$ derivative of the Laplace transform of $f$,
\begin{equation*}
	\frac{\mathrm{d}^n}{\mathrm{d}s^n}\left(\Laplace{f(t)}\right) = \frac{\mathrm{d}^n}{\mathrm{d}s^n}\left(\int_{0}^{\infty}{f(t)e^{-st} \mathrm{d}t}\right).
\end{equation*}
We're able to change the order of differentiation and integration here without affecting the result.
\begin{align*}
	&= \int_{0}^{\infty}{f(t)\frac{\mathrm{d}^n}{\mathrm{d}s^n}\left(e^{-st}\right) \mathrm{d}t} \\
	&= \int_{0}^{\infty}{f(t)(-t)^ne^{-st} \mathrm{d}t} \\
	&= (-1)^n\Laplace{t^nf(t)}.
\end{align*}
So,
\begin{equation*}
	\frac{\mathrm{d}^n}{\mathrm{d}s^n}\left(\Laplace{f(t)}\right) = (-1)^n\Laplace{t^nf(t)}.
\end{equation*}
This formula is useful in both directions: finding the derivatives of Laplace transforms, and finding the Laplace transforms of functions multiplied by $t^n$.
% Examples
\ifodd\includeLaplaceTransformExamples\begin{example}
	Find the Laplace transform of $e^{-t}\sin{(3t)}$.
\end{example}
\noindent
Using the translation property,
\begin{equation*}
	\Laplace{e^{-t}\sin{(3t)}} = \mathcal{L}\left\{\sin{(3t)}\right\}(s+1).
\end{equation*}
We from our $\sin$ formula that
\begin{equation*}
	\Laplace{\sin{(3t)}} = \frac{3}{s^2 + 3^2}.
\end{equation*}
So,
\begin{equation*}
	\Laplace{e^{-t}\sin{(3t)}} = \mathcal{L}\left\{\sin{(3t)}\right\}(s+1) = \frac{3}{(s+1)^2 + 3^2}.
\end{equation*}

\begin{example}
	Find the Laplace transform of $t^2(t^2 + 1)(t - 1)(t + 2)$.
\end{example}
\noindent
Using the derivative of the Laplace transform property,
\begin{equation*}
	\Laplace{t^2(t^2 + 1)(t - 1)(t + 2)} = (-1)^2\frac{\mathrm{d}^2}{\mathrm{d}s^2}\Laplace{(t^2 + 1)(t - 1)(t + 2)}.
\end{equation*}
Expanding out the polynomial,
\begin{equation*}
	 = \frac{\mathrm{d}^2}{\mathrm{d}s^2}\Laplace{t^4 + t^3 - t^2 + t - 2}.
\end{equation*}
Using the polynomial formula,
\begin{equation*}
	 = \frac{\mathrm{d}^2}{\mathrm{d}s^2}\left(\frac{4!}{s^5} + \frac{3!}{s^4} - \frac{2!}{s^3} + \frac{1!}{s^2} + \frac{2\cdot 0!}{s^1}\right).
\end{equation*}
Taking the second derivative,
\begin{equation*}
	 \Laplace{t^2(t^2 + 1)(t - 1)(t + 2)} = \frac{6!}{s^7} + \frac{5!}{s^6} - \frac{4!}{s^5} + \frac{3!}{s^4} + \frac{2\cdot 2!}{s^3}.
\end{equation*}

\begin{example}
	Find the Laplace transform of $\left(x^2\right)''$ using the derivative formula. Show it's equal to $\Laplace{2}$.
\end{example}
\noindent
Using the derivative formula,
\begin{equation*}
	\Laplace{\left(x^2\right)''} = s^2\Laplace{x^2} - s\left(x^2\right)'_{x=0} - \left(x^2\right)_{x=0}.
\end{equation*}
Using the polynomial formula,
\begin{equation*}
 	= s^2\frac{2!}{s^3} - 0 - 0.
\end{equation*}
Simplifying,
\begin{equation*}
	\Laplace{\left(x^2\right)''} = \frac{2}{s} = \Laplace{2}.
\end{equation*}

\begin{example}
	Find the Laplace transform of $e^{it}$. Show that it's equal to $\Laplace{\cos{t} + i\sin{t}}$.
\end{example}
Using the exponential formula,
\begin{equation*}
	\Laplace{e^{it}} = \frac{1}{s - i}.
\end{equation*}
Now we'll work on $\cos{t} + i\sin{t}$.
Using the linearity of the Laplace transform,
\begin{equation*}
	\Laplace{\cos{\theta} + i\sin{\theta}} = \Laplace{\cos{t}} + i\Laplace{\sin{t}}.
\end{equation*}
Using the $\sin$ and $\cos$ formulas,
\begin{equation*}
	= \frac{s}{s^2 + 1^2} + i\frac{1}{s^2 + 1^2}.
\end{equation*}
Combining into one fraction,
\begin{equation*}
	= \frac{s+i}{s^2 + 1^2}.
\end{equation*}
Although we normally avoid it, we can factor the denominator into its two complex roots,
\begin{equation*}
	= \frac{s+i}{(s+i)(s-i)}.
\end{equation*}
Canceling out common factors we see that we get the same Laplace transform as we did with $e^{it}$.
\begin{equation*}
	\Laplace{\cos{\theta} + i\sin{\theta}} = \frac{1}{s-i} = \Laplace{e^{it}}.
\end{equation*}
\fi

%\noindent
%The Laplace transforms of these common functions are summarized in the appendix.