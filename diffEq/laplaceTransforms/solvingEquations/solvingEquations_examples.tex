\begin{example}
	Solve the following IVP using a Laplace transform.
	\begin{equation*}
		\begin{cases}
			y'' - 2y' - 3y = 0 \\
			y(0) = 1 \\
			y'(0) = 1
		\end{cases}
	\end{equation*}
\end{example}
\noindent
Taking the Laplace transform of both sides,
\begin{align*}
	\Laplace{y'' - 2y' - 3y} = \Laplace{y''} - 2\Laplace{y'} - 3\Laplace{y} &= 0 \\
	\left(s^2\Laplace{y} - sy'(0) - y(0)\right) - 2\left(s\Laplace{y} - y(0)\right) - 3\left(\Laplace{y}\right) &= 0 \\
	= \Laplace{y}\left(s^2 - 2s - 3\right) &= s - 1.
\end{align*}
Note how the auxiliary equation appears in terms of $s$ here.\\
Solving for $\Laplace{y}$,
\begin{equation*}
	\Laplace{y} = \frac{s - 1}{s^2 - 2s - 3}.
\end{equation*}
Rearranging the right side into partial fractions,
\begin{equation*}
	\Laplace{y} = \frac{1/2}{s+1} + \frac{1/2}{s-3}.
\end{equation*}
Taking the inverse Laplace transform of both sides,
\begin{equation*}
	y = \frac{1}{2}e^{-t} + \frac{1}{2}e^{3t}.
\end{equation*}

\begin{example}
	Solve the following IVP using a Laplace transform.
	\begin{equation*}
		\begin{cases}
			y'' + 2y' + 5y = 0 \\
			y(0) = 1 \\
			y'(0) = 5
		\end{cases}
	\end{equation*}
\end{example}
\noindent
Taking the Laplace transform of both sides and solving for $\Laplace{y}$,
\begin{align*}
	\Laplace{y''} + 2\Laplace{y'} + 5\Laplace{y} &= 0 \\
	\Laplace{y}\left(s^2 + 2s + 5\right) - y'(0) - sy(0) - 2y(0) &= 0 \\
	\Laplace{y}\left(s^2 + 2s + 5\right) &= s + 7 \\
	\Laplace{y} &= \frac{s + 7}{s^2 + 2s + 5}.
\end{align*}
The denominator of the right hand side cannot be factored into linear terms.
Instead, we'll complete the square.
\begin{equation*}
	= \frac{s + 7}{(s+1)^2 + 2^2}
\end{equation*}
If the numerator was $s+1$, we'd have a $\cos$ that's been shifted left by 1 in the $s$ domain by -1.
If the numerator were a multiple of 2, we'd have a $\sin$ that's been similarly shifted.
We can rewrite the right hand side as two terms: one for $\cos$ and the other for $\sin$ with the appropriate numerators.
\begin{equation*}
	= \frac{s+1}{(s+1)^2 + 2^2} + 3\frac{2}{(s+1)^2 + 2^2}
\end{equation*}
Taking the inverse Laplace transform of both sides,
\begin{equation*}
	y = e^{-t}\cos{(2t)} + 3e^{-t}\sin{(2t)}.
\end{equation*}

\noindent
We can also use the Laplace transform on non-homogeneous equations.
The method will even take care of the cases where what would be our guess for the particular solution is already included in the homogeneous solution.
\begin{example}
	Solve the following IVP using a Laplace transform.
	\begin{equation*}
		\begin{cases}
			y'' + 2y' + 5y = e^{-t} \\
			y(0) = 1 \\
			y'(0) = 5
		\end{cases}
	\end{equation*}
\end{example}
\noindent
This equation is the same as the previous example except for the $e^{-t}$, so we can use some of our previous work.
\begin{align*}
	\Laplace{y}(s^2 + 2s + 5) &= s + 7 + \frac{1}{s+1} \\
	\Laplace{y} &= \frac{s^2 + 8s + 8}{(s+1)(s^2 + 2s + 5)}.
\end{align*}
Rearranging the right hand side into partial fractions with completed squares,
\begin{equation*}
	\Laplace{y} = \frac{1}{4}\left(\frac{1}{s+1}\right) + \frac{1}{4}\left(3\frac{s+1}{(s+1)^2 + 2^2} + 9\frac{2}{(s+1)^2 + 2^2}\right).
\end{equation*}
Taking the inverse Laplace transform of both sides,
\begin{equation*}
	y = \frac{1}{4}e^{-t} + \frac{3}{4}e^{-t}\cos{(2t)} + 3e^{-t}\sin{(2t)}.
\end{equation*}

\begin{example}
	Solve the following IVP using a Laplace transform.
	\begin{equation*}
		\begin{cases}
			8\cos{(2t)} - y'' = 4y\\
			y(0) = 1\\
			y'(0) = 0
		\end{cases}
	\end{equation*}
\end{example}
\noindent
Rearranging the equation into the standard form for a linear heterogeneous ODE,
\begin{equation*}
	y'' + 4y = 8\cos{(2t)}.
\end{equation*}
Taking the Laplace transform of both sides,
\begin{equation*}
	s^2\Laplace{y} - y'(0) - sy(0) + 4\Laplace{y} = 8\frac{s}{s^2 + 2^2}.
\end{equation*}
Plugging in the values given by the IVP and rearranging,
\begin{equation*}
	\Laplace{y}\left(s^2 + 4\right) = \frac{s^3 + 12s}{s^2 + 2^2}.
\end{equation*}
Dividing to solve for $\Laplace{y}$,
\begin{equation*}
	\Laplace{y} = \frac{s^3 + 12s}{(s^2 + 2^2)(s^2 + 2^2)}.
\end{equation*}
Rearranging the right side into partial fractions,
\begin{equation*}
	\Laplace{y} = \frac{s}{s^2 + 2^2} + 2\frac{2^2s}{(s^2+2^2)^2}.
\end{equation*}
Taking the inverse Laplace transform of both sides,
\begin{equation*}
	y = \cos{(2t)} + 2t\sin{(2t)}.
\end{equation*}