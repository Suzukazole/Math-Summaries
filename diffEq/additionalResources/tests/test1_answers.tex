\subsection{Test 1 Answers}
\begin{enumerate}[label=\arabic*.]
	\item 
		Using implicit differentiation,
		\begin{equation*}
			\dd{y}{x}-\frac{1}{y}\dd{y}{x} = 2x.
		\end{equation*}
		So,
		\begin{equation*}
			\dd{y}{x} = \frac{2x}{1-1/y} = \frac{2xy}{1-y}.
		\end{equation*}
		So, $y - \ln{y} = x^2 + 1$ is an implicit solution to $\dd{y}{x} = \frac{2x}{y-1}$ as we have demonstrated by differentiating and checking.
	\item
		We will use the integrating factor method. Rewriting the equation so all terms involving $y$ are on the left,
		\begin{equation*}
			y' + 4y = e^{-x}.
		\end{equation*}
		In this case,
		\begin{equation*}
			a(x) = 4 \text{ and } b(x) = e^{-x}.
		\end{equation*}
		So,
		\begin{equation*}
			\mu(x) = e^{\int{a(x) \mathrm{d}x}} = e^{4x}.
		\end{equation*}
		Multiplying both sides by $\mu(x)$,
		\begin{equation*}
			\left(e^{4x}y\right)' = e^{3x}.
		\end{equation*}
		Integrating both sides,
		\begin{equation*}
			e^{4x}y = \frac{1}{3}e^{3x} + C_1.
		\end{equation*}
		Solving for $y$,
		\begin{equation*}
			y = \frac{1}{3}e^{-x} + C_1e^{-4x}.
		\end{equation*}
		Plugging in $x = 0$ and $y=\frac{4}{3}$ to solve for $C_1$,
		\begin{equation*}
			\frac{4}{3} = \frac{1}{3} + C_1 \implies C_1 = 1.
		\end{equation*}
		So, our answer to the IVP is
		\begin{equation*}
			y = \frac{1}{3}e^{-x} + e^{-4x}.
		\end{equation*}
	\item
		The equation is separable and can be rewritten as
		\begin{equation*}
			\frac{x+2}{x} \mathrm{d}x = (t-3)^2 \mathrm{d}t \text{, } x\neq 0.
		\end{equation*}
		We'll come back later to see if $x = 0$ is a solution.
		Integrating both sides,
		\begin{equation*}
			x + 2\ln{\abs{x}} = \frac{(t-3)^3}{3} + C_1 \text{, } x\neq 0.
		\end{equation*}
		Plugging in $x = -1$ and $t = 3$ to solve for $C_1$,
		\begin{equation*}
			1 + 2\ln{1} = 0 + C_1 \implies C_1 = -1 \text{, } x\neq 0.
		\end{equation*}
		So, our solution to the IVP is
		\begin{equation*}
			x + 2\ln{\abs{x}} = \frac{(t-3)^3}{3} - 1 \text{, } x\neq 0.
		\end{equation*}
		Checking if $x = 0$ is a solution,
		\begin{equation*}
			0 = 0\frac{(t-3)^2}{2}.
		\end{equation*}
		So, $x = 0$ is a general solution.
		However for our solution $x(t) = 0$, $x(3) \neq -1$, to $x = 0$ is not a solution to the IVP.
		So, our solution to the IVP remains.
	\item
		This equation is linear and homogeneous, so we can find the general solution using the auxiliary equation. Extracting the auxiliary equation and finding the roots,
		\begin{equation*}
			r^3 + r^2 - 2 = 0 \implies r = 1, -1 \pm i.
		\end{equation*}
		So, the general solution is
		\begin{equation*}
			y = C_1e^{x} + C_2e^{-x}\cos{x} + C_3e^{-x}\sin{x}.
		\end{equation*}
	\item
		\begin{enumerate}[label=(\alph*)]
			\item
				As stated in the problem, 90kg of salt is in the tank initially.
			\item
				So find the amount of salt in the tank after 2 hours we'll need to set up a differential equation that models the situation.
				Let $y(t)$ be the number of kgs of salt in the tank after $t$ minutes. Let $V(t)$ be the volume of brine in the tank after $t$ minutes.
				Modeling the salt,
				\begin{equation*}
					\dd{y}{t} = \text{salt rate in} - \text{salt rate out} = 0 - \text{salt rate out}.
				\end{equation*}
				Modeling the volume,
				\begin{equation*}
					\dd{V}{t} = \text{brine rate in} - \text{brine rate out} = 3\frac{L}{min} - 6\frac{L}{min} = -3\frac{L}{min}.
				\end{equation*}
				We're also given that $V(0) = 2000L$, so we can find that
				\begin{equation*}
					V(t) = 2000 - 3t.
				\end{equation*}
				Now we can write an equation to find $\text{salt rate out}$.
				\begin{equation*}
					\text{salt rate out} = \frac{y \text{kg of salt}}{V \text{L of brine}}*\frac{6L}{\text{min}} = \frac{6y}{2000-3t}.
				\end{equation*}
				This allows us to write a differential equation for $y(t)$.
				\begin{equation*}
					\dd{y}{t} = -\text{salt rate out} = \frac{-6y}{2000-3t} = \frac{6y}{3t-2000}.
				\end{equation*}
				This equation is separable and can be rewritten as
				\begin{equation*}
					\frac{\mathrm{d}y}{y} = \frac{6\mathrm{d}t}{3t-200}.
				\end{equation*}
				Integrating both sides,
				\begin{equation*}
					\ln{\abs{y}} = 2\ln{\abs{3t-2000}}+C_1.
				\end{equation*}
				We know that $y > 0$ always because we can't have a negative amount of salt.
				Further, we're concerned with the time between $t= 0$ and $t = 120 < \frac{2000}{3}$, so $3t-2000 < 0$. This means we can rewrite our equation as
				\begin{equation*}
					\ln{y} = 2\ln{(2000-3t)} + C_1.
				\end{equation*}
				Exponentiating both sides,\footnote{$C_1$ is different from the $C_1$ previously, but it's still a constant.}
				\begin{equation*}
					y = C_1(2000-3t)^2.
				\end{equation*}
				Applying our initial condition of $y(0)=90$ to solve for $C_1$,
				\begin{equation*}
					90 = C_1(2000)^2 \implies C_1 = \frac{2000^2}{90}.
				\end{equation*}
				Plugging our solution for $C_1$ back into our general solution,
				\begin{equation*}
					y = 90\left(\frac{2000-3t}{2000}\right)^2.
				\end{equation*}
		\end{enumerate}
	\item
		Newton's law of cooling is
		\begin{equation*}
			\dd{T}{t} = k\left(T_e - T\right)
		\end{equation*}
		where $k$ is some constant, $T_e$ is the temperature of the environment, and $T$ is the current temperature of the object.
		In the detective's case, $T_e = 16$, so
		\begin{equation*}
			\dd{T}{t} = k(16-T)
		\end{equation*}
		with initial conditions $T(12) = 34$ and $T(13) = 32$.
		One could then solve this equation and apply the initial conditions to solve for $k$ and the constant of integration $C$.
		Then one would need to solve for $x$ in $T(x) = 37$.
		This value of $x$ will tell you the number of hours after midnight one the same day that the murder took place\footnote{For those that do solve the equation, $x \approx 10.691$, so the murder should have took place at about 10:41am.}.
\end{enumerate}