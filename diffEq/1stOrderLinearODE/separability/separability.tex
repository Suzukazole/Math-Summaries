\section{Separable Differential Equations}
\noindent
The most basic approach for solving a 1st-order differential equation is simply integrating both sides. You're probably already familiar with this technique from taking indefinite integrals. This approach only works when the independent and dependent variables can be arranged on different sides of the equation. We'll formalize this idea with separability.

\begin{definition}
	A 1st order ODE is separable if it can be written in the form
	\begin{equation*}
		\frac{\mathrm{d} y}{\mathrm{d} x} = f(x)g(y)
	\end{equation*}
\end{definition}

\noindent
Separable equations provide a special way of solving them that can be useful. If we treat the derivative like a fraction (which is not formally allowed but OK here),
\begin{equation*}
	\frac{\mathrm{d} y}{\mathrm{d} x} = f(x)g(y) \implies \frac{\mathrm{d} y}{g(y)} = f(x) \mathrm{d}x \implies \int{\frac{\mathrm{d} y}{g(y)}} = \int{f(x) \mathrm{d}x}
\end{equation*}
We then have a function in $y$ on the left and a function in $x$ on the right, meaning we only have to solve for $y$ to get the solution.

\begin{example}
	Solve the following 1st order ODE using separation of variables.
	\begin{equation*}
		\dd{y}{x} = \frac{5}{xy^3}
	\end{equation*}
\end{example}
\noindent
Separating so all terms involving $y$ are on the left,
\begin{equation*}
	y^3 \mathrm{d}y = \frac{5}{x} \mathrm{d}x
\end{equation*}
Integrating,
\begin{equation*}
	\int{y^3 \mathrm{d}y} = \int{\frac{5}{x} \mathrm{d}x} \implies \frac{y^4}{4} = 5\ln{x} + C \implies y = \sqrt[4]{20\ln{x} + C}
\end{equation*}

\begin{example}
	Solve the following IVP using separation of variables.
	\begin{equation*}
		\begin{cases}
			\dd{y}{x} = 2y^2 + xy^2 \\
			y(0) = 1
		\end{cases}
	\end{equation*}
\end{example}
\noindent
Separating,
\begin{equation*}
	\frac{\mathrm{d}y}{y^2} = \left(2 + x\right) \mathrm{d}x \text{, or } y = 0
\end{equation*}
We assume that $y \neq 0$ when dividing, but we need to be careful to include $y = 0$ as a possible solution and check if it satisfies the differential equation and initial conditions.\\

\noindent
Integrating,
\begin{equation*}
	\int{\frac{\mathrm{d}y}{y^2}} = \int{\left(2 + x\right) \mathrm{d}x} \implies \frac{-1}{y} = 2x + \frac{x^2}{2} + C
\end{equation*}
Solving for $y$,
\begin{equation*}
	y = \frac{-2}{x^2 + 4x + C}
\end{equation*}
We ignored $y = 0$, so we need to go back and check if it's a solution to the differential equation.
\begin{equation*}
	0 = 2(0)^2 + x(0)^2
\end{equation*}
Since it's a solution to the differential equation, we'll see if its satisfies the initial conditions of the IVP.
\begin{equation*}
	y(0) = 0 \neq 1 \text{, } y = 0
\end{equation*}
So, $y = 0$ is not a solution to the IVP.\\

\noindent
Solving for $C$ using the initial conditions,
\begin{equation*}
	y(0) = \frac{-2}{(0)^2 + 4(0) + C} = 1 \implies C = -2
\end{equation*}
So,
\begin{equation*}
	y = \frac{-2}{x^2 + 4x -2}
\end{equation*}
is the solution to the IVP.