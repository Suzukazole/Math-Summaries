\section{Integrating Factor Method}
All 1st order linear differential equations have the form
\begin{equation*}
	a_1(x)\dd{y}{x} + a_0(x)y = b_1(x),
\end{equation*}
which can be rewritten as
\begin{equation*}
	\dd{y}{x} + a(x)y = b(x).
\end{equation*}

\noindent
This equation isn't always separable, and we can't just integrate both sides unless $a_1(x)y$ = 0.\\

\noindent
If $a_0(x) = a_1'(x)$, then we could rewrite the equation and solve by doing the product rule in reverse.
\begin{equation*}
	\left(a_1(x)y\right)' = b_1(x) \implies y = \frac{\int{b_1(x) \mathrm{d}x}}{a_1(x)}
\end{equation*}

\noindent
It's possible to rearrange into this form by multiplying the equation by some function. Specifically, what were looking for is a function $\mu(x)$ such that
\begin{equation*}
	\mu(x)\dd{y}{x} + \mu(x)a(x)y = \mu(x)b(x) \text{ and } \mu'(x) = \mu(x)a(x).
\end{equation*}
This equation involving $\mu(x)$ is one that we know how to solve because it's separable.\footnote{Although we are taking an indefinite integral to find $\mu(x)$, we do not have a $+ C$ term.}
\begin{equation*}
	\mu'(x) = \mu(x)a(x) \implies \mu(x) = e^{\int{a(x) \mathrm{d}x}}
\end{equation*}
Substituting the solution for $\mu(x)$ back,
\begin{equation*}
	e^{\int{a(x) \mathrm{d}x}}\dd{y}{x} + e^{\int{a(x) \mathrm{d}x}}a(x)y = e^{\int{a(x) \mathrm{d}x}}b(x) \implies e^{\int{a(x) \mathrm{d}x}}\dd{y}{x} + \mu'(x)y = e^{\int{a(x) \mathrm{d}x}}b(x).
\end{equation*}
Applying the product rule in reverse,
\begin{equation*}
	y = \frac{\int{\mu(x)} b(x) \mathrm{d}x}{\mu(x)} \text{, } \mu(x) = e^{\int{a(x) \mathrm{d}x}}.
\end{equation*}

\begin{example}
	Solve the following 1st order linear ODE.
	\begin{equation*}
		y^\prime - y = 2e^{x}
	\end{equation*}
\end{example}
\noindent
$a(x) = -1$, so
\begin{equation*}
	\mu(x) = e^{\int{-1 \mathrm{d}x}} = e^{-x}
\end{equation*}
Applying to our equation and solving,
\begin{equation*}
	y^\prime e^{-x} - ye^{-x} = 2 \implies \left(ye^{-x}\right)^\prime = 2 \implies ye^{-x} = 2x + C \implies y = \frac{2x + C}{e^{-x}}.
\end{equation*}