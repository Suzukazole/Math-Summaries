\begin{example}
	Let's try to solve the following equation by guessing and checking likely solutions.
	\begin{equation*}
		y'' - 3y' + 2y = 0
	\end{equation*}
\end{example}
\noindent
Exponentials seem like good guesses. Let's try an exponential of the form $y = Ce^{rx}$ first.
\begin{equation*}
	\left(Ce^{rx}\right)'' - 3\left(Ce^{rx}\right)' + 2\left(Ce^{rx}\right) =  0
\end{equation*}
\begin{equation*}
	Cr^2e^{rx} - 3rCe^{rx} + 2Ce^{rx} = Ce^{rx}\left(r^2 - 3r + 2\right) = 0
\end{equation*}
Since $Ce^{rx} \neq 0$ unless $C = 0$, we only need to solve the quadratic. Note that the coefficients of the quadratic are the same as the coefficients in the original differential equation.
\begin{equation*}
	r^2 - 3r + 2 = 0 \implies r = 1 \text{, } 2
\end{equation*}
So, our two fundamental solutions are
\begin{equation*}
	\begin{cases}
		y = C_1e^{x} \\
		y = C_2e^{2x}
	\end{cases}
\end{equation*}
Since these two fundamental solutions are linearly independent, the general solution is the sum of the fundamental solutions.
\begin{equation*}
	y = C_1e^{x} + C_2e^{2x}
\end{equation*}