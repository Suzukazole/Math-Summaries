\subsection{The Auxiliary Equation}
\noindent
It's not a coincidence that the coefficients of the polynomial that we had to find the 0's of in the above example matched the coefficients of the differential equation. We call this polynomial the auxiliary equation, and it can help us solve linear, homogeneous differential equations with constant coefficients of any order.
\begin{definition}
	A nth order, linear, homogeneous differential equation with constant coefficients has the form
	\begin{equation*}
		a_ny^{(n)} + a_{n-1}y^{(n-1)} + \ldots + a_1y^\prime + a_0y = 0
	\end{equation*}
	The corresponding auxiliary equation is
	\begin{equation*}
		a_nr^n + a_{n-1}r^{n-1} + \ldots + a_1r + a_0 = 0
	\end{equation*}
\end{definition}

\noindent
We now have a method for solving these equations with the roots of the auxiliary equation are all unique. 
\begin{theorem}
	Let $\left\{r_1, \ldots, r_n\right\}$ be the set of unique roots to an auxiliary equation corresponding to a nth order, linear, homogeneous differential equation with constant coefficients. The set of fundamental solutions are $\left\{C_1e^{r_1x}, \ldots, C_ne^{r_nx}\right\}$, and the general solution is
	\begin{equation*}
		y = C_1e^{r_1x} + \ldots + C_ne^{r_nx}
	\end{equation*}
\end{theorem}

\noindent
We can easily extend this method to deal with roots of higher multiplicities.
\begin{theorem}
	Let $\alpha$ be a root with multiplicity $k$ to an auxiliary equation corresponding to a nth order, linear, homogeneous differential equation with constant coefficients. Then $e^{\alpha x}, xe^{\alpha x}, \ldots, x^{k-1}e^{\alpha x}$ are fundamental solutions.
\end{theorem}

\subsubsection{Complex Roots}
\noindent
Although the previous two theorems already cover complex roots, we can simplify solutions that have complex roots into functions we more easily understand.\\

\begin{theorem}
	If an auxiliary equation has roots $\alpha \pm \beta i$, then $C_1e^{\alpha x}\cos{(\beta x)}$ and  $C_2e^{\alpha x}\sin{(\beta x)}$ are fundamental solutions.
\end{theorem}
\begin{proof}
	The two corresponding fundamental solutions are
	\begin{equation*}
		C_1e^{(\alpha + \beta i)x} \text{, } C_2e^{(\alpha - \beta i)x}
	\end{equation*}
	so the part of the general solution for these two fundamental solutions are
	\begin{equation*}
		C_1e^{(\alpha + \beta i)x} + C_2e^{(\alpha - \beta i)x}
	\end{equation*}
	Using Euler's formula, we can also write this as
	\begin{equation*}
		C_1 e^{\alpha x}\left( \cos{(\beta t)} + i\sin{(\beta x)} \right) + C_2e^{\alpha x}\left( \cos{(-\beta x)} + i\sin{(-\beta x)} \right)
	\end{equation*}
	Using the fact that $\sin{(-x)} = -\sin{x}$, $\cos{(-x)} = -\cos{x}$ and separating into real and imaginary parts,
	\begin{equation*}
		 = \left[ C_1e^{\alpha x}\cos{(\beta x)} + C_2e^{\alpha x}\cos{(\beta x)} \right] + i\left[ C_1e^{\alpha x}\sin{(\beta x)} - C_2e^{\alpha x}\sin{(\beta x)} \right]
	\end{equation*}
	Simplifying \footnote{\label{note1}In this step the values of $C_1$ and $C_2$ might have changed, but they are still constants.},
	\begin{equation*}
		= e^{\alpha x}\left(C_1\cos{(\beta x)} + iC_2\sin{(\beta x)}\right)
	\end{equation*}
	At this point, we need to consider if the constants are real or imaginary. We would then take the real part as a fundamental solution.\\
	If they are both real, then a fundamental solution is\footnote{See footnote \ref{note1}}
	\begin{equation*}
		C_1e^{\alpha x}\cos{(\beta x)}
	\end{equation*}
	where $C_1$ is a real constant.
	If they are both imaginary, the a fundamental solution is\footnote{See footnote \ref{note1}}
	\begin{equation*}
		C_2e^{\alpha x}\sin{(\beta x)}
	\end{equation*}
	where $C_2$ is a real constant.
\end{proof}

\noindent
If complex roots are repeated, we just add the appropriate number of powers of $x$ in front of both the $\sin{(\beta x)}$ and $\cos{(\alpha a)}$ parts.
\pagebreak % It looked weird having the example start and then the footnote immediately after, so now the example starts on the next page

\begin{example}
	Find the general solution to the following differential equation.
	\begin{equation*}
		y^{(4)} + 2y^{\prime\prime} + y = 0
	\end{equation*}
\end{example}
First we extract the auxiliary equation and find its roots.
\begin{equation*}
	r^4 + 2r^2 + 1 = 0 \implies r = i (\text{double root}), -i (\text{double root})
\end{equation*}
Since we have complex roots $0 \pm 1i$, we know that the following are fundamental solutions
\begin{equation*}
	C_1e^{0x}\cos{1x} \to C_1\cos{x} \text{ and } C_2e^{0x}\sin{1x} \to C_2\sin{x}
\end{equation*}
Since both roots are double roots the following are also fundamental solutions.
\begin{equation*}
	C_3x\cos{x} \text{ and } C_4x\sin{x}
\end{equation*}
So, the general solution is
\begin{equation*}
	y = C_1\cos{x} + C_2\sin{x} + C_3x\cos{x} + C_4x\sin{x}
\end{equation*}