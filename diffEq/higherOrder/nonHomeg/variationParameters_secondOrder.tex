\subsubsection{Second Order Variation of Parameters}
\noindent
We'll modify our second order equation of have a 1 as the coefficient of the $y^{\prime\prime}$ term by dividing to get an equation of the form
\begin{equation*}
	y^{\prime\prime} + py^\prime + qy = g(x)
\end{equation*}
Just like for undetermined coefficients, we'll find homogeneous and particular solutions $y = y_h + y_p$. Since the equation is second-order, the solution to the homogeneous equation will yield two fundamental solutions $y_1$ and $y_2$ where $y_h = C_1y_1 + C_2y_2$.\\

\noindent
So, we can write $y$ as
\begin{equation*}
	y(x) = A(x)y_1 + B(x)y_2
\end{equation*}
where
\begin{equation*}
	\begin{cases}
		A^\prime y_1 + B^\prime y_2 = 0 \\
		A^\prime y_1^\prime + B^\prime y_2^\prime = g(x)
	\end{cases}
\end{equation*}
We will then solve this system to solve for $A^\prime$ and $B^\prime$ and integrate.

\ifodd\includeHigherOrderExamples\begin{example}
	Find the general solution to the following equation
	\begin{equation*}
		y'' + y = \csc{x}
	\end{equation*}
	given that $y_h = C_1\cos{x} + C_2\sin{x}$.
\end{example}
\noindent
$y_h$ gives us our two fundamental solutions
\begin{equation*}
	\begin{cases}
		y_1 = \cos{x} \\
		y_2 = \sin{x}
	\end{cases}
\end{equation*}
So, our system of equations is
\begin{equation*}
	\begin{cases}
		A'\cos{x} + B'\sin{x} = 0 \\
		-A'\sin{x} + B'\cos{x} = \csc{x}
	\end{cases} \to \begin{cases}
		A'\cos{x} + B'\sin{x} = 0 \\
		-A'\cos{x} + B'\frac{\cos^{2}{x}}{\sin{x}} = \frac{\cos{x}}{\sin^{2}{x}}
	\end{cases}
\end{equation*}
So, 
\begin{equation*}
	B'\frac{1}{\sin{x}} = \frac{\cos{x}}{\sin^{2}{x}} \implies B' = \frac{\cos{x}}{\sin{x}} = \cot{x} \implies B = \ln{\abs{\sin{x}}} + C_2
\end{equation*}
and
\begin{equation*}
	A'\cos{x} + \cos{x} = 0 \implies A' = -1 \implies A = -x + C_1
\end{equation*}
So, our general solution is
\begin{equation*}
	y = \left(C_1 - x\right)\cos{x} + \left(\ln{\abs{\sin{x}}} + C_2\right)\sin{x} = C_1\cos{x} + C_2\sin{x} - x\cos{x} + \sin{x}\ln{\abs{\sin{x}}}
\end{equation*}
Note how $y_h$ and $y_p$ appear together.\fi