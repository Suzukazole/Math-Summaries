\subsubsection{Higher Order Variation of Parameters}
\noindent
We are trying to find a solution to the nth order linear ODE
\begin{equation*}
	a_n(x)y^{(n)} + \ldots + a_0(x)y = g(x)
\end{equation*}
assuming that we already know the fundamental solutions for the corresponding homogeneous equation
\begin{equation*}
	y_h = C_1y_1 + \ldots + C_ny_n.
\end{equation*}

\noindent
For this method, we'll assume that $y$ can be written as
\begin{equation*}
	y = v_1(x)y_1 + \ldots + v_n(x)y_n
\end{equation*}
and we'll try to find $v_1, \ldots, v_n$.\\

\noindent
Since there are $n$ unknown functions, we'll need $n$ equations to find them all.
We can generate these by differentiating $y_p$.
\begin{equation*}
	y' = \left(v_1y_1' + \ldots + v_ny_n'\right) + \left(v_1'y_1 + \ldots + v_n'y_n\right)
\end{equation*}
to avoid second derivatives of $v_1, \ldots, v_n$ from entering the formula for $y''$, we also have the condition
\begin{equation*}
	v_1'y_1 + \ldots +v_n'y_n = 0.
\end{equation*}
We can now continue differentiating to get $n-2$ more equations involving $v_1', \ldots, v_n'$.
We also impose a final $n^{\text{th}}$ condition that
\begin{equation*}
	v_1'y^{(n-1)} + \ldots + v_n'y_n^{(n-1)} = g.
\end{equation*}
This gets us a system of $n$ equations,
\begin{equation*}
	\begin{cases}
		v_1'y_1 + \ldots + v_n' y_n & = 0 \\
		\vdots & \vdots \\
		v_1'y_1^{(n-2)} + \ldots + v_n'y_n^{(n-2)} & = 0 \\
		v_1'y_1^{(n-1)} + \ldots + v_n'y_n^{(n-1)} & = g
	\end{cases} .
\end{equation*}
Hopefully this system looks familiar from second order equations.\\

\noindent
We can rewrite this system in terms of matrices and vectors.
\begin{equation*}
	\begin{bmatrix}
		y_1 & \ldots & y_n \\
		\vdots & & \vdots\\
		y_1^{(n-2)} & \ldots & y_n^{(n-2)} \\
		y_1^{(n-1)} & \ldots & y_n^{(n-1)}
	\end{bmatrix} \begin{bmatrix}
		v_1' \\
		\vdots \\
		v_{n-1}' \\
		v_n'
	\end{bmatrix} = \begin{bmatrix}
		0 \\
		\vdots \\
		0 \\
		g
	\end{bmatrix}.
\end{equation*}
It's sufficient to show that a solution to this system exists if the determinant of the square matrix on the left is non-zero. The determinant of this matrix actually has a special name.

\begin{definition}
	The Wronskian of $n$ $n-1$ times differentiable functions $\left\{f_1, \ldots, f_n\right\}$ on an interval $I$ is
	\begin{equation*}
		W[f_1, \ldots, f_n](x) = \begin{vmatrix}
			f_1(x) & \ldots & f_n(x) \\
			f_1'(x) & \ldots & f_n'(x) \\
			\vdots &        & \vdots \\
			f_1^{(n-1)}(x) & \ldots & f_n^{(n-1)}(x)
		\end{vmatrix} \text{, } x \in I.
	\end{equation*}
\end{definition} 

\noindent
Using the Wronskian, we can solve the system using Cramer's Rule.
\begin{equation*}
	v'_i(x) = \frac{g(x)W_i(x)}{W[y_1, \ldots, y_n](x)} \text{, } i = 1, \ldots, n,
\end{equation*}
where $W_i(x)$ is the determinant of the matrix obtained from the Wronskian $W(x)$ by replacing the $i^{\text{th}}$ column with $\text{col}[0, \ldots, 1]$. Using the cofactor expansion along this column, we can write $W_i(x)$ as
\begin{equation*}
	W_i(x) = (-1)^{n-i}W[y_1, \ldots, y_{i-1}, y_{i+1}, \ldots, y_n](x) \text{, } i = 1, \ldots, n.
\end{equation*}
Now with a solution for $v_i'$, we can integrate to get $v_i$.
\begin{equation*}
	v_i = \int{\frac{g(x)W_i(x)}{W[y_1, \ldots, y_n](x)} \mathrm{d}x}.
\end{equation*}
Now with a solution for $v_i$, we can substitute back to find $y(x)$.
\begin{equation*}
	y(x) = \sum_{i=1}^{n}{y_i(x)\int{\frac{g(x)W_i(x)}{W[y_1, \ldots, y_n](x)} \mathrm{d}x}}
\end{equation*}

\begin{example}
	Find the general solution to the equation
	\begin{equation*}
		x^3y''' + x^2y'' - 2xy' = x^3\sin{x} \text{, } x > 0
	\end{equation*}
	given that $\left\{x, x^{-1}, x^2\right\}$ is the set of fundamental solutions.
\end{example}
\noindent
First we divide by $x^3$ to get a leading coefficient of 1.
\begin{equation*}
	y''' + x^{-1}y''' - 2x^{-2}y' = \sin{x} \text{, } x > 0.
\end{equation*}
Next we calculate the $W(x)$ and each $W_i(x)$ for the fundamental solution set.
\begin{align*}
	W[x,x^{-1}, x^2](x) &= \begin{vmatrix}
		x & x^{-1} & x^2 \\
		1 & -x^{-2} & 2x \\
		0 & 2x^{-3} & 2
	\end{vmatrix} = 6x^{-1} \\
	W_1(x) &= (-1)^{3-1}W[x^{-1}, x^2](x) = (-1)^{2}\begin{vmatrix}
		x^{-1} & x^2 \\
		-x^{-2} & 2x
	\end{vmatrix} = 3 \\
	W_2(x) &= (-1)^{3-2}\begin{vmatrix}
		x & x^{2} \\
		1 & 2x
	\end{vmatrix} = -x^2 \\
	W_2(x) &= (-1)^{3-2}\begin{vmatrix}
		x & x^{-1} \\
		1 & -x^{-2}
	\end{vmatrix} = -2x^{-1}.
\end{align*}
Now we can calculate $y$.
\begin{align*}
	y(x) &= x\int{\frac{(\sin{x})^3}{-6x^{-1}} \mathrm{d}x} + x^{-1}\int{\frac{(\sin{x})(-x^2)}{-6x^{-1}} \mathrm{d}x} + x^2\int{\frac{(\sin{x})(-2x^{-1})}{-6x^{-1}} \mathrm{d}x} \\
	&= x\int{\left(\frac{-1}{2}x\sin{x}\right)\mathrm{d}x} + x^{-1}\int{\left(\frac{1}{6}x^3\sin{x}\right)\mathrm{d}x} + x^2\int{\left(\frac{1}{3}\sin{x}\right)\mathrm{d}x} \\
	&= C_1x + c_2x^{-1} + C_3x^2 + \cos{x} - x^{-1}\sin{x}
\end{align*}