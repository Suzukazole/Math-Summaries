\subsection{Taylor Series}
\noindent
A Taylor series as a way of approximating a function about a point $x=a$ using polynomials. The first approximation just keeps the same value at $x=a$, the second approximation keep the same value and first derivative, etc. The definition is
\begin{equation*}
	f(x) = f(a)+f^\prime(a)(x-a) + \frac{f^{\prime\prime}(a)}{2!}(x-a)^2 + \ldots +  \frac{f^{(n)}(a)}{n!}(x-a)^n + \ldots
\end{equation*}
If we approximate a function about $x=0$, we call this a Maclaurin series.\\

\noindent
Below are some common Maclaurin series, and their radii of convergence if applicable.
\begin{enumerate}[label=]
	\item \begin{equation*}
		e^x = 1 + x + \frac{x^2}{2!} + \frac{x^3}{3!} + \ldots
	\end{equation*}
	\item \begin{equation*}
		\sin{x} = x - \frac{x^3}{3!} + \frac{x^5}{5!} - \ldots
	\end{equation*}
	\item \begin{equation*}
		\cos{x} = 1 - \frac{x^2}{2!} + \frac{x^4}{4!} - \ldots
	\end{equation*}
	\item \begin{equation*}
		\frac{1}{1+x} = 1 - x + x^2 - \ldots \text{, where } \abs{x} < 1
	\end{equation*}
	\item \begin{equation*}
		\ln{(1+x)} = x - \frac{x^2}{2} + \frac{x^3}{3} - \ldots \text{, where } \abs{x} < 1
	\end{equation*}
\end{enumerate}

\subsubsection{Euler's Identity}
\noindent
Let's see what happens when we look at the Maclaurin series for $e^{ix}$.
\begin{equation*}
	e^{ix} = 1 + (ix) + \frac{(ix)^2}{2!} + \frac{(ix)^3}{3!} + \frac{(ix)^4}{4!} + \frac{(ix)^5}{5!} \ldots
\end{equation*}
\begin{equation*}
	= 1 + ix - \frac{x^2}{2!} - i\frac{x^3}{3!} + \frac{x^4}{4!} + i\frac{x^5}{5!} - \ldots
\end{equation*}
\begin{equation*}
	= \left(1 - \frac{x^2}{2!} + \frac{x^4}{4!} - \ldots \right) + i\left(x - \frac{x^3}{3!} + \frac{x^5}{5!} - \ldots \right)
\end{equation*}
The two expressions in parenthesis are exactly the Maclaurin series for $\cos{x}$ and $\sin{x}$. So,
\begin{equation*}
	e^{ix} = \cos{x} + i\sin{x}
\end{equation*}
In the case that $x = \pi$,
\begin{equation*}
	e^{i\pi} = \cos{\pi} + i\sin{\pi} = -1 + 0
\end{equation*}
So,
\begin{equation*}
	e^{i\pi} + 1 = 0
\end{equation*}
