\subsubsection{Improper Fractions}
\noindent
If the degree of the numerator is greater than or equal to the degree of the denominator, we have a case of improper fractions. In this case, we have to do polynomial long division to get a quotient and remainder and then decompose the remainder if necessary. So,
\begin{equation*}
	\frac{P(x)}{Q(x)} = R(x) + \frac{S(x)}{Q(x)}
\end{equation*}

\begin{example}
	Find the partial fraction decomposition of $\frac{x^3+3}{x^2-2x-3}$.
\end{example}
First we do polynomial long division to find that
\begin{equation*}
	\frac{x^3+3}{x^2-2x-3} = x + 2 + \frac{7x+9}{x^2-2x-3}
\end{equation*}
Now that the numerator is of a lesser degree than the denominator, we can decompose it normally.
\begin{equation*}
	x^2-2x-3 = (x-3)(x+1)
\end{equation*}
so,
\begin{equation*}
	\frac{7x+9}{x^2-2x-3} = \frac{A_1}{x-3} + \frac{A_2}{x+1}
\end{equation*}
Multiplying each side by the denominator,
\begin{equation*}
	7x+9 = A_1(x+1) + A_2(x-3)
\end{equation*}
At $x=-1$,
\begin{equation*}
	2 = -4A_2 \implies A_2 = \frac{-1}{2}
\end{equation*}
at $x=3$,
\begin{equation*}
	30 = 4A_1 \implies A_1 = \frac{15}{2}
\end{equation*}
So,
\begin{equation*}
	\frac{x^3+3}{x^2-2x-3} = x + 2 + \frac{15/2}{x-3} + \frac{-1/2}{x+1}
\end{equation*}