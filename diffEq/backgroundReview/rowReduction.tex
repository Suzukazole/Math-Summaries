\subsection{Row Reduction}
\noindent
Row reduction is a way of solving a system of linear equations by representing the system as a matrix and altering the rows of the matrix until we get as close as possible to an identity matrix.\\

\noindent
Below is a list of legal row operations. Doing these does not change the solution to the system of equations.
\begin{enumerate}[label=]
	\item Multiplying or dividing each item in a row by a scalar.
		\begin{equation*}
			\begin{bmatrix}
				1 & 3 & 7 \\
				0 & 2 & -1 \\
				2 & 7 & 9
			\end{bmatrix}
			\stackrel{R_2 = R_2/2}{\to}
			\begin{bmatrix}
				1 & 3 & 7 \\
				0 & 1 & -1/2 \\
				2 & 7 & 9
			\end{bmatrix}
		\end{equation*}
	\item Adding a multiple of one row to another row.
		\begin{equation*}
			\begin{bmatrix}
				1 & 3 & 7 \\
				0 & 2 & -1 \\
				2 & 7 & 9
			\end{bmatrix}
			\stackrel{R_3 = R_3 - 2R_1}{\to}
			\begin{bmatrix}
				1 & 3 & 7 \\
				0 & 2 & -1 \\
				0 & 1 & -5
			\end{bmatrix}
		\end{equation*}
	\item Swapping two rows.
		\begin{equation*}
			\begin{bmatrix}
				1 & 3 & 7 \\
				0 & 2 & -1 \\
				0 & 1 & -5
			\end{bmatrix}
			\stackrel{\text{swap } R_2 \text{, } R_3}{\to}
			\begin{bmatrix}
				1 & 3 & 7 \\
				0 & 1 & -5 \\
				0 & 2 & -1
			\end{bmatrix}
		\end{equation*}
\end{enumerate}

\noindent
Using these rules, we solve a system of linear equations using a process called Gauss-Jordan Elimination.\\

\noindent
A system may have a contradiction, meaning no solution exists. This will look like a row of 0's on the left and a non-zero term on the far right of the row.
\begin{equation*}
	\left[
		\begin{array}{cc|c}
		1 & 0 & 1 \\
		0 & 1 & 0
		\end{array}
	\right]
	\implies
	\text{No solution}
\end{equation*}

\noindent
A system may be underdetermined, meaning one or more variables can be any number. This will look a non-zero column on the left without a leading 1 (bolded).
\begin{equation*}
	\left[
		\begin{array}{ccccc|c}
			1 & \textbf{-2} & 0 & 0 & \textbf{-3} & 2 \\
			0 & \textbf{0} & 1 & 0 & \textbf{1} & 5 \\
			0 & \textbf{0} & 0 & 1 & \textbf{2} & 4 \\
			0 & \textbf{0} & 0 & 0 & \textbf{0} & 0
		\end{array}
	\right]
	\implies
	\begin{bmatrix}
		x_1 \\
		x_2 \\
		x_3 \\
		x_4 \\
		x_5 \\
	\end{bmatrix} = \begin{bmatrix}
		2 \\
		0 \\
		5 \\
		4 \\
		0
	\end{bmatrix} + \alpha \begin{bmatrix}
		2 \\
		1 \\
		0 \\
		0 \\
		0
	\end{bmatrix} + \beta \begin{bmatrix}
		3 \\
		0 \\
		-4 \\
		-2 \\
		1
	\end{bmatrix} \text{, }\alpha\text{, }\beta \in \R
\end{equation*}

\ifodd\includeBackgroundReviewExamples\begin{example}
	Solve the following linear system of equations using row reduction.
	\begin{equation*}
		\begin{bmatrix}
			1 & 3 & 0 \\
			0 & 2 & 5 \\
			2 & 7 & 7
		\end{bmatrix} \vec{x} = \begin{bmatrix}
			4 \\
			13 \\
			16
		\end{bmatrix}
	\end{equation*}
\end{example}
\begin{equation*}
	\left[
		\begin{array}{ccc|c}
			1 & 3 & 7 & 0 \\
			0 & 2 & -1 & 5 \\
			2 & 7 & 9 & 7
		\end{array}
	\right] \stackrel{R_3 = R_3 - 2R_1}{\to} \left[
		\begin{array}{ccc|c}
			1 & 3 & 7 & 0 \\
			0 & 2 & -1 & 5 \\
			0 & 1 & -5 & 7
		\end{array}
	\right] \stackrel{\text{swap }R_2\text{, }R_3}{\to} \left[
		\begin{array}{ccc|c}
			1 & 3 & 7 & 0 \\
			0 & 1 & -5 & 7 \\
			0 & 2 & -1 & 5
		\end{array}
	\right]
\end{equation*}
\begin{equation*}
	\stackrel{R_3 = R_3 -2R_2}{\to} \left[
		\begin{array}{ccc|c}
			1 & 3 & 7 & 0 \\
			0 & 1 & -5 & 7 \\
			0 & 0 & 9 & -9
		\end{array}
	\right] \stackrel{R_3 = R_3/9}{\to} \left[
		\begin{array}{ccc|c}
			1 & 3 & 7 & 0 \\
			0 & 1 & -5 & 7 \\
			0 & 0 & 1 & -1
		\end{array} 
	\right] \stackrel{R_2 = R_2 + 5R_3}{\to} \left[
		\begin{array}{ccc|c}
			1 & 3 & 7 & 0 \\
			0 & 1 & 0 & 2 \\
			0 & 0 & 1 & -1
		\end{array}
	\right]
\end{equation*}
\begin{equation*}
	\stackrel{R_1 = R_1 - 7R_3}{\to} \left[
		\begin{array}{ccc|c}
			1 & 3 & 0 & 7 \\
			0 & 1 & 0 & 2 \\
			0 & 0 & 1 & -1
		\end{array}
	\right] \stackrel{R_1 = R_1 - 2R_2}{\to} \left[
		\begin{array}{ccc|c}
			1 & 0 & 0 & 1 \\
			0 & 1 & 0 & 2 \\
			0 & 0 & 1 & -1
		\end{array}
	\right].
\end{equation*}
So,
\begin{equation*}
	\vec{x} = \begin{bmatrix}
		1 \\
		2 \\
		-1
	\end{bmatrix}.
\end{equation*}\fi