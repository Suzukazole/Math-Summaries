\subsection{Exponentials \& Logarithms}
\begin{definition}
	e is the base of the natural logarithm. It's defined by the limit
	\begin{equation*}
		e = \lim\limits_{n\rightarrow\infty}{\left(1+\frac{1}{n}\right)^n}
	\end{equation*}
\end{definition}
$\exp{x} = e^x$ and $\ln{x}$ are inverse functions of each other such that
\begin{equation*}
	e^{\ln{x}} = x \text{, } \ln{e^x} = x
\end{equation*}

\noindent
Just like other exponents, the normal rules for adding, subtracting, and multiplying powers apply.
\begin{equation*}
	e^xe^y = e^{x+y}\text{, }\frac{e^x}{e^y}=e^{x-y}\text{, and }\left(e^x\right)^k=e^{xk}
\end{equation*}

\noindent
Similar rules apply for logarithms.
\begin{equation*}
	\ln{x}+\ln{y} = \ln{xy}\text{, }\ln{x}-\ln{y} = \ln{\left(\frac{x}{y}\right)}\text{, and }\ln{\left(a^b\right)}=b\ln{a}
\end{equation*}

\noindent
You can also change a logarithm of any base to a natural logarithm.
\begin{equation*}
	\log_{b}{a} = \frac{\ln{a}}{\ln{b}}
\end{equation*}

\noindent
$e$ is also unique in that it is the only real number $a$ satisfying the equation
\begin{equation*}
	\frac{\mathrm{d}}{\mathrm{d}x}a^x = a^x
\end{equation*}
meaning $e^x$ is its own derivative.