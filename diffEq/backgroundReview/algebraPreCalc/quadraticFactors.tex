\subsubsection{Quadratic Factors}
\noindent
If a quadratic doesn't have real roots, then we have a quadratic factor. Here, we'll assume that the quadratic factor isn't repeated. 
So, 
\begin{equation*}
	Q(x) = R(x)(ax^2+bx+c)\text{, }b^2-4ac < 0\text{, and }R(x)\text{ is not divisible by }ax^2+bx+c
\end{equation*}
In this case, we say
\begin{equation*}
	\frac{P(x)}{R(x)(ax^2+bx+c)} = \left(\text{Decomposition of }R(x)\right)+\frac{A_1x+B_1}{ax^2+bx+c}
\end{equation*}
and then solve for the constants in the numerator, possibly having to solve a system of equations or using previous resultss and less convenient values for $x$.

\begin{example}
	Find the partial fraction decomposition of $\frac{6x^2+21x+11}{x^3+5x^2+3x+15}$.
\end{example}
\begin{equation*}
	x^2+5x^2+3x+15 = (x+5)(x^2+3)
\end{equation*}
so,
\begin{equation*}
	\frac{6x^2+21x+11}{x^3+5x^2+3x+15} = \frac{A_1}{x+5}+\frac{A_2x+B_2}{x^2+3}
\end{equation*}
Multiplying each side by the denominator,
\begin{equation*}
	6x^2+21x+11 = A_1(x^2+3)+(A_2x+B_2)(x+5)
\end{equation*}
at $x=-5$,
\begin{equation*}
	56 = 28A_1 \implies A_1 = 2
\end{equation*}
Now we'll use the previous result and another value for $x$. We can use $x=0$ to not have to worry about the $A_2$ term.\\
At $x=0$,
\begin{equation*}
	11 = 2(3) + (B_2)(5) \implies B_2 = 1
\end{equation*}
Now we'll use the previous 2 results to find $A_2$. $x=1$ is a good choice to keep the numbers small.\\
At $x=1$,
\begin{equation*}
	38 = 2(1+3)+(A_2+1)(6) \implies A_2 = 4
\end{equation*}
So,
\begin{equation*}
	\frac{6x^2+21x+11}{x^3+5x^2+3x+15} = \frac{2}{x+5}+\frac{4x+1}{x^2+3}
\end{equation*}