\subsubsection{Integrals}
The definite integral of a function $f(x)$ from $x=a$ to $x=b$ where $a \leq b$ is the area between $f(x)$ and the x-axis bounded by the lines $x=a$ and $x=b$ where area above the x-axis is positive, and area below the x-axis is negative. Written formally,
\begin{equation*}
\int_{a}^{b}{f(x) \mathrm{d}x} = \lim\limits_{h \to 0}{\sum_{n=1}^{\frac{b-a}{h}}}{f(a + (n-1)h) \cdot h}
\end{equation*}
We also define an indefinite integral, or antiderivative of $f(x)$, notated $F(x)$ where
\begin{equation*}
F^\prime(x) = f(x) \implies \int{f(x)\mathrm{d}x} = F(x)
\end{equation*}
Note that there are infinitely many such functions $F$, since adding a constant to $F$ does not affect its derivative. To notate this, we add a constant $C$ to the indefinite integral. Given an initial condition for $f$, we can solve for $C$.\\

\noindent
Below are some properties of the integral. Let $f$ and $g$ be functions of $x$ and $p$, $a$, $b$, and $c$ where $a < b < c$, and $f$ and $g$ are continuous on the closed interval $[a,c]$.
\begin{enumerate}[label=]
	\item Linearity
	\begin{equation*}
	\int{(pf \pm g) \mathrm{d}x} = p\int{f \mathrm{d}x} \pm \int{g \mathrm{d}x}
	\end{equation*}
	\item Flipped Bounds
	\begin{equation*}
	\int_{a}^{b}{f \mathrm{d}x} = -\int_{b}^{a}{f \mathrm{d}x}
	\end{equation*}
	\item Union of Intervals
	\begin{equation*}
	\int_{a}^{b}{f \mathrm{d}x} + \int_{b}^{c}{f \mathrm{d}x} = \int_{a}^{c}{f \mathrm{d}x}
	\end{equation*}
	\item Power Rule
	\begin{equation*}
	\int{x^n \mathrm{d}x} = \frac{x^{n+1}}{n+1} + C \text{, }n \neq -1
	\end{equation*}
	\item U-Substitution
	\begin{equation*}
	\int{\left(f^\prime\circ g\right) g^\prime \mathrm{d}x} = f\circ g+ C
	\end{equation*}
	\item Integration by Parts
	\begin{equation*}
	\int{f^\prime g \mathrm{d}x} = fg - \int{fg^\prime \mathrm{d}x}
	\end{equation*}
	\item Fundamental Theorem of Calculus
	\begin{equation*}
	\dd{x}\int_{a}^{x}{f(s) \mathrm{d}s} = f(x)
	\end{equation*}
\end{enumerate}
Using the definition of the integral and the above rules, we can find the indefinite integral of some common functions.
\begin{enumerate}[label=]
	\item 
	\begin{equation*}
	\int{\frac{1}{x} \mathrm{d}x} = \ln{\abs{x}} + C
	\end{equation*}
	\item 
	\begin{equation*}
	\int{\sin{x} \mathrm{d}x} = -\cos{x} + C
	\end{equation*}
	\item 
	\begin{equation*}
	\int{\cos{x} \mathrm{d}x} = \sin{x} + C
	\end{equation*}
	\item 
	\begin{equation*}
	\int{\tan{x} \mathrm{d}x} = -\ln{\abs{\cos{x}}} + C
	\end{equation*}
\end{enumerate}