\subsection{System to Higher Order}
Let's say we have the linear system
\begin{equation*}
	\vec{x}' = A\vec{x} + \vec{f}
\end{equation*}
Writing the system using $x_1, \ldots x_n$ as the components of $\vec{x}$, $f_1, \ldots f_n$ as the components of $\vec{f}$, and $a_{ij}$ as the entry in $A$ on the $i^{th}$ row and $j^{th}$ column,
\begin{equation*}
	\begin{cases}
		x_1' = a_{11}x_1 + \ldots a_{1n}x_n + f_1 \\
		\vdots \\
		x_n' = a_{n1}x_1 + \ldots a_{nn}x_n + f_n
	\end{cases}
\end{equation*}
Let's arbitrarily assign $x_1 = y$. This will allow us to find expressions for $x_2, \ldots x_n$ in terms of $y$ and its derivatives. When we find $x_n$ in these terms and equate $x_n'$ with what's given in the system, we'll have an linear $n^{th}$ order equation.

\ifodd\includeLinearSystemsExamples\begin{example}
	Convert the following system of equations to a single equation.
	\begin{equation*}
		\vec{x}' = \begin{bmatrix}
			1 & 1 & 1 \\
			1 & 0 & 1 \\
			0 & 1 & 1
		\end{bmatrix}\vec{x} + \begin{bmatrix}
			1 \\
			t \\
			t^2
		\end{bmatrix}
	\end{equation*}
\end{example}
\noindent
Expanding the system out of matrix form,
\begin{equation*}
	\begin{cases}
		x_1' = x_1 + x_2 + x_3 + 1 \\
		x_2' = x_1 + x_3 + t \\
		x_3' = x_2 + x_3 + t^2 
	\end{cases}
\end{equation*}
Assuming $x_1 = y$,
\begin{equation*}
	y' = y + x_2 + x_3 + 1
\end{equation*}
\begin{equation*}
	\implies x_2 = y' - y - x_3 - 1
\end{equation*}
Taking the derivative of $x_2$ and equating it with what's given in the system,
\begin{equation*}
	x_2' = y'' - y' - x_3' = y + x_3 + t
\end{equation*}
Solving for $x_3'$ and equating it with what's given in the system,
\begin{equation*}
	x_3' = y'' - y' - t - x_3 = y' - y + t^2 - 1
\end{equation*}
Putting this expression for $x_3'$ back into our expression for $x_2'$,
\begin{equation*}
	x_2' = y'' - 2y' + y - t^2 + 1 = y + x_3 + t
\end{equation*}
\begin{equation*}
	\implies x_3 = y'' - 2y' - t^2 - t + 1
\end{equation*}
Taking the derivative of $x_3$ and equating it to what's given in the system,
\begin{equation*}
	x_3' = y''' - 2y'' - 2t - 1 = y' - y - 1 + t^2
\end{equation*}
So, we have our order 3 equation,
\begin{equation*}
	y''' - 2y'' - y' + y = t^2 + 2t
\end{equation*}\fi

\noindent
The auxiliary polynomial of this higher order equation $p(r)$ and the characteristic polynomial $p(\lambda)$ of the linear system will have exactly the same roots.