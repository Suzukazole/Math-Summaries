\begin{example}
	Find the particular solution to the system
	\begin{equation*}
		\vec{x}' = \begin{bmatrix}
			1 & 2 \\
			3 & 2
		\end{bmatrix}\vec{x} + \begin{bmatrix}
			2e^t \\
			4e^t
		\end{bmatrix}
	\end{equation*}
\end{example}
\noindent
All entries in $\vec{f}$ are exponentials, so we'll guess that
\begin{equation*}
	\vec{x_p} = e^t\begin{bmatrix}
		A \\
		B
	\end{bmatrix} = e^t\vec{a}
\end{equation*}
Plugging our guess into the equation
\begin{equation*}
	\left(e^t\vec{a}\right)' = \begin{bmatrix}
		1 & 2 \\
		3 & 2
	\end{bmatrix}e^t\vec{a} + \begin{bmatrix}
		2e^t \\
		4e^t
	\end{bmatrix}
\end{equation*}
\begin{equation*}
	e^t\begin{bmatrix}
		A \\
		B
	\end{bmatrix} = e^t\begin{bmatrix}
		1 & 2 \\
		3 & 2
	\end{bmatrix}\begin{bmatrix}
		A \\ 
		B
	\end{bmatrix} + e^t\begin{bmatrix}
		2 \\
		4
	\end{bmatrix}
\end{equation*}
Dividing by $e^t$, which is never 0,
\begin{equation*}
	\begin{bmatrix}
		A \\
		B
	\end{bmatrix} = \begin{bmatrix}
		A + 2B \\
		3A + 2B 
	\end{bmatrix} + \begin{bmatrix}
		2 \\
		4
	\end{bmatrix}
\end{equation*}
Rearranging,
\begin{equation*}
	\begin{bmatrix}
		0 & 2 \\
		3 & 1
	\end{bmatrix}\begin{bmatrix}
		A \\
		B
	\end{bmatrix} = \begin{bmatrix}
		-2 \\
		-4
	\end{bmatrix} \implies \begin{bmatrix}
		A \\
		B
	\end{bmatrix} = \begin{bmatrix}
		-1 \\
		-1
	\end{bmatrix}
\end{equation*}
So,
\begin{equation*}
	\vec{x_p} = e^t\begin{bmatrix}
		-1 \\
		-1
	\end{bmatrix}
\end{equation*}

\noindent
It's possible that we'll have to solve for multiple vectors of scalars to find $\vec{x_p}$.
\begin{example}
	Find the particular solution to the system
	\begin{equation*}
		\vec{x}' = \begin{bmatrix}
			2 & 2 \\
			2 & 2
		\end{bmatrix}\vec{x} + \begin{bmatrix}
			-4\cos{t} \\
			-\sin{t}
		\end{bmatrix}
	\end{equation*}
\end{example}
Just like when dealing with $\sin$ and $\cos$ in single equations, we need both $\sin$ and $\cos$ in our guess.
\begin{equation*}
	\vec{x_p} = \sin{t}\vec{a} + \cos{t}\vec{b}
\end{equation*}
Plugging into the equation where $A$ is the matrix of all 2's,
\begin{equation*}
	\cos{t}\vec{a} - \sin(t)\vec{b} = \sin{t}A\vec{a} + \cos{t}A\vec{b} + \sin{t}\begin{bmatrix}
		0 \\
		-1
	\end{bmatrix} + \cos{t}\begin{bmatrix}
		-4 \\
		0
	\end{bmatrix}
\end{equation*}
This gives us a system
\begin{equation*}
	\begin{cases}
		-\vec{b} = A\vec{a} + \begin{bmatrix}
			0 \\
			-1
		\end{bmatrix} \\
		\vec{a} = A\vec{b} + \begin{bmatrix}
			-4 \\
			0
		\end{bmatrix}
	\end{cases}
\end{equation*}
Substituting,
\begin{equation*}
	-\vec{b} = A\left(A\vec{b} + \begin{bmatrix}
		-4 \\
		0
	\end{bmatrix}\right) + \begin{bmatrix}
		0 \\
		-1
	\end{bmatrix} \implies (A^2 + I)\vec{b} = -A\begin{bmatrix}
		-4 \\
		0
	\end{bmatrix} - \begin{bmatrix}
		0 \\
		-1
	\end{bmatrix}
\end{equation*}
\begin{equation*}
	A^2 + I = \begin{bmatrix}
		9 & 8 \\
		8 & 9
	\end{bmatrix} \text{ and } -A\begin{bmatrix}
	-4 \\
	0
	\end{bmatrix} - \begin{bmatrix}
	0 \\
	-1
	\end{bmatrix} = \begin{bmatrix}
		8 \\
		9
	\end{bmatrix}
\end{equation*}
So,
\begin{equation*}
	\begin{bmatrix}
		9 & 8 \\
		8 & 9
	\end{bmatrix}\vec{b} = \begin{bmatrix}
		8 \\
		9
	\end{bmatrix} \implies \vec{b} = \begin{bmatrix}
		0 \\
		1
	\end{bmatrix}
\end{equation*}
and
\begin{equation*}
	\vec{a} = A\vec{b} + \begin{bmatrix}
		-4 \\
		0
	\end{bmatrix} \implies \vec{a} = \begin{bmatrix}
		-2 \\
		2
	\end{bmatrix}
\end{equation*}
Finally we have $\vec{x_p}$,
\begin{equation*}
	\vec{x_p} = \sin{t}\begin{bmatrix}
		-2 \\
		2
	\end{bmatrix} + \cos{t}\begin{bmatrix}
		0 \\
		1
	\end{bmatrix}
\end{equation*}