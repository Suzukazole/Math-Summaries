\subsection{Variation of Parameters for Systems}
\noindent
Let $X$ be a fundamental matrix for the homogeneous system.
\begin{equation*}
	\vec{x_h}' = A\vec{x_h}.
\end{equation*}
That is,
\begin{equation*}
	\vec{x_h} = X\vec{c}
\end{equation*}
where
\begin{equation*}
	\vec{c} = \begin{bmatrix}
	C_1 \\
	\vdots \\
	C_n
	\end{bmatrix}
\end{equation*}
and the entries of matrix $A$ can be any continuous functions of $t$.

\noindent
We are looking for the particular solution $\vec{x_p}$, to the system
\begin{equation*}
	\vec{x} = A\vec{x} + \vec{f}
\end{equation*}
where $\vec{x_p}$ is of the form
\begin{equation*}
	\vec{x_p} = X\vec{v}
\end{equation*}
where $\vec{v}$ is a vector of functions of $t$ that we'll have to find.

\noindent
Differentiating $\vec{x_p}$,
\begin{equation*}
	\vec{x_p}' = X\vec{v}' + X'\vec{v}.
\end{equation*}
From the system we're trying to solve we know that
\begin{equation*}
	X\vec{v}' + X'\vec{v} = A(X\vec{v}) + \vec{f}.
\end{equation*}
Since $X' = AX$,
\begin{equation*}
	X\vec{v}' = \vec{f}.
\end{equation*}

\noindent
Since the columns of $X$ are always linearly independent, we know that $X^{-1}$ always exists.
Multiplying by $X^{-1}$,
\begin{equation*}
	\vec{v}' = X^{-1}\vec{f}.
\end{equation*}
Integrating with respect to $t$,
\begin{equation*}
	\vec{v} = \int{X^{-1}\vec{f} \mathrm{d}t}.
\end{equation*}
So,
\begin{equation*}
	\vec{x_p} = X\int{X^{-1}\vec{f} \mathrm{d}t},
\end{equation*}
and
\begin{equation*}
	\vec{x} = X\vec{c} + X\int{X^{-1}\vec{f} \mathrm{d}t}.
\end{equation*}

\begin{example}
	Find the general solution to the system by variation of parameters
	\begin{equation*}
		\vec{x}' = \begin{bmatrix}
			2 & -3 \\
			1 & -2
		\end{bmatrix}\vec{x} + \begin{bmatrix}
			e^{2t} \\
			1
		\end{bmatrix}
	\end{equation*}
	given the fundamental matrix
	\begin{equation*}
		X = \begin{bmatrix}
			3e^t & e^{-t} \\
			e^t & e^{-t}
		\end{bmatrix}.
	\end{equation*}
\end{example}
\noindent
Finding $\vec{x_h}$ is simply multiplying $X$ by a vector, so we'll save that for the end and focus on $\vec{x_p}$. First we need to find $X^{-1}$. The details are left out, but the process is identical to that for matrices of numbers.
\begin{equation*}
	X^{-1} = \begin{bmatrix}
		\frac{1}{2}e^{-t} & -\frac{1}{2}e^{-t} \\
		-\frac{1}{2}e^t & \frac{3}{2}e^t
	\end{bmatrix}.
\end{equation*}
Multiplying by $\vec{f}$.
\begin{equation*}
	X\vec{f} = \begin{bmatrix}
		\frac{1}{2}e^t - \frac{1}{2}e^{-t} \\
		-\frac{1}{2}e^{3t} + \frac{3}{2}e^t
	\end{bmatrix}.
\end{equation*}
Integrating with respect to $t$,
\begin{equation*}
	\int{X^{-1}\vec{f} \mathrm{d}t} = \begin{bmatrix}
		\frac{1}{2}e^t + \frac{1}{2}e^{-t} \\
		-\frac{1}{6}e^{3t} + \frac{3}{2}e^t
	\end{bmatrix}.
\end{equation*}
Note that although we are doing an indefinite integral, we don't have a constant (or in this case a vector) of integration.
Multiplying by $X$ to obtain $\vec{x_p}$,
\begin{equation*}
	\vec{x_p} = X\int{X^{-1}\vec{f} \mathrm{d}t} = \begin{bmatrix}
		\frac{4}{3}e^{2t} + 3 \\
		\frac{1}{3}e^{2t} + 2
	\end{bmatrix}.
\end{equation*}
Finding $\vec{x_h}$,
\begin{equation*}
	\vec{x_h} = X\vec{c} = \begin{bmatrix}
		3C_1e^t + C_2e^{-t} \\
		C_1e^t + C_2e^{-t}
	\end{bmatrix}.
\end{equation*}
Putting $\vec{x_h}$ and $\vec{x_p}$ together to get the general solution,
\begin{equation*}
	\vec{x} = \begin{bmatrix}
		3C_1e^{3t} + C_2e^{-t} + \frac{4}{3}e^{2t} + 3 \\
		C_1e^t + C_2e^{-t} + \frac{1}{3}e^{2t} + 2
	\end{bmatrix}.
\end{equation*}