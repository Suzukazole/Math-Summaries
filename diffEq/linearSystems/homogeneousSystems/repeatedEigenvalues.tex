\subsubsection{Repeated Eigenvalues}
\noindent
If $A$ has an eigenvalue with multiplicity $k$, that eigenvalue needs to generate $k$ fundamental solutions.
If this eigenvalue generates $k$ linearly independent eigenvectors, then the process is much like with distinct eigenvalues.
Otherwise, the matrix $A$ is defective.
\begin{theorem}
	If $\lambda$ is an eigenvalue with multiplicity $k$, and $\left\{\vec{v_1}, \ldots, \vec{v_k}\right\}$ are corresponding linearly independent eigenvectors, the the set of fundamental solutions generated by $\lambda$ is $\left\{e^{\lambda t}\vec{v_1}, \ldots, e^{\lambda t}\vec{v_k}\right\}$.
\end{theorem}

\begin{example}
	Find the general solution to the system
	\begin{equation*}
		\vec{x}' = \begin{bmatrix}
			0 & 1 & 1 \\
			1 & 0 & 1 \\
			1 & 1 & 0
		\end{bmatrix} \vec{x}.
	\end{equation*}
\end{example}
\noindent
Finding the eigenvalues by finding the roots of the characteristic polynomial of $A$,
\begin{equation*}
	p(\lambda) = \det{(A - \lambda I)} = -\lambda^3 + 3\lambda + 2 \implies \lambda = 2, -1, -1.
\end{equation*}
Finding the eigenvector for $\lambda = 2$,
\begin{equation*}
	(A - 2I)\vec{v} = \vec{0} \implies \vec{v} = C_1\begin{bmatrix}
		1 \\
		1 \\
		1
	\end{bmatrix}.
\end{equation*}
Finding the eigenvectors for $\lambda = -1$,
\begin{equation*}
	(A + I)\vec{v} = \vec{0} \implies \vec{v} = C_2\begin{bmatrix}
		-1 \\
		1 \\
		0
	\end{bmatrix} + C_3\begin{bmatrix}
		-1 \\
		0 \\
		1
	\end{bmatrix}.
\end{equation*}
So, our solution is
\begin{equation*}
	\vec{x} = C_1e^{2t}\begin{bmatrix}
		1 \\
		1 \\
		1
	\end{bmatrix} + C_2e^{-t}\begin{bmatrix}
		-1 \\
		1 \\
		0
	\end{bmatrix} + C_3e^{-t}\begin{bmatrix}
		-1 \\
		0 \\
		1
	\end{bmatrix}.
\end{equation*}