\subsubsection{Real Distinct Eigenvalues}
\noindent
Real, distinct eigenvalues are the simplest case, similar to real, distinct roots of an auxiliary equation.
\begin{theorem}
	Let $\left\{\lambda_1, \ldots, \lambda_n\right\}$ be the set of unique eigenvalues and $\left\{\vec{v_1}, \ldots, \vec{v_n}\right\}$ be the corresponding set of non-zero, unique, eigenvectors for an $n \times n$ matrix $A$. Then the set of fundamental solutions to the system $\vec{x}' = A\vec{x}$ is $\left\{e^{\lambda_1 t}\vec{v_1}, \ldots, e^{\lambda_n t}\vec{v_n} \right\}$
\end{theorem}

\begin{example}
	Find the general solution to the system
	\begin{equation*}
		\vec{x}' = \begin{bmatrix}
			3 & 0 & 0 \\
			-5 & -2 & 0 \\
			1 & 1 & -1
		\end{bmatrix} \vec{x}.
	\end{equation*}
\end{example}
\noindent
For a triangular matrix, the eigenvalues are simply the diagonal entries.
\begin{equation*}
	\lambda = 3, -2, -1.
\end{equation*}
Finding the eigenvector for $\lambda = 3$,
\begin{equation*}
	(A - 3I)\vec{v} = \vec{0} \implies \vec{v} = C_1\begin{bmatrix}
		-1 \\
		1 \\
		0
	\end{bmatrix}.
\end{equation*}
Finding the eigenvector for $\lambda = -2$,
\begin{equation*}
	(A + 2I)\vec{v} = \vec{0} \implies \vec{v} = C_2\begin{bmatrix}
		0 \\
		-1 \\
		1
	\end{bmatrix}.
\end{equation*}
Finding the eigenvector for $\lambda = -1$,
\begin{equation*}
	(A + I)\vec{v} = \vec{0} \implies \vec{v} = C_3\begin{bmatrix}
		0 \\
		0 \\
		1
	\end{bmatrix}.
\end{equation*}
So, our general solution is
\begin{equation*}
	\vec{x} = C_1e^{3t}\begin{bmatrix}
		-1 \\
		1 \\
		0
	\end{bmatrix} + C_2e^{-2t}\begin{bmatrix}
		0 \\
		-1 \\
		1
	\end{bmatrix} + C_3e^{-t}\begin{bmatrix}
		0 \\
		0 \\
		1
	\end{bmatrix}.
\end{equation*}