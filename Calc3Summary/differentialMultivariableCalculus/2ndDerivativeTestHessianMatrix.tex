\subsubsection{Second Derivative Test \& Hessian Matrix}
\noindent
Recall from single-variable calculus that if we have a critical point (derivative is 0), we can find whether the point is a minimum or maximum by using the second derivative test. If the second derivative is $<0$ at the critical point, then the critical point is a maximum, and if the second derivative is $>0$ at the critical point, then the critical point is a minimum. A similar process works in higher dimensions.\\

\noindent
In higher dimensions:
\begin{itemize}
	\item If $f_{xx}$ and $f_{yy}$ are $>0$ at a critical point, then the critical point is a minimum.
	\item If $f_{xx}$ and $f_{yy}$ are $<0$ at a critical point, then the critical point is a maximum.
	\item If $f_{xx}$ and $f_{yy}$ don't agree on sign, then the point is a saddle point.
\end{itemize}

\noindent
Although there are only 3 options for what a critical point on a surface can be, we need a process of identifying them hat works for for higher dimensional objects too.

[INSERT IMAGE]

\begin{definition}
	The Hessian Matrix (for $f(x,y)$): $$H=\begin{bmatrix}f_{xx} & f_{xy} \\d_{yx} & f_{yy} \end{bmatrix}$$
\end{definition}
\noindent
Note that $\det{H}=f_{xx}f_{yy}-f_{xy}^2$.\\

\noindent
If $(x_0,y_0)$ is a critical point:
\begin{itemize}
	\item $\det{H(x_0,y_0)>0}$ means $(x_0,y_0)$ is an extrema. 
	\begin{itemize}
		\item If $f_{xx}(x_0,y_0)>0$, then $(x_0,y_0)$ is a minima.
		\item If $f_{xx}(x_0,y_0)<0$, then $(x_0,y_0)$ is a maxima.
	\end{itemize}
	\item $\det{H(x_0,y_0)}<0$ means $(x_0,y_0)$ is a saddle point.
	\item $\det{H(x_0,y_0)}=0$ means the 2nd derivatives test is inconclusive.
\end{itemize}
