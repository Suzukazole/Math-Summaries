\subsubsection{Area of a Closed Region}
\noindent
The area inside of $D$ is $A=\iint\limits_{D}{\mathrm{d}x\mathrm{d}y}$.\\
$=\iint\limits_{D}{\left(\nabla\times\vec{F}\right)\mathrm{d}x\mathrm{d}y}$ if $\nabla\times\vec{F}=1$. One such vector field is $\vec{F}=\langle -y/2, x/2\rangle$.\\
$=\oint_{C}{(-y/2)\mathrm{d}x+(x/2)\mathrm{d}y}$ by Green's Theorem for Circulation.\\
$=\frac{1}{2}\oint\limits_{C}{xy^\prime-yx^\prime}$\\
So, if we have some counter-clockwise oriented parametric function $(x(t),y(t))$ where $t_0\leq t\leq t_1$ that parameterizes $C$, then $A=\frac{1}{2}\int_{t_0}^{t_1}{\left(x\frac{\mathrm{d}y}{\mathrm{d}t}-y\frac{\mathrm{d}x}{\mathrm{d}t}\right)\mathrm{d}t}$.\\

\noindent
We can also choose different vector field where $\nabla\times\vec{F}=1$ so that\\ $A=\oint\limits_{C}{xy^\prime}=\oint\limits_{C}{yx^\prime}$.\\

\noindent
For example, let's compute the area of a circle with radius $R$.\\
\indent
We'll parameterize the circle as $(R\cos{t},R\sin{t}),t\in [0,2\pi]$.\\
\indent
$A=\frac{1}{2}\int_{0}^{2\pi}{((R\cos{t})(R\cos{t})-(R\sin{t})(-R\sin{t}))\mathrm{d}t}$\\
\indent
$=\frac{1}{2}\int_{0}^{2\pi}{R^2\mathrm{d}t}$\\
\indent
$=\frac{1}{2}R^2 2\pi$\\
\indent
$A=\pi R^2$