\subsection{Green's Theorem for Circulation}
\begin{theorem}[Green's Theorem for Circulation]
	Let $C$ be a closed, counter-clockwise oriented curve in $\mathbb{R}^2$. For any differentiable vector field $\vec{F}(x,y)=\langle Q(x,y), R(x,y)\rangle$, $\oint\limits_{C}{R\mathrm{d}x+Q\mathrm{d}y}=\iint\limits_{D}{\left(\frac{\partial Q}{\partial x}-\frac{\partial R}{\partial y}\right)\mathrm{d}x\mathrm{d}y}$, where $D$ is the interior of $C$.
\end{theorem}
\noindent
Or, in more modern notation,
$$\oint\limits_{C}{\vec{F}\cdot\mathrm{d}\vec{r}}=\iint\limits_{D}{\nabla\times\vec{F}\mathrm{d}A}$$

[INSERT IMAGE]

\noindent
This is saying that summing up the interior of the derivative is the to summing up the boundary of the function, very similar to the FTC.\\
One can think of this in a physical sense as saying that the work done by the vector field in moving a particle counter-clockwise on $C$ is equal to the rotation (curl) inside of  $C$ ($D$).\\
This theorem also relates the idea of path independence and curl of a conservative vector field that we proved the 2D case for. The left side shows path independence and will be 0 for conservative vector fields, and the right side shows curl, which will also be 0 for conservative vector fields.

\begin{proof}[Partial Proof]
	We will prove the 2D case, but the underlying argument is easily generalized.\\
	$$\oint\limits_{C}{\vec{F}\cdot\mathrm{d}\vec{r}}=\oint\limits_{c}{\langle P(x,y), 0\rangle\cdot\mathrm{d}\vec{r}}+\oint\limits_{C}{\langle 0, Q(x,y)\rangle\cdot\mathrm{d}\vec{r}}$$\\
	If $D$ is convex, we an break $C$ into 2 curves on the same interval, $C_1$ and $C_2$ such that $C_1:\vec{r}\langle x, h_1(x)\rangle$ and $C_2:\vec{r}\langle x,h_2(x)\rangle$, where $x\in[a,b]$.
	
	[INSERT IMAGE]
	
	$$=\int_{a}^{b}{P(x,h_1(x))\mathrm{d}x}-\int_{a}^{b}{P(x,h_2(x))\mathrm{d}x}$$
	$$=-\int_{a}^{b}{P(x,h_2(x))-P(x,h_1(x))\mathrm{d}x}$$
	$$=-\int_{a}^{b}{P(x,y)\rvert_{y=h_1(x)}^{y=h_2(x)}\mathrm{d}x}$$
	$$=-\int_{a}^{b}{\int_{h_1(x)}^{h_2(x)}{\frac{\partial P}{\partial y}\mathrm{d}y}\mathrm{d}x}$$
	$$=-\iint\limits_{D}{\frac{\partial P}{\partial y}\mathrm{d}A}+\iint\limits_{D}{\frac{\partial Q}{\partial x}\mathrm{d}A}$$
	$$=\iint\limits_{D}{\nabla\times\vec{F}\mathrm{d}A}$$
\end{proof}

\noindent
For example, let's use Green's Theorem for Circulation to compute $\oint\limits_{C}{\vec{F}\cdot\mathrm{d}\vec{r}}$ where $\vec{F}=\langle x^2, xy+x^2\rangle$ and $C$ is the unit circle.\\
\indent
$\oint\limits_{C}{\vec{F}\cdot\mathrm{d}\vec{r}}=\iint\limits_{D}{\nabla\times\vec{F}\mathrm{d}A}$\\
\indent
$\nabla\times\vec{F}=y+2x$\\
Since $C$ is the unit circle, we'll evaluate the integral using polar coordinates. So $\mathrm{d}A=r\mathrm{d}r\mathrm{d}\theta$.\\
$\iint\limits_{D}{\nabla\times\vec{F}\mathrm{d}A}=\int_{0}^{2\pi}{\int_{0}^{1}{(r\sin{\theta}+2r\cos{\theta})r\mathrm{d}r}\mathrm{d}\theta}$\\
\indent
$=\int_{0}^{2\pi}{\left(\frac{\sin{\theta}}{3}+\frac{2\cos{\theta}}{3}\right)\mathrm{d}\theta}=0$\\
\indent
Note that although this particular circulation is 0, we know that the vector field is not conservative because the curl is not 0.

\subsubsection{Area of a Closed Region}
\noindent
The area inside of $D$ is 
\begin{align*}
	A &= \iint\limits_{D}{\mathrm{d}x\mathrm{d}y} \\
	&= \iint\limits_{D}{\left(\nabla \times \vec{F}\right)\mathrm{d}x\mathrm{d}y}
\end{align*}
if $\nabla \times \vec{F} = 1$. One such vector field is $\vec{F} = \langle -y/2, x/2 \rangle$.
\begin{equation*}
	= \oint_{C}{(-y/2)\mathrm{d}x + (x/2)\mathrm{d}y}
\end{equation*}
by Green's Theorem for Circulation.
\begin{equation*}
	= \frac{1}{2}\oint\limits_{C}{xy^\prime - yx^\prime}.
\end{equation*}
So, if we have some counter-clockwise oriented parametric function $(x(t), y(t))$ where $t_0 \leq t \leq t_1$ that parameterizes $C$, then 
\begin{equation*}
	A = \frac{1}{2}\int_{t_0}^{t_1}{\left(x\frac{\mathrm{d}y}{\mathrm{d}t} - y\frac{\mathrm{d}x}{\mathrm{d}t}\right)\mathrm{d}t}
\end{equation*}

\noindent
We can also choose different vector field where $\nabla \times \vec{F} = 1$ so that
\begin{equation*}
	 A = \oint\limits_{C}{xy^\prime} = \oint\limits_{C}{yx^\prime}
\end{equation*}

\noindent
For example, let's compute the area of a circle with radius $R$.
We'll parameterize the circle as $(R\cos{t}, R\sin{t}), t \in [0,2\pi]$.\\
\begin{align*}
	A &= \frac{1}{2}\int_{0}^{2\pi}{((R\cos{t})(R\cos{t}) - (R\sin{t})(-R\sin{t}))\mathrm{d}t} \\
	&= \frac{1}{2}\int_{0}^{2\pi}{R^2\mathrm{d}t} \\
	&= \frac{1}{2}R^2 2\pi \\
	&= \pi R^2
\end{align*}