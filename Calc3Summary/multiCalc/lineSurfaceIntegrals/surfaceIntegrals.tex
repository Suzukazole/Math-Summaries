\section{Surface Integrals}
\noindent
We can parameterize any surface as $\vec{r}(u,v)=\langle x(u,v), y(u,v), z(u,v)\rangle$ (or using some other coordinate system). This is because we are simple taking one 2D surface, the uv-plane, and transforming it into another surface in such a way that areas ear each other in the uv-plane are near each other on the surface. This fact that areas stay close to each other allows us to make statement about the complicated surface while working with the simple uv-plane.

[INSERT IMAGE]

\noindent
The change of the surface in the $u$ direction is $\frac{\partial\hat{r}}{\partial u}\mathrm{d}u$\\
The change of the surface in the $v$ direction is $\frac{\partial\hat{r}}{\partial v}\mathrm{d}v$\\
So, the area of the surface in relation to $u$ and $v$ is the area of the paraellogram spanned by these two surface: a cross product.\\
$$\mathrm{d}s=\norm{\left<\frac{\partial\vec{r}}{\partial u}\mathrm{d}u\right>\times\left<\frac{\partial\vec{r}}{\partial v}\mathrm{d}v\right>}=\norm{\vec{r_u}\times\vec{r_v}}\mathrm{d}u\mathrm{d}v$$

\begin{definition}
	For a surface $S$ parameterized by $\vec{r}(u,v)$ and $(u,v)\subset D$, the surface area is\\
	$$A(S)=\iint\limits_{S}{\mathrm{d}s}=\iint\limits_{D}{\norm{\vec{r_u}\times\vec{r_v}}\mathrm{d}u\mathrm{d}v}$$
\end{definition}

\subsection{Surface Integrals of Scalar Functions}
\begin{definition}
	The surface integral of a scalar function $f(x,y,z)$ is
	\begin{equation*}
		\iint\limits_{S}{f(x,y,z)\mathrm{d}s} = \iint\limits_{D}{(f\circ\vec{r})\norm{\vec{r_u} \times \vec{r_v}}\mathrm{d}A}
	\end{equation*}
\end{definition}

\noindent
Let’s apply a surface integral to a real problem. Let the cap of the sphere or radius $8\text{m}$ centered at the origin between $z = 7$ and $z = 8$ have a charge density $\sigma(x,y,z) = z \text{ } \mu \text{C}/ \text{m}^2$. Find the total charge $Q$ on the cap.

[INSERT IMAGE]

\indent
We will parameterize the cap using spherical coordinates.\\
\indent
$D = \left\{(\rho, \theta, \phi) \mid \rho=8, 0 \leq \theta \leq 2\pi, 0 \leq \phi \leq \cos^{-1}{\left(\frac{7}{8}\right)}\right\}$\\
\indent
$\vec{r}(\theta,\phi) = \langle 8\sin{\phi}\cos{\theta}, 8\sin{\phi}\sin{\theta}, 8\cos{\phi}\rangle$\\
\indent
$\vec{r_\theta} = \langle -8\sin{\phi}\sin{\theta}, 8\sin{\phi}\cos{\theta}, 0\rangle$, $\vec{r_\phi} = \langle 8\cos{\phi}\cos{\theta}, 8\cos{\phi}\sin{\theta}, -8\sin{\phi}\rangle$\\
\indent
$\sigma\circ\vec{r} = 8\cos{\phi}$\\
\indent
$\norm{\vec{r_\theta} \times \vec{r_\phi}} = 64\sin{\phi}$\\
\indent
$Q = \int_{0}^{2\pi}{\int_{0}^{\cos^{-1}{7/8}}{64\sin{\phi} \cdot8 \mathrm{d}\phi\mathrm{d}\theta}}$\\
\indent
$= 8^3 2\pi \int_{0}^{\cos^{-1}{7/8}}{\sin{\phi}\mathrm{d}\phi}$\\
\indent
$= 8^3 2\phi(-\cos{\phi})\rvert_{0}^{cos^{-1}{7/8}} = 8^3 2\pi\left(-7/8 + 1\right) = 128\pi\text{ }\mu\text{C}$
\subsection{Surface Integrals of Vector Fields}
\begin{definition}
	The surface integral of a vector field $\vec{F}$ through a surface $S$ is 
	\begin{equation*}
		\iint\limits_{S}{\vec{F} \cdot \hat{n}\mathrm{d}s}
	\end{equation*}
	where $\hat{n}$ is a unit vector normal to the surface.
	This integral can also be written as 
	\begin{equation*}
		\iint\limits_{S}{\vec{F} \cdot \mathrm{d}\vec{s}}
	\end{equation*}
\end{definition}

\subsubsection{Flux}
\noindent
We can rewrite the surface integral because $\hat{n} = \frac{\vec{r_u} \times \vec{r_v}}{\norm{\vec{r_u} \times \vec{r_v}}}$ and $\mathrm{d}s = \norm{\vec{r_u} \times \vec{r_v}}$.

[INSERT IMAGE]

\noindent
So, $\iint\limits_{S}{\vec{F} \cdot \mathrm{d}s} = \iint\limits_{D}{\left(\vec{F}\circ\vec{r}\right) \cdot \left(\vec{r_u} \times \vec{r_v}\right)\mathrm{d}u\mathrm{d}v}$, where $D$ is the domain of $S$ in uv-space (a uv-plane), and $\vec{r}(u,v)$ parameterizes the surface $S$. This quantity is called the "directed surface area integral" or "flux" through $S$.\\

\noindent
Flux tells us how much a vector field penetrates a surface. If the field is parallel to the surface (perpendicular to the normal vector) then the flux is 0. As the field and normal vector to the surface become more aligned, the flux increases. \\

\noindent
Flux has many practical applications. One of the fundamental equations governing electricity and magnetism talks about electric flux: the amount of an electric field that goes through a surface. We will investigate this equation later.\\

\noindent
For example, let an electric field be $\vec{E} = \langle x, y, 0 \rangle \text{ N}/ \text{C}$. Compute the electric flux, $\Phi_{E}$, through the portion of the paraboloid $z=25-x^2-y^2$ above the xy-plane. Assume a unit distance of 1 meter.
\indent
We will take advantage of a paraboloid's radial symmetry and use cylindrical coordinates.\\
\indent
$\vec{r}(u,v) = \langle u\cos{v}, u\sin{v}, 25 - u^2 \rangle$ where $0 \leq u \leq 5$ and $0 \leq v \leq 2\pi$.\\
\indent
$\vec{r_u} = \langle \cos{v}, \sin{v}, -2u \rangle$ and $\vec{r_v}=\langle -u\sin{v}, u\cos{v}, 0 \rangle$.\\
\indent
$\vec{r_u} \times \vec{r_v} = \langle 2u^2\cos{v}, 2u^2\sin{v}, 0 \rangle$\\
\indent
$\vec{E}\circ\vec{r} = \langle u\cos{v}, u\sin{v}, 0 \rangle$\\
\indent
$\left(\vec{E}\circ\vec{r}\right) \cdot (\vec{r_u}) \times \vec{r_v}) = 2u^3$\\
\indent
$\Phi_{E} = \int_{0}^{2\pi}{\int_{0}^{5}{2u^3\mathrm{d}u}\mathrm{d}v} = 2\pi\left(\frac{u^4}{2}\right)\rvert_{0}^{5} = 625\text{ Nm}^2/\text{C}$