\subsection{Planes}
\noindent
A plane can also be formed using a point in the plane, $P$, and a vector perpendicular to the plane, $\vec{n}$. All vectors $\langle x,y,z \rangle$ that originate from $P$ and remain in the plane must be perpendicular to $\vec{n}$, so their dot product with $\vec{n}$ would be 0. So, the point-normal form of a plane is $\vec{n}\cdot\left(\langle x,y,z \rangle - \vec{P}\right)=0$.\\
\small{Note: Conventionally, $\vec{n}$ is a unit vector, $\hat{n}$.}

[INSERT IMAGE]

\noindent
One can also construct a plane from 3 non-collinear points in the plane. One can still take advantage of point-normal form here by choosing 1 point to be $P_0$ and drawing vectors from this point to the two other points. The cross product of these two vectors is $\vec{n}$. $\left(\left(\vec{P_1}-\vec{P_0}\right)\times\left(\vec{P_2}-\vec{P_0}\right)\right)\cdot\left(\langle x,y,z \rangle - \vec{P_0}\right)=0$ where $P_0$, $P_1$, and $P_2$ are the three points in the plane.\\

\noindent
One can also construct a plane from a point in the plane, $P_0$, and a line in the plane, $\vec{r}(t)=\vec{P_1}+t\vec{v}$, that doesn't pass through $P_0$. One can get this setup into point-normal form by choosing a an output of $\vec{r}(t)$, like $\vec{P_1}$, and constructing a vector that points from $\vec{P_1}$ to $\vec{P_0}$, $\vec{P_1}-\vec{P_0}$, and crossing this with $\vec{v}$ to find $\vec{n}$. $\left(\vec{v}\times\left(\vec{P_1}-\vec{P_0}\right)\right)\cdot\left(\langle x,y,z \rangle - \vec{P_0}\right)=0$.\\

\noindent
One can also construct a plane from two intersecting lines, $\vec{r_1}(t)=\vec{P_0}+t\vec{v_1}$ and $\vec{r_2}(t)=\vec{P_0}+t\vec{v_2}$, where $P_0$ is where the two lines intersect. One can cross $\vec{v_1}$ with $\vec{v_2}$ to get the normal vector.
$\left(\vec{v_1}\times\vec{v_2}\right)\cdot\left(\langle x,y,z \rangle - \vec{P-0}\right)=0$.