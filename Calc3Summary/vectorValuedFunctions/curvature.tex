\subsection{Curvature}
Curvature is 1 divided by the radius of the circle that best approximates the curve at a point. Tighter turns have smaller radii and higher curvature.\\
We can use $\hat{T}$ to find the curvature at a point on $\vec{r}(t)$.\\
$$\kappa(t)=\norm{\frac{\mathrm{d}\hat{T}}{\mathrm{d}s}}=\norm{\frac{\mathrm{d}\hat{T}}{\mathrm{d}t}\left(\frac{\mathrm{d}s}{\mathrm{d}t}\right)^{-1}}=\norm{\frac{\mathrm{d}\hat{T}}{\mathrm{d}t}}\frac{1}{v(t)}$$

\noindent
For example, let's find $\kappa(t)$ for the circle in the yz-plane: $\vec{r}(t)=\langle 7, R\sin{t}, R\cos{t} \rangle$.\\
\indent
$\vec{r^\prime}(t)=\langle 0, R\cos{t}, -R\sin{t}\rangle$ and $v(t)=\sqrt{0^2+(R\cos{t})^2+(-R\sin{t})^2}=R$\\
\indent
$\hat{T}(t)=\frac{1}{R}\vec{r}(t)=\langle0,\cos{t},-\sin{t}\rangle$, $\frac{\mathrm{d}\hat{T}}{\mathrm{d}t}=\langle 0,\sin{t},-\cos{t}\rangle$, $\norm{\frac{\mathrm{d}\hat{T}}{\mathrm{d}t}}=1$\\
\indent
$\kappa(t)=\frac{1}{R}$. This relationship is true for all circles.