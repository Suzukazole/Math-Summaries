\subsection{Dot Products}
\noindent
A dot product is a way of multiplying two vectors so that the result is a scalar. $\vec{a}\cdot\vec{b}=\norm{\vec{a}}\norm{\vec{b}}\cos{\theta}$ where $\theta$ is the angle between $\vec{a}$ and $\vec{b}$. One way to think of the dot product is as a measure of how much two vectors point in the same direction.\\
We can also show using the law of cosines that $\vec{a}\cdot\vec{b}=a_1b_1+a_2b_2+...+a_nb_n$.\\
Knowing the lengths of two vectors and their dot product we can calculate the angle between them as\\
\begin{center}
	$\theta=\cos^{-1}{\left(\frac{\vec{a}\cdot\vec{b}}{\norm{\vec{a}}\norm{\vec{b}}}\right)}$
\end{center}

[INSERT IMAGE]

\noindent
Although similar to scalar multiplication, dot products have some properties that set them apart.
\begin{itemize}
	\item Commutative: $\vec{a}\cdot\vec{b}=\vec{b}\cdot\vec{a}$, the same as scalar multiplication.
	\item Distributive: $\vec{a}\cdot\left(\vec{b}+\vec{c}\right)=\vec{a}\cdot\vec{b}+\vec{a}\cdot\vec{c}$, the same as scalar multiplication.
	\item NOT Associative: $\left(\vec{a}\cdot\vec{b}\right)\cdot\vec{c}$ is a nonsense expression. However, like scalar multiplication, dot products are scalar associative.\\ 
	$\left(c\cdot\vec{a}\right)\cdot\vec{b}=\vec{a}\cdot\left(c\cdot\vec{b}\right)$
\end{itemize}